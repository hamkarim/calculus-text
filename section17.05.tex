\section{Second Order Homogeneous Equations}{}{}
\secrdef{sec:second order homogeneous}
\nobreak
A second order differential equation is one containing the second
derivative. These are in general quite complicated, but one fairly
simple type is useful: the second order linear equation with constant
coefficients. 

\example Consider the intial value problem $\ddot y-\dot y-2y=0$,
$y(0)=5$, $\dot y(0)=0$. We make an inspired guess: might there be a
solution of the form $\ds e^{rt}$? This seems at least plausible,
since in this case $\ds\ddot y$, $\ds\dot y$, and $y$ all
involve $\ds e^{rt}$. 

If such a function is a solution then
$$\eqalign{
r^2 e^{rt}-r e^{rt}-2e^{rt}&=0\cr
e^{rt}(r^2-r-2)&=0\cr
(r^2-r-2)&=0\cr
(r-2)(r+1)&=0,\cr}
$$
so $r$ is $2$ or $-1$. Not only are $\ds f=e^{2t}$ and $\ds g=e^{-t}$
solutions, but notice that $\ds y=Af+Bg$ is also, for any constants $A$
and $B$:
$$\eqalign{
(Af+Bg)''-(Af+Bg)'-2(Af+Bg)&=Af''+Bg''-Af'-Bg'-2Af-2Bg\cr
&=A(f''-f'-2f)+B(g''-g'-2g)\cr
&=A(0)+B(0)=0.\cr}
$$
Can we find $A$ and $B$ so that this is a solution to the initial
value problem? Let's substitute:
$$
5=y(0)=Af(0)+Bg(0)=Ae^0+Be^0=A+B
$$
and 
$$0=\dot y(0)=Af'(0)+Bg'(0)=A2e^{0}+B(-1)e^0=2A-B.$$
So we need to find $A$ and $B$ that make both $5=A+B$ and $0=2A-B$
true. This is a simple set of simultaneous equations: solve $B=2A$,
substitute to get $5=A+2A=3A$. Then $A=5/3$ and $B=10/3$, and the
desired solution is $\ds (5/3)e^{2t}+(10/3)e^{-t}$. You now see why
the initial condition in this case included both $y(0)$ and $\dot
y(0)$: we needed two equations in the two unknowns $A$ and $B$
\endexample

You should of course wonder whether there might be other solutions;
the answer is no. We will not prove this, but here is the theorem that
tells us what we need to know:

\thm Given 
\thmrdef{thm:solns to second order homogeneous}
the differential equation $\ds a\ddot y+b\dot y+cy=0$, $a\not=0$,
consider the quadratic polynomial $ax^2+bx+c$, called the
     {\dfont characteristic polynomial\index{characteristic
         polynomial}}. Using the quadratic formula, this polynomial
     always has one or two roots, call them $r$ and $s$.  The general
     solution of the differential equation is:

\beginlist
\item{(a)} $\ds y=Ae^{rt}+Be^{st}$, if the roots $r$ and $s$ are real
  numbers and $r\not=s$.

\item{(b)} $\ds y=Ae^{rt}+Bte^{rt}$, if $r=s$ is real.

\item{(c)} $\ds y=A\cos(\beta t)e^{\alpha t}+B\sin(\beta t)e^{\alpha t}$, 
if the roots $r$ and $s$ are complex numbers $\alpha+\beta i$ and
$\alpha-\beta i$.

\endlist
\endthmnoproof

\example Suppose 
\thmrdef{example:damped spring oscillation}
a mass $m$ is hung on a spring with spring
constant $k$. If the spring is compressed or stretched and then
released, the mass will oscillate up and down. Because of friction,
the oscillation will be damped: eventually the motion will cease. The
damping will depend on the amount of friction; for example, if the
system is suspended in oil the motion will cease sooner than if the
system is in air. Using some simple physics, it is not hard to see
that the position of the mass is described by this differential
equation:
$\ds m\ddot y+b\dot y+ky=0$. Using $m=1$, $b=4$, and $k=5$ we find the
motion of the mass. The characteristic polynomial is 
$x^2+4x+5$ with roots $(-4\pm\sqrt{16-20})/2=-2\pm i$. Thus the
general solution is
$\ds y=A\cos(t)e^{-2t}+B\sin(t)e^{-2t}$.
Suppose we know that $y(0)=1$ and $\dot y(0)=2$. Then as before we
form two simultaneous equations: from $y(0)=1$ we get
$1=A\cos(0)e^0+B\sin(0)e^0=A$. For the second we compute
$$\ddot y=-2Ae^{-2t}\cos(t)+Ae^{-2t}(-\sin(t))-2Be^{-2t}\sin(t)+
Be^{-2t}\cos(t),$$
and then
$$2=-2Ae^0\cos(0)-Ae^0\sin(0)-2Be^0\sin(0)+Be^0\cos(0)
=-2A+B.$$
So we get $A=1$, $B=4$, and $\ds y=\cos(t)e^{-2t}+4\sin(t)e^{-2t}$.

Here is a useful trick that makes this easier to understand: We have
$\ds y=(\cos t+4\sin t)e^{-2t}$. The expression $\cos t+4 \sin t$ is a
bit reminiscent of the trigonometric formula
$\cos(\alpha-\beta)=\cos(\alpha)\cos(\beta)+\sin(\alpha)\sin(\beta)$
with $\alpha=t$.
Let's rewrite it a bit as
$$\sqrt{17}\left({1\over\sqrt{17}}\cos t + {4\over\sqrt{17}}\sin t\right).$$
Note that $\ds (1/\sqrt{17})^2+(4/\sqrt{17})^2=1$, 
which means that there is an angle
$\beta$ with $\ds \cos\beta=1/\sqrt{17}$ and 
$\ds \sin\beta=4/\sqrt{17}$ (of course, $\beta$ may not be a ``nice'' angle). Then
$$\cos t+4\sin t = \sqrt{17}\left(\cos t\cos \beta+\sin\beta\sin t\right)
=\sqrt{17}\cos(t-\beta).$$
Thus, the solution may also be written
$\ds y=\sqrt{17}e^{-2t}\cos(t-\beta)$.
This is a cosine curve  that has been shifted $\beta$ to the
right; the $\ds \sqrt{17}e^{-2t}$ has the effect of diminishing the
amplitude of the cosine as $t$ increases; see
figure~\xrefn{fig:damped oscillation}. The oscillation is damped very
quickly, so in the first graph it is not clear that this is an
oscillation. The second graph shows a restricted range for $t$.
\endexample

Other physical systems that oscillate can also be described by such
differential equations. Some electric circuits, for example, generate
oscillating current.

\figure
\texonly
\hbox to \hsize{\hfill
\def\yarrow{-- +(-1.5pt,-3pt) +(0pt,0pt) -- +(1.5pt,-3pt) +(0pt,0pt)}
\def\xarrow{-- +(-3pt,-1.3pt) +(0pt,0pt) -- +(-3pt,1.5pt) +(0pt,0pt) }
\tikzpicture[domain=0:3,x=1.2cm,y=3cm]
\draw (0,0) -- (5.2,0) \xarrow node [right] {$x$};
\draw (0,0) -- (0,1.3) \yarrow node [above] {$y$};
\gpad
\draw[color=black] plot[smooth,id=\the\gpnum,domain=0:5] function{sqrt(17)*exp(-2*x)*cos(x-asin(4/sqrt(17)))};
\foreach \x in {1,2,3,4,5} \draw (\x,0) -- (\x,-2pt) node[anchor=north] {\eightpoint $\x$};
\foreach \y in {0,1} \draw (0,\y) -- (-2pt,\y) node[anchor=east]
         {\eightpoint $\y$};
\endtikzpicture
\hfill
\tikzpicture[domain=2.5:5,x=1.6cm,y=120cm]
\draw (2.5,0) -- (5.2,0) \xarrow node [right] {$x$};
\draw (2.5,0) -- (2.5,0.03) \yarrow node [above] {$y$};
\gpad
\draw[color=black] plot[smooth,id=\the\gpnum,domain=2.5:5] function{sqrt(17)*exp(-2*x)*cos(x-asin(4/sqrt(17)))};
\foreach \x in {3,4,5} \draw (\x,0) -- (\x,-2pt) node[anchor=north] {\eightpoint $\x$};
\foreach \y in {0,0.01,0.02} \draw (2.5,\y) -- (2.4,\y) node[anchor=east] {\eightpoint $\y$};
\endtikzpicture
\hfill}\endtexonly
\figrdef{fig:damped oscillation}
\htmlfigure{DE-damped_oscillation.html}
\begincaption
Graph of a damped oscillation.
\endcaption
\endfigure

\example Find the solution to the intial value problem
$\ds\ddot y-4\dot y+4y=0$, $y(0)=-3$, $\dot y(0)=1$. The
characteristic polynomial is $x^2-4x+4=(x-2)^2$, so there is one root,
$r=2$, 
and the general solution is $\ds Ae^{2t}+Bte^{2t}$. Substituting
$t=0$ we get $-3=A+0=A$. The first derivative is
$\ds 2Ae^{2t}+2Bte^{2t}+Be^{2t}$; substituting $t=0$ gives
$1=2A+0+B=2A+B=2(-3)+B=-6+B$, so $B=7$. The solution is
$\ds -3e^{2t}+7te^{2t}$.
\endexample

\exercises

\exercise Verify that the function in part (a) of
theorem~\xrefn{thm:solns to second order homogeneous} is a solution to
the differential equation $\ds a\ddot y+b\dot y+cy=0$.
\endexercise

\exercise Verify that the function in part (b) of
theorem~\xrefn{thm:solns to second order homogeneous} is a solution to
the differential equation $\ds a\ddot y+b\dot y+cy=0$.
\endexercise

\exercise Verify that the function in part (c) of
theorem~\xrefn{thm:solns to second order homogeneous} is a solution to
the differential equation $\ds a\ddot y+b\dot y+cy=0$.
\endexercise

\exercise Solve the initial value problem $\ds\ddot y-\omega^2y=0$,
$y(0)=1$, $\ds\dot y(0)=1$, assuming $\omega\not=0$.
\answer $\ds {\omega+1\over2\omega}e^{\omega t}+
{\omega-1\over2\omega}e^{-\omega t}$
\endanswer
\endexercise

\exercise Solve the initial value problem $\ds2\ddot y+18y=0$,
$y(0)=2$, $\ds\dot y(0)=15$.
\answer $\ds 2\cos(3t)+5\sin(3t)$
\endanswer

%k9
\endexercise

\exercise Solve the initial value problem 
$\ds \ddot y+6\dot y +5y=0$,
$y(0)=1$, $\ds\dot y(0)=0$.
\answer $\ds -(1/4)e^{-5t}+(5/4)e^{-t}$
\endanswer
%k10
\endexercise

\exercise Solve the initial value problem 
$\ds\ddot y-\dot y-12y=0$,
$y(0)=0$, $\ds\dot y(0)=14$.
\answer $\ds-2e^{-3t}+2e^{4t}$
\endanswer
%k11
\endexercise

\exercise Solve the initial value problem 
$\ds\ddot y+12\dot y+36y=0$,
$y(0)=5$, $\ds\dot y(0)=-10$.
\answer $\ds 5e^{-6t}+20te^{-6t}$
\endanswer
%k12
\endexercise

\exercise Solve the initial value problem 
$\ds\ddot y-8\dot y+16y=0$,
$y(0)=-3$, $\ds\dot y(0)=4$.
\answer $\ds (16t-3)e^{4t}$
\endanswer
%k13
\endexercise

\exercise Solve the initial value problem 
$\ds\ddot y+5y=0$,
$y(0)=-2$, $\ds\dot y(0)=5$.
\answer $\ds -2\cos(\sqrt5t)+\sqrt{5}\sin(\sqrt{5}t)$
\endanswer
%k14
\endexercise

\exercise Solve the initial value problem 
$\ds\ddot y+y=0$,
$y(\pi/4)=0$, $\ds\dot y(\pi/4)=2$.
\answer $\ds -\sqrt2\cos t+\sqrt2\sin t$
\endanswer
%k15
\endexercise

\exercise Solve the initial value problem 
$\ds\ddot y+12\dot y+37y=0$,
$y(0)=4$, $\ds\dot y(0)=0$.
\answer $\ds e^{-6t}\left(4\cos t+24\sin t\right)$
\endanswer
%k16
\endexercise

\exercise Solve the initial value problem 
$\ds\ddot y+6\dot y+18y=0$,
$y(0)=0$, $\ds\dot y(0)=6$.
\answer $\ds 2e^{-3t}\sin(3t)$
\endanswer
%k17
\endexercise

\exercise Solve the initial value problem 
$\ds\ddot y+4y=0$,
$y(0)=\sqrt3$, $\ds\dot y(0)=2$. Put your answer in the form developed
at the end of exercise~\xrefn{example:damped spring oscillation}.
\answer $\ds 2\cos(2t-\pi/6)$
\endanswer
%k18
\endexercise

\exercise Solve the initial value problem 
$\ds\ddot y+100y=0$,
$y(0)=5$, $\ds\dot y(0)=50$. Put your answer in the form developed
at the end of exercise~\xrefn{example:damped spring oscillation}.
\answer $\ds 5\sqrt2\cos(10t-\pi/4)$
\endanswer
%k19
\endexercise

\exercise Solve the initial value problem 
$\ds\ddot y+4\dot y+13y=0$,
$y(0)=1$, $\ds\dot y(0)=1$. Put your answer in the form developed
at the end of exercise~\xrefn{example:damped spring oscillation}.
\answer $\ds \sqrt2 e^{-2t}\cos(3t-\pi/4)$
\endanswer
%k20
\endexercise

\exercise Solve the initial value problem 
$\ds\ddot y-8\dot y+25y=0$,
$y(0)=3$, $\ds\dot y(0)=0$. Put your answer in the form developed
at the end of exercise~\xrefn{example:damped spring oscillation}.
\answer $\ds 5e^{4t}\cos(3t+\arcsin(4/5))$
\endanswer
%k21
\endexercise

\exercise A mass-spring system $\ds m\ddot y+b\dot y+kx$ has
$k=29$, $b=4$, and $m=1$. At time $t=0$ the position is $y(0)=2$ and
the velocity is $\dot y(0)=1$. Find $y(t)$.
\answer $\ds (2\cos(5t)+\sin(5t))e^{-2t}$
\endanswer
%k22
\endexercise

\exercise A mass-spring system $\ds m\ddot y+b\dot y+kx$ has
$k=24$, $b=12$, and $m=3$. At time $t=0$ the position is $y(0)=0$ and
the velocity is $\dot y(0)=-1$. Find $y(t)$.
\answer $\ds-(1/2)e^{-2t}\sin(2t)$
\endanswer
\endexercise

\exercise Consider 
\exrdef{exer:second order really first order}
the differential equation $\ds a\ddot y + b\dot
y=0$, with $a$ and $b$ both non-zero. Find the general solution by the
method of this section. Now let $\ds g=\dot y$; the equation may be
written as $\ds a\dot g+bg=0$, a first order linear homogeneous
equation. Solve this for $g$, then use the relationship $\ds g=\dot y$ to
find $y$.

%k23
\endexercise

\exercise Suppose that $y(t)$ is a solution to $\ds a\ddot y+b\dot
y+cy=0$, $y(t_0)=0$, $\ds\dot y(t_0)=0$. Show that $y(t)=0$.
\endexercise

\endexercises
