\section{Asymptotes and Other Things to Look For}  {}{}
\nobreak
A vertical asymptote\index{asymptote} is a place where the function
becomes infinite, typically because the formula for the function has a
denominator that becomes zero.  For example, the reciprocal function
$f(x)=1/x$ has a vertical asymptote at $x=0$, and the function $\tan
x$ has a vertical asymptote at $x=\pi/2$ (and also at $x=-\pi/2$,
$x=3\pi/2$, etc.).  Whenever the formula for a function contains a
denominator it is worth looking for a vertical asymptote by 
checking to see if the denominator can ever be zero, and then checking
the limit at such points. Note that there is not always a vertical
asymptote where the derivative is zero: $f(x)=(\sin x)/x$ has a zero
denominator at $x=0$, but since $\ds \lim_{x\to 0}(\sin x)/x=1$ there is
no asymptote there.

A horizontal asymptote is a horizontal line to which $f(x)$ gets closer and
closer as $x$ approaches $\infty$ (or as $x$ approaches $-\infty$).  For
example, the reciprocal function has the $x$-axis for a horizontal
asymptote.  Horizontal asymptotes can be identified by computing 
the limits $\ds \lim_{x \to \infty}f(x)$ and $\ds \lim_{x \to -\infty}f(x)$.
Since $\ds \lim_{x \to \infty}1/x=\lim_{x \to -\infty}1/x=0$, the line
$y=0$ (that is, the $x$-axis) is a horizontal asymptote in both directions.

Some functions have asymptotes that are neither horizontal nor
vertical, but some other line. Such asymptotes are somewhat more
difficult to identify and we will ignore them.

If the domain of the function does not extend out to infinity, we should
also ask what happens as $x$ approaches the boundary of the domain.  For
example, the function $\ds y=f(x)=1/\sqrt{r^2-x^2}$ has domain $-r<x<r$, and
$y$ becomes infinite as $x$ approaches either $r$ or $-r$. In this
case we might also identify this behavior because when $x=\pm r$ the
denominator of the function is zero.

If there are any points where the derivative fails to exist (a cusp or
corner), then we should take special note of what the function does at such
a point.

Finally, it is worthwhile to notice any symmetry.  A function $f(x)$ that
has the same value for $-x$ as for $x$, i.e., $f(-x)=f(x)$, is called an
``even function.''  Its graph is symmetric with respect to the $y$-axis.
Some examples of even functions are: $\ds x^n$ when $n$ is an even number,
$\cos x$, and $\ds \sin^2x$.  On the other hand, a function that satisfies the
property $f(-x)=-f(x)$ is called an ``odd function.''  Its graph is
symmetric with respect to the origin.  Some examples of odd functions are:
$x^n$ when $n$ is an odd number, $\sin x$, and $\tan x$.  Of course, most
functions are neither even nor odd, and do not have any particular
symmetry.
 
\exercises

Sketch the curves. Identify clearly any interesting features, including
local maximum and minimum points, inflection points, asymptotes, and
intercepts. 

\twocol

\exercise $\ds y=x^5-5x^4+5x^3$
\endexercise

\exercise $\ds y=x^3-3x^2-9x+5$
\endexercise

\exercise $\ds y=(x-1)^2(x+3)^{2/3}$
\endexercise

\exercise $\ds x^2+x^2y^2=a^2y^2$, $a>0$.
\endexercise

\iflatetranscendentals
\elselatetranscendentals

\exercise $\ds y=xe^x$
\endexercise

\exercise $\ds y=(e^x+e^{-x})/2$
\endexercise

\exercise $\ds y=e^{-x}\cos x$
\endexercise

\exercise $\ds y=e^x-\sin x$
\endexercise

\exercise $\ds y=e^x/x$
\endexercise

\filatetranscendentals


\exercise $\ds y = 4x+\sqrt{1-x}$
\endexercise

\exercise $\ds y = (x+1)/\sqrt{5x^2 + 35}$
\endexercise

\exercise $\ds y= x^5 - x$
\endexercise

\exercise $\ds y = 6x + \sin 3x$
\endexercise

\exercise $\ds y = x+ 1/x$
\endexercise

\exercise $\ds y = x^2+ 1/x$
\endexercise

\exercise $\ds y = (x+5)^{1/4}$
\endexercise

\exercise $\ds y = \tan^2 x$
\endexercise

\exercise $\ds y =\cos^2 x - \sin^2 x$
\endexercise

\exercise $\ds y = \sin^3 x$
\endexercise

\exercise $\ds y=x(x^2+1)$
\endexercise

\exercise $\ds y=x^3+6x^2 + 9x$
\endexercise

\exercise $\ds y=x/(x^2-9)$
\endexercise

\exercise $\ds y=x^2/(x^2+9)$
\endexercise

\exercise $\ds y=2\sqrt{x} - x$
\endexercise

\exercise $\ds y=3\sin(x) - \sin^3(x)$, for $x\in[0,2\pi]$
\endexercise

\exercise $\ds y=(x-1)/(x^2)$

\endtwocol

\msk
\noindent
For each of the following five functions, identify any vertical and horizontal
asymptotes, and identify intervals on which the function is 
concave up and increasing; concave up and decreasing; concave
down and increasing; concave down and decreasing.
\endexercise

\exercise $f(\theta)=\sec(\theta)$
\endexercise

\exercise $\ds f(x) = 1/(1+x^2)$
\endexercise

\exercise $\ds f(x) = (x-3)/(2x-2)$ 
\endexercise

\exercise $\ds f(x) = 1/(1-x^2)$
\endexercise

\exercise $\ds f(x) = 1+1/(x^2)$
\endexercise

\exercise Let $\ds f(x) = 1/(x^2-a^2)$, where $a\geq0$.  Find any
 vertical and horizontal asymptotes and the intervals upon which the
 given function is concave up and increasing; concave up and
 decreasing; concave down and increasing; concave down and decreasing.
 Discuss how the value of $a$ affects these features.



\endexercise

\endexercises
