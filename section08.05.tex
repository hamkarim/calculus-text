\section{Rational Functions}{}{}
\nobreak
A {\dfont rational function\index{rational function}\index{function!rational}}
is a fraction with polynomials in the numerator and
denominator.  For example, 
$$
  {x^3\over x^2+x-6},
  \qquad\qquad
  {1\over (x-3)^2},
  \qquad\qquad
  {x^2+1\over x^2-1},
$$ 
are all rational functions of $x$.  There is a general technique
called ``partial fractions''\index{partial fractions} that, in
principle, allows us to integrate any rational function.  The
algebraic steps in the technique are rather cumbersome if the
polynomial in the denominator has degree more than 2, and the
technique requires that we factor the denominator, something that is
not always possible.  However, in practice one does not often run
across rational functions with high degree polynomials in the
denominator for which one has to find the antiderivative function.  So
we shall explain how to find the antiderivative of a rational function
only when the denominator is a quadratic polynomial $\ds ax^2+bx+c$.

We should mention a special type of rational function that we already
know how to integrate: If the denominator has the form $\ds (ax+b)^n$,
the substitution $u=ax+b$ will always work.  The denominator becomes
$\ds u^n$, and each $x$ in the numerator is replaced by $(u-b)/a$, and
$dx=du/a$. While it may be tedious to complete the integration if the
numerator has high degree, it is merely a matter of algebra.

\example
Find $\ds\int{x^3\over(3-2x)^5}\,dx.$ Using the substitution 
$u=3-2x$ we get
$$\eqalign{
  \int{x^3\over(3-2x)^5}\,dx
  &={1\over -2}\int {\left({u-3\over-2}\right)^3\over u^5}\,du
  ={1\over 16}\int {u^3-9u^2+27u-27\over u^5}\,du\cr
  &={1\over 16}\int u^{-2}-9u^{-3}+27u^{-4}-27u^{-5}\,du\cr
  &={1\over 16}\left({u^{-1}\over-1}-{9u^{-2}\over-2}+{27u^{-3}\over-3}
  -{27u^{-4}\over-4}\right)+C\cr
  &={1\over 16}\left({(3-2x)^{-1}\over-1}-{9(3-2x)^{-2}\over-2}+
  {27(3-2x)^{-3}\over-3}
  -{27(3-2x)^{-4}\over-4}\right)+C\cr
  &=-{1\over
    16(3-2x)}+{9\over32(3-2x)^2}-{9\over16(3-2x)^3}+{27\over64(3-2x)^4}+C\cr 
}$$
\vglue-10pt\endexample

We now proceed to the case in which the denominator is a quadratic
polynomial.  We can always factor out the coefficient of $\ds x^2$ and put
it outside the integral, so we can assume that the denominator has the
form $\ds x^2+bx+c$.  There are three possible cases, depending on how
the quadratic factors: either $\ds x^2+bx+c=(x-r)(x-s)$,
$\ds x^2+bx+c=(x-r)^2$, or it doesn't factor. We can use the quadratic
formula to decide which of these we have, and to factor the quadratic
if it is possible.

\example
Determine whether $\ds x^2+x+1$ factors, and factor it if possible.
The quadratic formula tells us that $\ds x^2+x+1=0$ when
$$x={-1\pm\sqrt{1-4}\over 2}.$$
Since there is no square root of $-3$, this quadratic does not factor.
\endexample

\example
Determine whether $\ds x^2-x-1$ factors, and factor it if possible.
The quadratic formula tells us that $\ds x^2-x-1=0$ when
$$x={1\pm\sqrt{1+4}\over 2}={1\pm\sqrt{5}\over2}.$$
Therefore
$$
  x^2-x-1=\left(x-{1+\sqrt{5}\over2}\right)\left(x-{1-\sqrt{5}\over2}\right).
$$
\vskip-10pt\endexample

If $\ds x^2+bx+c=(x-r)^2$ then we have the special case we have already
seen, that can be handled with a substitution. The other two cases
require different approaches.

If  $\ds x^2+bx+c=(x-r)(x-s)$, we have an integral of the form
$$\int{p(x)\over (x-r)(x-s)}\,dx$$
where $p(x)$ is a polynomial. The first step is to make sure that
$p(x)$ has degree less than 2.

\example
Rewrite $\ds\int {x^3\over (x-2)(x+3)}\,dx$ in terms of an integral
with a numerator that has degree less than 2. To do this we use 
\url{http://en.wikipedia.org/wiki/Polynomial_long_division}long
division of polynomials\endurl\index{long division of polynomials} to 
discover that
$$
  {x^3\over (x-2)(x+3)}={x^3\over x^2+x-6}=x-1+{7x-6\over x^2+x-6}=
  x-1+{7x-6\over (x-2)(x+3)},
$$
so 
$$
  \int {x^3\over (x-2)(x+3)}\,dx=\int x-1\,dx +\int {7x-6\over
  (x-2)(x+3)}\,dx.
$$
The first integral is easy, so only the second requires some work.
\endexample

Now consider the following simple algebra of fractions:
$$
  {A\over x-r}+{B\over x-s}={A(x-s)+B(x-r)\over (x-r)(x-s)}=
  {(A+B)x-As-Br\over (x-r)(x-s)}.
$$
That is, adding two fractions with constant numerator and denominators
$(x-r)$ and $(x-s)$ produces a fraction with denominator $(x-r)(x-s)$
and a polynomial of degree less than 2 for the numerator. We want to
reverse this process: starting with a single fraction, we want to
write it as a sum of two simpler fractions. An example should make it
clear how to proceed.

\example Evaluate $\ds\int {x^3\over (x-2)(x+3)}\,dx$.  We start by
writing $\ds{7x-6\over (x-2)(x+3)}$ as the sum of two fractions.  We
want to end up with
$${7x-6\over (x-2)(x+3)}={A\over x-2}+{B\over x+3}.$$
If we go ahead and add the fractions on the right hand side we get
$${7x-6\over (x-2)(x+3)}={(A+B)x+3A-2B\over (x-2)(x+3)}.$$
So all we need to do is find $A$ and $B$ so that $7x-6=(A+B)x+3A-2B$,
which is to say, we need $7=A+B$ and $-6=3A-2B$. This is a problem
you've seen before: solve a system of two equations in two
unknowns. There are many ways to proceed; here's one: If $7=A+B$ then
$B=7-A$ and so $-6=3A-2B=3A-2(7-A)=3A-14+2A=5A-14$. This is easy to
solve for $A$: $\ds A= 8/5$, and then $B=7-A=7-8/5=27/5$. Thus
$$
  \int {7x-6\over (x-2)(x+3)}\,dx=
  \int {8\over5}{1\over x-2}+{27\over5}{1\over x+3}\,dx=
  {8\over5}\ln |x-2|+{27\over5}\ln|x+3|+C.
$$
The answer to the original problem is now
$$\eqalign{
  \int {x^3\over (x-2)(x+3)}\,dx
  &=\int x-1\,dx +\int {7x-6\over (x-2)(x+3)}\,dx\cr
  &={x^2\over 2}-x+{8\over5}\ln |x-2|+{27\over5}\ln|x+3|+C.\cr
}$$
\vskip-10pt\endexample

Now suppose that $\ds x^2+bx+c$ doesn't factor. Again we can use long
division to ensure that the numerator has degree less than 2, then we
complete the square.

\example 
Evaluate $\ds\int {x+1\over x^2+4x+8}\,dx$. The quadratic denominator
does not factor. We could complete the square and use a trigonometric
substitution, but it is simpler to rearrange the integrand:
$$
  \int {x+1\over x^2+4x+8}\,dx = \int {x+2\over x^2+4x+8}\,dx -
  \int {1\over x^2+4x+8}\,dx.
$$
The first integral is an easy substitution problem, using $u=x^2+4x+8$:
$$
  \int {x+2\over x^2+4x+8}\,dx={1\over2}\int {du\over u}=
  {1\over2}\ln|x^2+4x+8|.
$$
For the second integral we complete the square:
$$
  x^2+4x+8=(x+2)^2+4=4\left(\left({x+2\over2}\right)^2+1\right),
$$
making the integral
$$ 
  {1\over4}\int {1\over\left({x+2\over2}\right)^2+1}\,dx.
$$
Using $\ds u={x+2\over2}$ we get
$$
  {1\over4}\int {1\over\left({x+2\over2}\right)^2+1}\,dx=
  {1\over4}\int {2\over u^2+1}\,dx=
  {1\over2}\arctan\left({x+2\over2}\right).
$$
The final answer is now 
$$
  \int {x+1\over x^2+4x+8}\,dx={1\over2}\ln|x^2+4x+8|-
  {1\over2}\arctan\left({x+2\over2}\right)+C.
$$
\endexample

% Too nasty?
% \example
% Evaluate $\ds\int {x\over x^2+x+1}\,dx$. We have seen that this
% quadratic does not factor. We complete the square:
% $$ 
%   x^2+x+1=x^2+x+{1\over 4} + 1 -{1\over4}=x^2+x+{1\over 4}+{3\over4}=
%   \left(x+{1\over2}\right)^2+{3\over4}.
% $$
% Now factor out $3/4$:
% $$
%   {3\over4}\left({4\over3}\left(x+{1\over2}\right)^2+1\right)
%   ={3\over4}\left(\left({2\over\sqrt{3}}x+{1\over\sqrt3}\right)^2+1\right).
% $$
% This can be done as a rather messy trigonometric substitution problem,
% but it is easier to do a bit more rewriting:
% $$
%   \int {x\over x^2+x+1}\,dx=\int{x+(1/2)\over x^2+x+1}\,dx-
%   \int{1/2\over x^2+x+1}\,dx.
% $$
% Now the first integral is an easy substitution, using $u=x^2+x+1$:
% $$
%   \int {x+(1/2)\over x^2+x+1}\,dx={1\over2}\int {du\over u}=
%   {1\over2}\ln|x^2+x+1|.
% $$
% The second integral is 
% $$
%   {1\over2}\int{dx\over x^2+x+1}=
%   {1\over2}\int{1\over(3/4)((2x/\sqrt3+1/\sqrt3)^2+1)}\,dx.
% $$
% Using $u=2x/\sqrt3+1/\sqrt3$ this becomes
% $$ 
%   {1\over\sqrt3}\int{du\over u^2+1}={1\over\sqrt3}\arctan(u)=
%   {1\over\sqrt3}\arctan\left({2\over\sqrt3}x+{1\over\sqrt3}\right).
% $$
% Thus the final answer is
% $$
%   \int {x\over x^2+x+1}\,dx = {1\over2}\ln|x^2+x+1|-
%   {1\over\sqrt3}\arctan\left({2\over\sqrt3}x+{1\over\sqrt3}\right)+C.
% $$
% \endexample

\exercises

Find the antiderivatives.
\twocol

\exercise $\ds\int {1\over 4-x^2}\,dx$
\answer $-\ln|x-2|/4+\ln|x+2|/4+C$
\endanswer
\endexercise

\exercise $\ds\int {x^4\over 4-x^2}\,dx$
\answer $\ds -x^3/3-4x-4\ln|x-2|+$\hfill\break$4\ln|x+2| +C$
\endanswer
\endexercise

\exercise $\ds\int {1\over x^2+10x+25}\,dx$
\answer $-1/(x+5) +C$
\endanswer
\endexercise

\exercise $\ds\int {x^2\over 4-x^2}\,dx$
\answer $-x-\ln|x-2|+\ln|x+2| +C$
\endanswer
\endexercise

\exercise $\ds\int {x^4\over 4+x^2}\,dx$
\answer $\ds -4x+x^3/3+8\arctan(x/2) +C$
\endanswer
\endexercise

\exercise $\ds\int {1\over x^2+10x+29}\,dx$
\answer $(1/2)\arctan(x/2+5/2) +C$
\endanswer
\endexercise

\exercise $\ds\int {x^3\over 4+x^2}\,dx$
\answer $\ds x^2/2-2\ln(4+x^2) +C$
\endanswer
\endexercise

\exercise $\ds\int {1\over x^2+10x+21}\,dx$
\answer $(1/4)\ln|x+3|-(1/4)\ln|x+7| +C$
\endanswer
\endexercise

\exercise $\ds\int {1\over 2x^2-x-3}\,dx$
\answer $(1/5)\ln|2x-3|-(1/5)\ln|1+x| +C$
\endanswer
\endexercise

\exercise $\ds\int {1\over x^2+3x}\,dx$
\answer $(1/3)\ln|x|-(1/3)\ln|x+3| +C$
\endanswer

\endtwocol
\endexercise

\endexercises
