\section{Calculus with vector functions}{}{}
\secrdef{section:calculus with vector functions}

A vector function ${\bf r}(t)=\langle f(t),g(t),h(t)\rangle$ is a
function of one variable---that is, there is only one ``input''
value. What makes vector functions more complicated than the functions
$y=f(x)$ that we studied in the first part of this book is of course
that the ``output'' values are now three-dimensional vectors instead
of simply numbers. It is natural to wonder if there is a corresponding
notion of derivative for vector functions. In the simpler case of a
function $y=s(t)$, in which $t$ represents time and $s(t)$ is position
on a line, we have seen that the derivative $s'(t)$ represents
velocity; we might hope that in a similar way the derivative of a
vector function would tell us something about the velocity of an
object moving in three dimensions.

One way to approach the question of the derivative for vector
functions is to write down an expression that is analogous to the
derivative we already understand, and see if we can make sense of
it. This gives us
$$\eqalign{
  {\bf r}'(t)&=\lim_{\Delta t\to0}{{\bf r}(t+\Delta t)-{\bf r}(t)\over
  \Delta t}\cr
  &=\lim_{\Delta t\to0}{\langle f(t+\Delta t)-f(t),g(t+\Delta t)-g(t),
  h(t+\Delta t)-h(t)\rangle\over \Delta t}\cr
  &=\lim_{\Delta t\to0}\langle {f(t+\Delta t)-f(t)\over\Delta t},
  {g(t+\Delta t)-g(t)\over\Delta t},
  {h(t+\Delta t)-h(t)\over \Delta t}\rangle\cr
  &=\langle f'(t),g'(t),h'(t)\rangle,\cr
}$$
if we say that what we mean by the limit of a vector is the vector of
the individual coordinate limits. So starting with a familiar
expression for what appears to be a derivative, we find that we can
make good computational sense out of it---but what does it actually
mean?

We know how to interpret ${\bf r}(t+\Delta t)$ and ${\bf r}(t)$---they
are vectors that point to locations in space; if $t$ is time, we can
think of these points as positions of a moving object at times that
are $\Delta t$ apart. We also know what $\Delta {\bf r}=
{\bf r}(t+\Delta t)-{\bf r}(t)$
means---it is a vector that points from the head of ${\bf r}(t)$ to
the head of ${\bf r}(t+\Delta t)$, assuming both have their tails at
the origin. So when $\Delta t$ is small, 
$\Delta {\bf r}$ is a tiny vector pointing from one
point on the path of the object to a nearby point. As $\Delta t$ gets
close to 0, this vector points in a direction that is closer and
closer to the direction in which the object is moving; geometrically,
it approaches a vector tangent to the path of the object at a
particular point.

\figure
\texonly
\vbox{\beginpicture
\normalgraphs
\ninepoint
\setcoordinatesystem units <6truemm,6truemm>
\setplotarea x from 0 to 11, y from 0 to 5.1
\arrow <5pt> [0.17, 0.5] from 0 0 to 2 3
\altarrow <5pt,5pt> [0.17, 0.5] from 2 3 to 7 4
\altarrow <5pt,5pt> [0.17, 0.5] from 0 0 to 7 4
\put {${\bf r}(t)$} [l] <8pt,7pt> at 1 1.5
\put {$\Delta{\bf r}$} [br] <0pt,-10pt> at 4.5 3.5
\put {${\bf r}(t+\Delta t)$} [tl]
  <3pt,-3pt> at 3.5 2
\setquadratic
%\plot 0.000 1.926 0.275 2.117 0.550 2.293 0.825 2.454 1.100 2.602 
%1.375 2.737 1.650 2.860 1.925 2.971 2.200 3.073 2.475 3.164 
%2.750 3.247 3.025 3.321 3.300 3.389 3.575 3.450 3.850 3.505 
%4.125 3.556 4.400 3.602 4.675 3.646 4.950 3.687 5.225 3.727 
%5.500 3.766 5.775 3.805 6.050 3.844 6.325 3.886 6.600 3.930 
%6.875 3.977 7.150 4.029 7.425 4.085 7.700 4.147 7.975 4.216 
%8.250 4.292 8.525 4.376 8.800 4.469 9.075 4.572 9.350 4.685 
%9.625 4.810 9.900 4.947 10.175 5.097 10.450 5.261 10.725 5.439 
%11.000 5.633 /
\plot 0.000 0.850 0.275 1.243 0.550 1.603 0.825 1.930 1.100 2.227 
1.375 2.494 1.650 2.733 1.925 2.946 2.200 3.134 2.475 3.299 
2.750 3.443 3.025 3.565 3.300 3.669 3.575 3.756 3.850 3.827 
4.125 3.884 4.400 3.928 4.675 3.961 4.950 3.984 5.225 3.999 
5.500 4.007 5.775 4.010 6.050 4.010 6.325 4.007 6.600 4.003 
6.875 4.001 7.150 4.000 7.425 4.004 7.700 4.012 7.975 4.028 
8.250 4.052 8.525 4.086 8.800 4.132 9.075 4.190 9.350 4.262 
9.625 4.351 9.900 4.457 10.175 4.582 10.450 4.727 10.725 4.894 
11.000 5.085 /
\endpicture}
\endtexonly
\htmlfigure{Vector_Functions_approximating_derivative.html}
\figrdef{fig:vector derivative}
\begincaption
Approximating the derivative.
\endcaption
\endfigure

Unfortunately, the vector $\Delta{\bf r}$ approaches 0 in length; the
vector $\langle 0,0,0\rangle$ is not very informative. By dividing by
$\Delta t$, when it is small, we effectively keep magnifying the
length of $\Delta{\bf r}$ so that in the limit it doesn't disappear. Thus the
limiting vector $\langle f'(t),g'(t),h'(t)\rangle$ will (usually) be a
good, non-zero vector that is tangent to the curve.

What about the length of this vector? It's nice that we've kept it
away from zero, but what does it measure, if anything?
Consider the length of one of the vectors that approaches the tangent
vector:
%$$\eqalign{
%|\langle {f(t+\Delta t)-f(t)\over\Delta t}&,
%{g(t+\Delta t)-g(t)\over\Delta t},
%{h(t+\Delta t)-h(t)\over \Delta t}\rangle|=\cr
%&{\sqrt{(f(t+\Delta t)-f(t))^2+(g(t+\Delta t)-g(t))^2+(h(t+\Delta t)-h(t))^2}
%\over\Delta t}\cr
%}$$
$$\left|{{\bf r}(t+\Delta t)-{\bf r}(t)\over
\Delta t}\right|={|{\bf r}(t+\Delta t)-{\bf r}(t)|\over|\Delta t|}$$
The numerator is the length of the vector that points from one position
of the object to a ``nearby'' position; this length is approximately
the distance traveled by the object between times $t$ and $t+\Delta
t$. Dividing this distance by the length of time it takes to travel
that distance gives the average speed. As $\Delta t$ approaches zero,
this average speed approaches the actual, instantaneous speed of the
object at time $t$. 

So by performing an ``obvious'' calculation to get something that
looks like the derivative of ${\bf r}(t)$, we get precisely what we
would want from such a derivative: the vector ${\bf r}'(t)$
points in the direction of travel of the object and its length tells
us the speed of travel. In the case that $t$ is time, then, we call 
${\bf v}(t)={\bf r}'(t)$ the velocity vector. Even if $t$ is not time,
${\bf r}'(t)$ is useful---it is a vector tangent to the curve.

\example We have seen that ${\bf r}=\langle \cos t,\sin t,t\rangle$ is a helix. 
We compute ${\bf r}'=\langle -\sin t,\cos t,1\rangle$, and
$|{\bf r}'|=\sqrt{\sin^2 t+\cos^2 t+1}=\sqrt2$. So thinking of this as
a description of a moving object, its speed is always $\sqrt2$; see
figure~\xrefn{fig:helix with tangent}. For an animated view of the
tangent vectors, see the Java applet.
\endexample

\figure
\texonly
\vbox{\beginpicture
\normalgraphs
\ninepoint
\setcoordinatesystem units <3truecm,3truecm>
\setplotarea x from 0 to 1.1, y from 0 to 1.1
\put {\hbox{\epsfxsize5cm\epsfbox{helix_with_tangent.eps}}} at 0 0
\endpicture}
\begincaption
A tangent vector on the helix.
(\expandafter\url\expandafter{\liveurl helix_with_tangent.html}%
AP\endurl)
\endcaption
\endtexonly
\figrdef{fig:helix with tangent}
\htmlfigure{Vector_Functions_helix_with_tangent.html}
\begincaption
A tangent vector on the helix.
\endcaption
\endfigure

\example The velocity vector for $\langle \cos t,\sin
t,\cos t\rangle$ is $\langle -\sin t,\cos
t,-\sin t\rangle$. As before, the first two coordinates mean that from
above this curve looks like a circle. The $z$ coordinate is now also
periodic, so that as the object moves around the curve its height
oscillates up and down. In fact it turns out that the curve is a
tilted ellipse, as shown in figure~\xrefn{fig:tilted ellipse}.
\endexample

\figure
\texonly
\vbox{\beginpicture
\normalgraphs
\ninepoint
\setcoordinatesystem units <3truecm,3truecm>
\setplotarea x from 0 to 1.1, y from 0 to 1.1
\put {\hbox{\epsfxsize5cm\epsfbox{tilted_ellipse.eps}}} at 0 0
\endpicture}
\begincaption
The ellipse ${\bf r}=\langle \cos t,\sin
t,\cos t\rangle$.
(\expandafter\url\expandafter{\liveurl ellipse_with_tangent.html}%
AP\endurl)
\endcaption
\endtexonly
\figrdef{fig:tilted ellipse}
\htmlfigure{Vector_Functions_ellipse_with_tangent.html}
\begincaption
The ellipse ${\bf r}=\langle \cos t,\sin
t,\cos t\rangle$.
\endcaption
\endfigure

\example The velocity vector for $\langle \cos t,\sin
t,\cos 2t\rangle$ is $\langle -\sin t,\cos
t,-2\sin 2t\rangle$. The $z$ coordinate is now oscillating twice as
fast as in the previous example, so the graph is not surprising; see
figure~\xrefn{fig:roller coaster}.
\endexample

\figure
\texonly
\vbox{\beginpicture
\normalgraphs
\ninepoint
\setcoordinatesystem units <3truecm,3truecm>
\setplotarea x from 0 to 1.1, y from 0 to 1.1
\put {\hbox{\epsfxsize5cm\epsfbox{coaster_with_tangent.eps}}} at 0 0
\endpicture}
\begincaption
$\langle \cos t,\sin
t,\cos 2t\rangle$.
(\expandafter\url\expandafter{\liveurl coaster_with_tangent.html}%
AP\endurl)
\endcaption
\endtexonly
\figrdef{fig:roller coaster}
\htmlfigure{Vector_Functions_coaster_with_tangent.html}
\begincaption
$\langle \cos t,\sin
t,\cos 2t\rangle$.
\endcaption
\endfigure

\example Find the angle between the curves $\langle t,1-t,3+t^2 \rangle$ and 
$\langle 3-t,t-2,t^2\rangle$ where they meet.

The angle between two curves at a point is the angle between their
tangent vectors---any tangent vectors will do, so we can use the
derivatives. We need to find the point of intersection, evaluate the
two derivatives there, and finally find the angle between them.

To find the point of intersection, we need to solve the equations
$$\eqalign{
t&=3-u\cr
1-t&=u-2\cr
3+t^2&=u^2\cr
}$$
Solving either of the first two equations for $u$ and substituting in
the third gives $3+t^2=(3-t)^2$, which means $t=1$. This together with
$u=2$ satisfies all three equations. Thus the two curves meet at
$\langle 1,0,4\rangle$, the first when $t=1$ and the second when
$t=2$.

The derivatives are $\langle 1,-1,2t\rangle$ and
$\langle -1,1,2t\rangle$; at the intersection point these are
$\langle 1,-1,2\rangle$ and $\langle -1,1,4\rangle$. The cosine of the
angle between them is then
$$\cos\theta = {-1-1+8\over\sqrt6\sqrt{18}}={1\over\sqrt3},$$
so $\theta=\arccos(1/\sqrt3)\approx0.96$.
\endexample

The derivatives of vector functions obey some familiar looking rules,
which we will occasionally need.

\thm Suppose ${\bf r}(t)$ and ${\bf s}(t)$ are differentiable functions,
$f(t)$ is a differentiable function, and $a$ is a real number.

\beginlist

\item{a.} $\ds {d\over dt} a{\bf r}(t)= a{\bf r}'(t)$

\item{b.} $\ds {d\over dt} ({\bf r}(t)+{\bf s}(t))=
{\bf r}'(t)+{\bf s}'(t)$

\item{c.} $\ds {d\over dt} f(t){\bf r}(t)= f(t){\bf r}'(t)+f'(t){\bf r}(t)$

\item{d.} $\ds {d\over dt} ({\bf r}(t)\cdot{\bf s}(t))=
{\bf r}'(t)\cdot{\bf s}(t)+{\bf r}(t)\cdot{\bf s}'(t)$

\item{e.} $\ds {d\over dt} ({\bf r}(t)\times{\bf s}(t))=
{\bf r}'(t)\times{\bf s}(t)+{\bf r}(t)\times{\bf s}'(t)$

\item{f.} $\ds {d\over dt} {\bf r}(f(t))= {\bf r}'(f(t))f'(t)$

\endlist
\thmrdef{thm:vector derivative properties}
\endthmnoproof

Note that because the cross product is not commutative you must
remember to do the three cross products in formula (e) in the correct
order. 

When the derivative of a function $f(t)$ is zero, we know that the
function has a horizontal tangent line, and may have a local maximum
or minimum point. If ${\bf r}'(t)={\bf 0}$, the geometric
interpretation is quite different, though the interpretation in terms
of motion is similar. Certainly we know that the object has speed zero
at such a point, and it may thus be abruptly changing direction. In
three dimensions there are many ways to change direction;
geometrically this often means the curve has a cusp or a point, as in
the path of a ball that bounces off the floor or a wall.

\example Suppose that ${\bf r}(t)=\langle 1+t^3,t^2,1\rangle$, so
${\bf r}'(t)=\langle 3t^2,2t,0\rangle$. This is $\bf 0$ at $t=0$, and
there is indeed a cusp at the point $\langle 1,0,1\rangle$, as
shown in figure~\xrefn{fig:spacecurve cusp}.
\endexample

\figure
\texonly
\vbox{\beginpicture
\normalgraphs
\ninepoint
\setcoordinatesystem units <3truecm,3truecm>
\setplotarea x from 0 to 1.1, y from 0 to 1.1
\put {\hbox{\epsfxsize5cm\epsfbox{spacecurve_cusp.eps}}} at 0 0
\endpicture}
\begincaption
$\langle 1+t^3,t^2,1\rangle$ has a cusp at $\langle 1,0,1\rangle$.
(\expandafter\url\expandafter{\liveurl spacecurve_cusp.html}%
AP\endurl)
\endcaption
\endtexonly
\figrdef{fig:spacecurve cusp}
\htmlfigure{Vector_Functions_spacecurve_cusp.html}
\begincaption
$\langle 1+t^3,t^2,1\rangle$ has a cusp at $\langle 1,0,1\rangle$.
\endcaption
\endfigure

Sometimes we will be interested in the direction of ${\bf r}'$ but not
its length. In some cases, we can still work with ${\bf r}'$, as when
we find the angle between two curves. On other occasions it will be
useful to work with a unit vector\index{unit vector} in the same
direction as ${\bf r}'$; of course, we can compute such a vector by
dividing ${\bf r}'$ by its own length. This standard unit tangent
vector is usually denoted by ${\bf T}$:
$${\bf T}={{\bf r}'\over|{\bf r}'|}.$$
In a sense, when we computed the angle between two tangent vectors we
have already made use of the unit tangent, since
$$\cos\theta = {{\bf r}'\cdot{\bf s}'\over|{\bf r}'||{\bf s}'|}=
{{\bf r}'\over|{\bf r}'|}\cdot{{\bf s}'\over|{\bf s}'|}$$

Now that we know how to make sense of ${\bf r}'$, we immediately know
what an antiderivative must be, namely
$$\int {\bf r}(t)\,dt = \langle \int f(t)\,dt,\int g(t)\,dt,\int h(t)\,dt,
\rangle,$$ if ${\bf r}=\langle f(t),g(t),h(t)\rangle$. What about
definite integrals?  Suppose that ${\bf v}(t)$ gives the velocity of
an object at time $t$. Then ${\bf v}(t)\Delta t$ is a vector that
approximates the displacement of the object over the time $\Delta t$:
${\bf v}(t)\Delta t$ points in the direction of travel, and $|{\bf
  v}(t)\Delta t|=|{\bf v}(t)||\Delta t|$ is the speed of the object
times $\Delta t$, which is approximately the distance traveled. Thus,
if we sum many such tiny vectors:
$$\sum_{i=0}^{n-1}{\bf v}(t_i)\Delta t$$
we get an approximation to the displacement vector over the time
interval $[t_0,t_n]$. If we take the limit we get the exact
value of the displacement vector:
$$\lim\sum_{i=0}^{n-1}{\bf v}(t_i)\Delta t = \int_{t_0}^{t_n}{\bf
v}(t)\,dt = {\bf r}(t_n)-{\bf r}(t_0).$$
Thus, given the velocity vector we can compute the vector function
${\bf r}$ giving the location of the object:
$${\bf r}(t)={\bf r}_0+\int_0^t {\bf v}(u)\,du.$$

\example An object moves with velocity vector $\langle \cos t, \sin t,
\cos t\rangle$, starting at $\langle 1,1,1\rangle$. Find the function
${\bf r}$ giving its location.
$$\eqalign{
{\bf r}(t)&=\langle 1,1,1\rangle+\int_0^t \langle \cos u, \sin u,
\cos u\rangle\,du\cr
&=\langle 1,1,1\rangle+\left.\langle \sin u, -\cos u,\sin u\rangle
\right|_0^t\cr
&=\langle 1,1,1\rangle+\langle \sin t, -\cos t,\sin t\rangle-
\langle 0,-1,0\rangle\cr
&=\langle 1+\sin t, 2-\cos t,1+\sin t\rangle\cr
}$$
See figure~\xrefn{fig:velocity cos sin cos}.
\endexample

\figure
\texonly
\vbox{\beginpicture
\normalgraphs
\ninepoint
\setcoordinatesystem units <3truecm,3truecm>
\setplotarea x from 0 to 1.1, y from 0 to 1.1
\put {\hbox{\epsfxsize5cm\epsfbox{path_with_tangent.eps}}} at 0 0
\endpicture}
\begincaption
Path of the object with its initial velocity vector.
(\expandafter\url\expandafter{\liveurl path_with_tangent.html}%
AP\endurl)
\endcaption
\endtexonly
\figrdef{fig:velocity cos sin cos}
\htmlfigure{Vector_Functions_path_with_tangent.html}
\begincaption
Path of the object with its initial velocity vector.
\endcaption
\endfigure


\exercises

\exercise Find ${\bf r}'$ and $\bf T$ for
${\bf r} = \langle t^2,1,t\rangle$.
\answer $\langle 2t,0,1\rangle$, ${\bf r}'/\sqrt{1+4t^2}$
\endanswer
\endexercise

\exercise Find ${\bf r}'$ and $\bf T$ for
${\bf r} = \langle \cos t, \sin 2t, t^2\rangle$.
\answer $\langle -\sin t, 2\cos 2t,2t\rangle$,
${\bf r}'/\sqrt{\sin^2t + 4\cos^2(2t)+4t^2}$
\endanswer
\endexercise

\exercise Find ${\bf r}'$ and $\bf T$ for
${\bf r} = \langle \cos(e^t),\sin(e^t),\sin t\rangle$.
\answer $\langle -e^t\sin(e^t),e^t\cos(e^t),\cos t\rangle$,
${\bf r}'/\sqrt{e^{2t}+\cos^2 t}$
\endanswer
\endexercise

\exercise Find a vector function for the line tangent to the helix
$\langle \cos t,\sin t, t\rangle$ when $t=\pi/4$.
\answer $\langle \sqrt2/2,\sqrt2/2,\pi/4\rangle+
t\langle -\sqrt2/2,\sqrt2/2,1\rangle$
\endanswer
\endexercise

\exercise Find a vector function for the line tangent to 
$\langle \cos t,\sin t, \cos 4t \rangle$ when $t=\pi/3$.
\answer $\langle 1/2,\sqrt3/2,-1/2\rangle+
t\langle -\sqrt3/2,1/2,2\sqrt3\rangle$
\endanswer
\endexercise

\exercise Find the cosine of the angle between the curves $\langle
0,t^2,t\rangle$ and $\langle \cos(\pi t/2),\sin(\pi t/2), t\rangle$
where they intersect.
\answer $2/\sqrt5/\sqrt{4+\pi^2}$
\endanswer
\endexercise

\exercise Find the cosine of the angle between the curves $\langle
\cos t,-\sin(t)/4,\sin t\rangle$ and $\langle \cos t,\sin t, \sin(2t)\rangle$
where they intersect.
\answer $7\sqrt{5}\sqrt{17}/85$, $-9\sqrt{5}\sqrt{17}/85$
\endanswer
\endexercise

\exercise Suppose that $|{\bf r}(t)|=k$, for some constant $k$. This
means that $\bf r$ describes some path on the sphere of radius $k$
with center at the origin. Show that $\bf r$ is perpendicular to ${\bf
  r}'$ at every point. Hint: Use Theorem~\xrefn{thm:vector derivative
  properties}, part (d).
\exrdef{exercise:derivative is perpendicular}
\endexercise

\exercise A bug is crawling along the spoke of a wheel that lies along
a radius of the wheel. The bug is crawling at 1 unit per second and
the wheel is rotating at 1 radian per second. Suppose the wheel lies
in the $y$-$z$ plane with center at the origin, and at time $t=0$ the
spoke lies along the positive $y$ axis and the bug is at the origin. 
Find a vector function ${\bf r}(t)$
for the position of the bug at time $t$, the velocity vector
${\bf r}'(t)$, the unit tangent ${\bf T}(t)$, and the speed of the bug
$|{\bf r}'(t)|$.
\answer $\langle 0,t\cos t,t\sin t\rangle$, 
$\langle 0,\cos t-t\sin t,\sin t+t\cos t\rangle$,
${\bf r}'/\sqrt{1+t^2}$, $\sqrt{1+t^2}$
\endanswer
\endexercise

\exercise An object moves with velocity vector $\langle \cos t, \sin t,
t\rangle$, starting at $\langle 0,0,0\rangle$ when $t=0$. Find the function
${\bf r}$ giving its location.
\answer $\langle \sin t,1-\cos t,t^2/2\rangle$
\endanswer

\endexercise

\exercise The position function of a particle is given by ${\bf r}(t) =
\langle t^2,5t,t^2-16t\rangle$, $t\geq 0$.  When is the speed of the particle
a minimum?
\answer $t=4$
\endanswer
\endexercise

\exercise A particle moves so that its position is given by
$\langle \cos t, \sin t, \cos(6t)\rangle$. Find the maximum and
minimum speeds of the particle.
\answer $37$, $1$
\endanswer
\endexercise

\exercise An object moves with velocity vector $\langle t, t^2,
\cos t\rangle$, starting at $\langle 0,0,0\rangle$ when $t=0$.
 Find the function
${\bf r}$ giving its location.
\answer $\langle t^2/2,t^3/3,\sin t\rangle$
\endanswer
\endexercise

\exercise What is the physical interpretation of the dot product of two
  vector valued functions?  What is the physical interpretation of the
  cross product of two vector valued functions?
\endexercise

\exercise Show, using the rules of cross products and differentiation,
  that
$${d\over dt} ({\bf r}(t) \times {\bf r}'(t))= 
{\bf r}(t) \times {\bf r}''(t).$$
\endexercise

\exercise Determine the point at which ${\bf f}(t)=\langle t, t^2, t^3
\rangle$ and ${\bf g}(t) =\langle \cos(t), \cos(2t), t+1 \rangle$
intersect, and find the angle between the curves at that point.  (Hint:
You'll need to set this one up like a line intersection problem,
writing one in $s$ and one in $t$.) If these two functions were the
trajectories of two airplanes on the same scale of time, would the
planes collide at their point of intersection?  Explain.
\answer $(1,1,1)$ when $t=1$ and $s=0$; $\theta=\arccos(3/\sqrt{14})$; no
\endanswer
\endexercise

\exercise Find the equation of the plane perpendicular to the curve ${\bf r}(t)
 = \langle 2\sin(3t),t,2\cos(3t)\rangle$ at the point $(0,\pi,-2)$.
\answer $-6x+(y-\pi)=0$
\endanswer
\endexercise

\exercise Find the equation of the plane perpendicular to 
$\langle \cos t, \sin t, \cos(6t)\rangle$ when $t=\pi/4$.
\answer $\ds -x/\sqrt2+y/\sqrt2+6z=0$
\endanswer
\endexercise

\exercise At what point on the curve 
${\bf r}(t) = \langle t^3,3t,t^4\rangle$ is the
plane perpendicular to the curve also parallel to the plane $6x+6y-8z=1$?
\answer $(-1,-3,1)$
\endanswer
\endexercise

\exercise Find the equation of the line tangent to 
$\langle \cos t, \sin t, \cos(6t)\rangle$ when $t=\pi/4$.
\answer $\langle 1/\sqrt2,1/\sqrt2,0\rangle+t\langle -1,1,6\sqrt2\rangle$
\endanswer
\endexercise

\endexercises
