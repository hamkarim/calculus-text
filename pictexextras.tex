\catcode`!=11 
\newdimen\!!xB
\newdimen\!!yB
\newdimen\!!xE
\newdimen\!!yE
\newdimen\!!HYP
\newdimen\!!cosine
\newdimen\!!sine
\newdimen\!!DIAMOFFSET
\newdimen\!!RADIUSOFFSET
\newdimen\!!RADIUS
\newdimen\!OFFSETAMOUNT
\!OFFSETAMOUNT=2pt
% *******************************
% *** BRACES                  ***
% *******************************
%
% ** User commands
% **  \putbrace <RADIUS> from XFROM YFROM to XTO YTO
%
% ** Draws a simple ``curly brace'' between given points, using
% ** quarter circles with radius RADIUS at the ends and to form the
% ** point in the middle. The brace is offset by \!OFFSETAMOUNT in the
% ** direction of the point.
%
\def\putbrace <#1> from #2 #3 to #4 #5 {%
\!!RADIUS=#1\!!RADIUSOFFSET=\!OFFSETAMOUNT
\advance\!!RADIUSOFFSET by \!!RADIUS
\!!DIAMOFFSET=\!!RADIUSOFFSET\advance\!!DIAMOFFSET by \!!RADIUS
\!!xB=\Xdistance{#2}\!!yB=\Ydistance{#3}
\!!xE=\Xdistance{#4}\!!yE=\Ydistance{#5}
\advance\!!yE by -\!!yB\advance\!!xE by -\!!xB
\placehypotenuse for <\!!xE> and <\!!yE> in <\!!HYP>
\Divide <\!!xE> by <\!!HYP> forming <\!!xB>
\placevalueinpts of <\!!xB> in {\!!cosine}
\Divide <\!!yE> by <\!!HYP> forming <\!!yB>
\placevalueinpts of <\!!yB> in {\!!sine}
\put {\beginpicture
\setdimensionmode
\startrotation by {\!!cosine} {\!!sine} about {0pt} {0pt}
\!!xE=\!!HYP \divide\!!xE by 2 \advance\!!xE by -1.9696\!!RADIUS
\ellipticalarc axes ratio 2:1 80 degrees from {\!!xE} {\!!RADIUSOFFSET}
                center at {\!!xE} {\!!DIAMOFFSET} %
\plot {2\!!RADIUS} {\!!RADIUSOFFSET} {\!!xE} {\!!RADIUSOFFSET} /
\advance\!!xE by 3.93923\!!RADIUS\!!xB=\!!HYP\advance\!!xB by -2\!!RADIUS
\ellipticalarc axes ratio 2:1 -80 degrees from {\!!xE} {\!!RADIUSOFFSET}
                center at {\!!xE} {\!!DIAMOFFSET} %
\plot {\!!xE} {\!!RADIUSOFFSET} {\!!xB} {\!!RADIUSOFFSET} /
\ellipticalarc axes ratio 2:1 -90 degrees from {0pt} {\!OFFSETAMOUNT} 
                center at {2\!!RADIUS} {\!OFFSETAMOUNT}
\ellipticalarc axes ratio 2:1 90 degrees from {\!!HYP} {\!OFFSETAMOUNT} 
                center at {\!!xB} {\!OFFSETAMOUNT}
\endpicture} at #2 #3
}

% *******************************
% *** ARROWS  (Draws arrows)  ***
% *******************************
%
% ** User commands
% **  \altarrow <ARROW HEAD LENGTH,HEAD POSITION> [MID FRACTION, BASE FRACTION]
% **    [<XSHIFT,YSHIFT>] from XFROM YFROM to XTO YTO
%
% ** Draws an arrow from (XFROM,YFROM) to (XTO,YTO).  The arrow head
% ** is constructed of two quadratic arcs, which extend back a distance
% ** ARROW HEAD LENGTH (a dimension) on both sides of the arrow shaft
% ** from the tip of the arrowhead. The arrowhead tip is at distance
% ** HEAD POSITION from the XTO YTO end of the line.
%
% ** All the way back the arcs are a distance BASE FRACTION*ARROW HEAD
% ** LENGTH apart, while half-way back they are a distance MID FRACTION*
% ** ARROW HEAD LENGTH apart. <XSHIFT,YSHIFT> is optional, and has
% ** its usual interpreation. See Subsection 5.4 of the manual.

\def\one#1,#2.{#1}
\def\two#1,#2.{#2}
\newdimen\!xstart
\newdimen\!ystart

\def\altarrow <#1,#2> [#3,#4]{%
  \!ifnextchar<{\!altarrow{#1}{#2}{#3}{#4}}{\!altarrow{#1}{#2}{#3}{#4}<\!zpt,\!zpt> }}

\def\!altarrow#1#2#3#4<#5> from #6 #7 to #8 #9 {%
%
% ** convert to dimensions
  \!xloc=\!M{#8}\!xunit   
  \!yloc=\!M{#9}\!yunit
  \!dxpos=\!xloc  \!dimenA=\!M{#6}\!xunit  \advance \!dxpos -\!dimenA
  \!dypos=\!yloc  \!dimenA=\!M{#7}\!yunit  \advance \!dypos -\!dimenA
  \let\!MAH=\!M%                         ** save current c/d mode
  \!setdimenmode%                        ** go into dimension mode
%
  \!xshift=\one#5.\relax  \!yshift=\two#5.\relax%  ** pick up shift
  \!reverserotateonly\!xshift\!yshift%   ** back rotate shift
  \advance\!xshift\!xloc  \advance\!yshift\!yloc
%
% **  draw shaft of arrow
  \!xS=-\!dxpos  \advance\!xS\!xshift
  \!yS=-\!dypos  \advance\!yS\!yshift
  \!start (\!xS,\!yS)
  \!ljoin (\!xshift,\!yshift)
%
% ** find 32*cosine and 32*sine of angle of rotation
  \!Pythag\!dxpos\!dypos\!arclength
  \!divide\!dxpos\!arclength\!dxpos  
  \!dxpos=32\!dxpos  \!removept\!dxpos\!!cos
  \!divide\!dypos\!arclength\!dypos  
  \!dypos=32\!dypos  \!removept\!dypos\!!sin
% 
% ** construct arrowhead
  \!myhalfhead{#1}{#2}{#3}{#4}%                ** draw half of arrow head
  \!myhalfhead{#1}{#2}{-#3}{-#4}%              ** draw other half
%
  \let\!M=\!MAH%                         ** restore old c/d mode
  \ignorespaces}
%
% ** draw half of arrow head
  \def\!myhalfhead#1#2#3#4{%
    \!dimenC=-#1%                
    \divide \!dimenC 2 %                 ** half way back
    \!dimenD=#3\!dimenC%                 ** half the mid width
    \advance\!dimenC by -#2
    \!rotate(\!dimenC,\!dimenD)by(\!!cos,\!!sin)to(\!xM,\!yM)
    \!dimenC=-#1%                        ** all the way back
    \!dimenD=#4\!dimenC
    \!dimenD=.5\!dimenD%                 ** half the full width
    \advance\!dimenC by -#2
    \!rotate(\!dimenC,\!dimenD)by(\!!cos,\!!sin)to(\!xE,\!yE)
    \!dimenC=0pt%
    \!dimenD=0pt%
    \advance\!dimenC by -#2
    \!rotate(\!dimenC,\!dimenD)by(\!!cos,\!!sin)to(\!xstart,\!ystart)
    \advance\!xstart by \!xshift
    \advance\!ystart by \!yshift
    \!start (\!xstart,\!ystart)
    \advance\!xM\!xshift  \advance\!yM\!yshift
    \advance\!xE\!xshift  \advance\!yE\!yshift
    \!qjoin (\!xM,\!yM) (\!xE,\!yE) 
    \ignorespaces}

\catcode`!=12
