\section{The Cross Product}{}{}
\nobreak
Another useful operation: Given two vectors, find a third vector
perpendicular to the first two. There are of course an infinite number
of such vectors of different lengths. Nevertheless, let us find one.
Suppose $\ds {\bf A}=\langle
a_1,a_2,a_3\rangle$ and $\ds {\bf B}=\langle b_1,b_2,b_3\rangle$. We want
to find a vector $\ds {\bf v} = \langle v_1,v_2,v_3\rangle$ with
${\bf v}\cdot{\bf A}={\bf v}\cdot{\bf B}=0$, or
$$\eqalign{
  a_1v_1+a_2v_2+a_3v_3&=0,\cr
  b_1v_1+b_2v_2+b_3v_3&=0.\cr
}$$
Multiply the first equation by $\ds b_3$ and the second by $\ds a_3$ and
subtract to get
$$\eqalign{
  b_3a_1v_1+b_3a_2v_2+b_3a_3v_3&=0\cr
  a_3b_1v_1+a_3b_2v_2+a_3b_3v_3&=0\cr
  (a_1b_3-b_1a_3)v_1 + (a_2b_3-b_2a_3)v_2&=0\cr
}$$
Of course, this equation in two variables has many solutions; a
particularly easy one to see is $\ds v_1=a_2b_3-b_2a_3$,
$\ds v_2=b_1a_3-a_1b_3$. Substituting back into either of the original
equations and solving for $\ds v_3$ gives $\ds v_3=a_1b_2-b_1a_2$.

This particular answer to the problem turns out to have some nice
properties, and it is dignified with a name: the 
{\dfont cross product\index{cross product}\index{vector!cross product}}:
$$
  {\bf A}\times{\bf B} = \langle
  a_2b_3-b_2a_3,b_1a_3-a_1b_3,a_1b_2-b_1a_2\rangle.
$$
While there is a nice pattern to this vector, it can be a bit
difficult to memorize;  here is a convenient mnemonic.
The determinant of a two by two matrix is
$$\left|\matrix{a&b\cr
c&d\cr}\right|=ad-cb.$$
This is extended to the determinant of a three by three matrix:
$$\eqalign{
  \left|\matrix{x&y&z\cr
  a_1&a_2&a_3\cr
  b_1&b_2&b_3\cr}\right|&=x\left|\matrix{a_2&a_3\cr
  b_2&b_3\cr}\right|-y\left|\matrix{a_1&a_3\cr
  b_1&b_3\cr}\right|+z\left|\matrix{a_1&a_2\cr
  b_1&b_2\cr}\right|\cr
  &=x(a_2b_3-b_2a_3)-y(a_1b_3-b_1a_3)+z(a_1b_2-b_1a_2)\cr
  &=x(a_2b_3-b_2a_3)+y(b_1a_3-a_1b_3)+z(a_1b_2-b_1a_2).\cr}
$$
Each of the two by two matrices is formed by deleting the top row and
one column of the three by three matrix; the subtraction of the middle
term must also be memorized. This is not the place to extol the uses
of the determinant; suffice it to say that determinants are
extraordinarily useful and important. Here we want to use it merely as
a mnemonic device. You will have noticed that the three expressions in
parentheses on the last line are precisely the three coordinates of
the cross product; replacing $x$, $y$, $z$ by $\bf i$, $\bf j$, $\bf
k$ gives us
$$\eqalign{
  \left|\matrix{{\bf i}&{\bf j}&{\bf k}\cr
  a_1&a_2&a_3\cr
  b_1&b_2&b_3\cr}\right|
  &=(a_2b_3-b_2a_3){\bf i}-(a_1b_3-b_1a_3){\bf j}+(a_1b_2-b_1a_2){\bf
    k}\cr
  &=(a_2b_3-b_2a_3){\bf i}+(b_1a_3-a_1b_3){\bf j}+(a_1b_2-b_1a_2){\bf
    k}\cr
  &=\langle
  a_2b_3-b_2a_3,b_1a_3-a_1b_3,a_1b_2-b_1a_2\rangle\cr
  &={\bf A}\times{\bf B}.\cr}
$$

Given $\bf A$ and $\bf B$, there are typically two possible directions and an
infinite number of magnitudes that will give a vector perpendicular to
both $\bf A$ and $\bf B$. As we have picked a particular one, we
should investigate the magnitude and direction.

We know how to compute the magnitude of ${\bf A}\times{\bf B}$; it's a
bit messy but not difficult. It is somewhat easier to work initially
with the square of the magnitude, so as to avoid the square root:
$$\eqalign{
  |{\bf A}\times{\bf B}|^2&=
  (a_2b_3-b_2a_3)^2+(b_1a_3-a_1b_3)^2+(a_1b_2-b_1a_2)^2\cr
  &=a_2^2b_3^2-2a_2b_3b_2a_3+b_2^2a_3^2+b_1^2a_3^2-2b_1a_3a_1b_3+a_1^2b_3^2+a_1^2b_2^2-2a_1b_2b_1a_2+b_1^2a_2^2\cr
}$$
While it is far from obvious, this nasty looking expression can be
simplified: 
$$\eqalign{
  |{\bf A}\times{\bf B}|^2&=
  (a_1^2+a_2^2+a_3^2)(b_1^2+b_2^2+b_3^2)-(a_1b_1+a_2b_2+a_3b_3)^2\cr
  &=|{\bf A}|^2|{\bf B}|^2-({\bf A}\cdot{\bf B})^2\cr
  &=|{\bf A}|^2|{\bf B}|^2-|{\bf A}|^2|{\bf B}|^2\cos^2\theta\cr
  &=|{\bf A}|^2|{\bf B}|^2(1-\cos^2\theta)\cr
  &=|{\bf A}|^2|{\bf B}|^2\sin^2\theta\cr
  |{\bf A}\times{\bf B}|&=|{\bf A}||{\bf B}|\sin\theta\cr
}$$
The magnitude of ${\bf A}\times{\bf B}$ is thus very similar to the dot
product. In particular, notice that if $\bf A$ is parallel to $\bf B$,
the angle between them is zero, so $\sin\theta=0$, so 
$|{\bf A}\times{\bf B}|=0$, and likewise if they are anti-parallel, 
$\sin\theta=0$, and
$|{\bf A}\times{\bf B}|=0$. Conversely, if $|{\bf A}\times{\bf B}|=0$
and $|{\bf A}|$ and $|{\bf B}|$ are not zero, it must be that
$\sin\theta=0$, so $\bf A$ is parallel or anti-parallel to $\bf B$. 

\pagerdef{page:parallelogram area} Here is a curious fact about this
quantity that turns out to be quite useful later on: Given two
vectors, we can put them tail to tail and form a
parallelogram, as in figure~\xrefn{fig:area of parallelogram}. The
height of the parallelogram, $h$, is $|{\bf A}|\sin\theta$, and the
base is $|{\bf B}|$, so the area of the
parallelogram\index{parallelogram!area of} is $|{\bf A}||{\bf
B}|\sin\theta$, exactly the magnitude of $|{\bf A}\times{\bf B}|$.

\figure
\texonly
\vbox{\beginpicture
\normalgraphs
\ninepoint
\setcoordinatesystem units <6truemm,6truemm>
\setplotarea x from 0 to 7, y from 0 to 3
\arrow <4pt> [0.35, 1] from 0 0 to 5 0
\arrow <4pt> [0.35, 1] from 0 0 to 2 3
\setdashes
\plot 5 0 7 3 2 3 2 0 /
\put {$\bf A$} [br] <-3pt,3pt> at 1 1.5
\put {$\bf B$} [t] <0pt,-3pt> at 2.5 0
\put {$h$} [l] <3pt,0pt> at 2 1.5
\put {$\theta$} [bl] <7pt,4pt> at 0 0
\endpicture}
\endtexonly
\figrdef{fig:area of parallelogram}
\htmlfigure{Three_Dimensions-area_parallelogram_cross_product.html}
\begincaption
A parallelogram.
\endcaption
\endfigure

What about the direction of the cross product? Remarkably, there is a
simple rule that describes the direction. Let's look at a simple
example: Let ${\bf A}=\langle a,0,0\rangle$, ${\bf B}=\langle 
b,c,0\rangle$. If the vectors are placed with tails at the origin,
$\bf A$ lies along the $x$-axis and $\bf B$ lies in the $x$-$y$ plane,
so we know the cross product will point either up or down. The cross
product is 
$$\eqalign{
  {\bf A}\times {\bf B}=\left|\matrix{{\bf i}&{\bf j}&{\bf k}\cr
  a&0&0\cr
  b&c&0\cr}\right|
  &=\langle
  0,0,ac\rangle.\cr}
$$
As predicted, this is a vector pointing up or down, depending on the
sign of $ac$. Suppose that $a>0$, so the sign depends only on $c$: if
$c>0$, $ac>0$ and the vector points up; if $c<0$, the vector points
down. On the other hand, if $a<0$ and $c>0$, the vector points down,
while if $a<0$ and $c<0$, the vector points up. Here is how to
interpret these facts with a single rule: Imagine rotating vector
$\bf A$ until it points in the same direction as $\bf B$; there are
two ways to do this---use the rotation that goes through the smaller
angle. If $a>0$ and $c>0$, or $a<0$ and $c<0$, the rotation will be
counter-clockwise when viewed from above; in the other two cases, $\bf
A$ must be rotated clockwise to reach $\bf B$. The rule is:
counter-clockwise means up, clockwise means down. If $\bf A$ and $\bf
B$ are any vectors in the $x$-$y$ plane, the same rule applies---$\bf
A$ need not be parallel to the $x$-axis.

Although it is somewhat difficult computationally to see how this
plays out for any two starting vectors, the rule is essentially the
same. Place $\bf A$ and $\bf B$ tail to tail. The plane in which $\bf
A$ and $\bf B$ lie may be viewed from two sides; view it from the side
for which $\bf A$ must rotate counter-clockwise to reach $\bf B$; then
the vector ${\bf A}\times{\bf B}$ points toward you.

This rule is usually called the {\dfont right hand rule\index{right hand rule}}.
Imagine placing the heel of your right hand at the point where the tails are
joined, so that your slightly curled fingers indicate the direction of
rotation from $\bf A$ to $\bf B$. Then your thumb points in the
direction of the cross product ${\bf A}\times{\bf B}$.

One immediate consequence of these facts is that 
${\bf A}\times{\bf B}\not={\bf B}\times{\bf A}$, because the two
cross products point in the opposite direction. On the other hand,
since 
$$
  |{\bf A}\times{\bf B}|=|{\bf A}||{\bf B}|\sin\theta
  =|{\bf B}||{\bf A}|\sin\theta=|{\bf B}\times{\bf A}|,
$$
the lengths of the two cross products are equal, so
we know that ${\bf A}\times{\bf B}=-({\bf B}\times{\bf A})$.

The cross product has some familiar-looking properties that will be
useful later, so we list them here. As with the dot product, these can
be proved by performing the appropriate calculations on coordinates,
after which we may sometimes avoid such calculations by using the
properties. 

\thm
If ${\bf u}$, ${\bf v}$, and ${\bf w}$ are vectors and $a$ is a real
number, then
\thmrdef{thm:cross product properties}

\beginlist
\item{1.} ${\bf u}\times({\bf v}+{\bf w}) = 
{\bf u}\times{\bf v}+{\bf u}\times{\bf w}$

\nobreak
\item{2.} $({\bf v}+{\bf w})\times{\bf u} = 
{\bf v}\times{\bf u}+{\bf w}\times{\bf u}$

\nobreak
\item{3.} $(a{\bf u})\times{\bf v}=a({\bf u}\times{\bf v})
={\bf u}\times(a{\bf v})$

\nobreak
\item{4.} ${\bf u}\cdot({\bf v}\times{\bf w}) = 
({\bf u}\times{\bf v})\cdot{\bf w}$

\nobreak
\item{5.}${\bf u}\times({\bf v}\times{\bf w}) =
({\bf u}\cdot{\bf w}){\bf v}-({\bf u}\cdot{\bf v}){\bf w}$
\endthmnoproof

\endlist

\exercises

\exercise Find the cross product of $\langle 1,1,1\rangle$ and 
$\langle 1,2,3\rangle$. 
\answer $\langle 1,-2,1\rangle$
\endanswer
\endexercise

\exercise Find the cross product of $\langle 1,0,2\rangle$ and 
$\langle -1,-2,4\rangle$. 
\answer $\langle 4,-6,-2\rangle$
\endanswer
\endexercise

\exercise Find the cross product of $\langle -2,1,3\rangle$ and 
$\langle 5,2,-1\rangle$. 
\answer $\langle -7,13,-9\rangle$
\endanswer
\endexercise

\exercise Find the cross product of $\langle 1,0,0\rangle$ and 
$\langle 0,0,1\rangle$. 
\answer  $\langle 0,-1,0\rangle$
\endanswer
\endexercise

\exercise Two vectors ${\bf u}$ and ${\bf v}$ are separated by an
angle of $\pi/6$, and $|{\bf u}|=2$ and $|{\bf v}|=3$. Find 
$|{\bf u}\times{\bf v}|$.
\answer $3$
\endanswer
\endexercise

\exercise Two vectors ${\bf u}$ and ${\bf v}$ are separated by an
angle of $\pi/4$, and $|{\bf u}|=3$ and $|{\bf v}|=7$. Find 
$|{\bf u}\times{\bf v}|$.
\answer $\ds 21/\sqrt2$
\endanswer
\endexercise

\exercise Find the area of the parallelogram with vertices $(0,0)$, $(1,2)$,
$(3,7)$, and $(2,5)$.
\answer $1$
\endanswer
\endexercise

\exercise Find and explain the value of $({\bf i} \times {\bf j})
\times {\bf k}$ and $({\bf i} + {\bf j}) \times ({\bf i} - {\bf j})$.
\endexercise

\exercise Prove that for all vectors ${\bf u}$ and ${\bf v}$,
$({\bf u}\times{\bf v})\cdot{\bf v}=0$.
\endexercise

\exercise Prove Theorem~\xrefn{thm:cross product properties}.
\endexercise

\exercise Define the triple product of three vectors, ${\bf x}$,
${\bf y}$, and ${\bf z}$, to be the scalar ${\bf x} \cdot ({\bf y} \times
{\bf z})$.  Show that three vectors lie in the same plane if and only if
their triple product is zero. Verify that $\langle 1, 5, -2 \rangle$,
$\langle 4, 3, 0 \rangle$ and $\langle 6, 13, -4 \rangle$ are
coplanar.
\endexercise

\endexercises

