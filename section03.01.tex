\section{The Power Rule}{}{}
\index{power rule}

We start with the derivative of a power function\index{power
function}, $\ds f(x)=x^n$. Here $n$ is a number of any kind: integer,
rational, positive, negative, even irrational, as in $\ds x^\pi$. We have
already computed some simple examples, so the formula should not be a
complete surprise:
$${d\over dx}x^n = nx^{n-1}.$$
It is not easy to show this is true for any $n$. We will do some of
the easier cases now, and discuss the rest later.

The easiest, and most common, is the case that $n$ is a positive
integer. To compute the derivative we need to compute the following
limit:
$${d\over dx}x^n = \lim_{\Delta x\to0} {(x+\Delta x)^n-x^n\over \Delta
  x}.
$$
For a specific, fairly small value of $n$, we could do this by
straightforward algebra.

\example
Find the derivative of $\ds f(x)=x^3$.
$$\eqalign{
{d\over dx}x^3 &= \lim_{\Delta x\to0} {(x+\Delta x)^3-x^3\over \Delta
  x}.\cr
&=\lim_{\Delta x\to0} {x^3+3x^2\Delta x+3x\Delta x^2 + \Delta x^3
-x^3\over \Delta x}.\cr
&=\lim_{\Delta x\to0}{3x^2\Delta x+3x\Delta x^2 + \Delta x^3\over \Delta x}.\cr
&=\lim_{\Delta x\to0}3x^2+3x\Delta x + \Delta x^2 = 3x^2.\cr
}$$
\vskip-10pt
\endexample

The general case is really not much harder as long as we don't try to
do too much. The key is understanding what happens when $\ds (x+\Delta x)^n$
is multiplied out:
$$(x+\Delta x)^n=x^n + nx^{n-1}\Delta x + a_2x^{n-2}\Delta x^2+\cdots+
+a_{n-1}x\Delta x^{n-1} + \Delta x^n.$$
We know that multiplying out will give a large number of terms all of
the form $\ds x^i\Delta x^j$, and in fact that $i+j=n$ in every term. One
way to see this is to understand that one method for multiplying out 
$\ds (x+\Delta x)^n$ is the following: In every $(x+\Delta x)$ factor,
pick either the $x$ or the $\Delta x$, then multiply the $n$ choices
together; do this in all possible ways. For example, for 
$\ds (x+\Delta x)^3$, there are eight possible ways to do this:
$$\eqalign{
(x+\Delta x)(x+\Delta x)(x+\Delta x)&=xxx + xx\Delta x + x\Delta x x
+ x\Delta x \Delta x\cr
&\qquad+ \Delta x xx + \Delta xx\Delta x + \Delta x\Delta x x
+ \Delta x\Delta x \Delta x\cr
&= x^3 + x^2\Delta x +x^2\Delta x +x\Delta x^2\cr
&\quad+x^2\Delta x +x\Delta x^2 +x\Delta x^2 +\Delta x^3\cr
&=x^3 + 3x^2\Delta x + 3x\Delta x^2+\Delta x^3\cr
}$$
No matter what $n$ is, there are $n$ ways to pick $\Delta x$ in one
factor and $x$ in the remaining $n-1$ factors; this means one term is
$\ds nx^{n-1}\Delta x$. The other coefficients are somewhat harder to
understand, but we don't really need them, so in the formula above
they have simply been called $\ds a_2$, $\ds a_3$, and so on. We know that every
one of these terms contains $\Delta x$ to at least the power 2. Now
let's look at the limit:
$$\eqalign{
{d\over dx}x^n &= \lim_{\Delta x\to0} {(x+\Delta x)^n-x^n\over \Delta
  x}\cr
&=\lim_{\Delta x\to0} {x^n + nx^{n-1}\Delta x + a_2x^{n-2}\Delta x^2+\cdots+
a_{n-1}x\Delta x^{n-1} + \Delta x^n-x^n\over \Delta x}\cr
&=\lim_{\Delta x\to0} {nx^{n-1}\Delta x + a_2x^{n-2}\Delta x^2+\cdots+
a_{n-1}x\Delta x^{n-1} + \Delta x^n\over \Delta x}\cr
&=\lim_{\Delta x\to0} nx^{n-1} + a_2x^{n-2}\Delta x+\cdots+
a_{n-1}x\Delta x^{n-2} + \Delta x^{n-1} = nx^{n-1}.\cr
}$$

Now without much trouble we can verify the formula for negative
integers. First let's look at an example:

\example Find the derivative of $\ds y=x^{-3}$. Using the formula,
$\ds y'=-3x^{-3-1}=-3x^{-4}$. 
\endexample

Here is the general computation. Suppose $n$ is a negative integer;
the algebra is easier to follow if we use $n=-m$ in the computation,
where $m$ is a positive integer.
$$\eqalign{
{d\over dx}x^n &= {d\over dx}x^{-m} =
\lim_{\Delta x\to0} {(x+\Delta x)^{-m}-x^{-m}\over \Delta
  x}\cr
&=\lim_{\Delta x\to0} { {1\over (x+\Delta x)^m} - {1\over x^m} \over
  \Delta x} \cr
&=\lim_{\Delta x\to0} { x^m - (x+\Delta x)^m \over
(x+\Delta x)^m x^m \Delta x} \cr
&=\lim_{\Delta x\to0} { x^m - (x^m + mx^{m-1}\Delta x + a_2x^{m-2}\Delta x^2+\cdots+
a_{m-1}x\Delta x^{m-1} + \Delta x^m)\over
(x+\Delta x)^m x^m \Delta x} \cr
&=\lim_{\Delta x\to0} { -mx^{m-1} - a_2x^{m-2}\Delta x-\cdots-
a_{m-1}x\Delta x^{m-2} - \Delta x^{m-1})\over
(x+\Delta x)^m x^m} \cr
&={ -mx^{m-1} \over x^mx^m}=
{ -mx^{m-1} \over x^{2m}}=
-mx^{m-1-2m}= nx^{-m-1} = nx^{n-1}.\cr
}$$

We will later see why the other cases of the power rule work, but from
now on we will use the power rule whenever $n$ is any real number.
Let's note here a simple case in which the power rule applies, or
almost applies, but is not really needed. Suppose that $f(x)=1$;
remember that this ``1'' is a function, not ``merely'' a number, and
that $f(x)=1$ has a graph that is a horizontal line, with slope zero
everywhere. So we know that $f'(x)=0$. We might also write $\ds f(x)=x^0$,
though there is some question about just what this means at $x=0$. If
we apply the power rule, we get $\ds f'(x)=0x^{-1}=0/x=0$, again noting
that there is a problem at $x=0$. So the power rule ``works'' in this
case, but it's really best to just remember that the derivative of any
constant function is zero.

\exercises

Find the derivatives of the given functions.

\twocol

\exercise $\ds x^{100}$
\answer $\ds 100x^{99}$
\endanswer
\endexercise

\exercise $\ds x^{-100}$
\answer $\ds -100x^{-101}$
\endanswer
\endexercise

\exercise $\displaystyle {1\over x^5}$
\answer $\ds -5x^{-6}$
\endanswer
\endexercise

\exercise $\ds x^\pi$
\answer $\ds \pi x^{\pi-1}$
\endanswer
\endexercise

\exercise $\ds x^{3/4}$
\answer $\ds (3/4)x^{-1/4}$
\endanswer
\endexercise

\exercise $\ds x^{-9/7}$
\answer $\ds -(9/7)x^{-16/7}$
\endanswer

\endtwocol
\endexercise

\endexercises
