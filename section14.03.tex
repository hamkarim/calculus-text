\section{Partial Differentiation}{}{}

When we first considered what the derivative of a vector function
might mean, there was really not much difficulty in understanding
either how such a thing might be computed or what it might measure. In
the case of functions of two variables, things are a bit harder to
understand. If we think of a function of two variables in terms of its
graph, a surface, there is a more-or-less obvious derivative-like
question we might ask, namely, how ``steep'' is the surface. But it's
not clear that this has a simple answer, nor how we might proceed. We
will start with what seem to be very small steps toward the goal;
surprisingly, it turns out that these simple ideas hold the keys to a
more general understanding.

\figure
\texonly
\vbox{\beginpicture
\normalgraphs
\ninepoint
\setcoordinatesystem units <3truecm,3truecm>
\setplotarea x from 0 to 1.1, y from 0 to 1.1
\put {\hbox{\epsfxsize8cm\epsfbox{parabolic_crosssection.eps}}} at 0 0
\endpicture}
\begincaption
$f(x,y)=x^2 + y^2$, cut by the plane $x+y=1$
(\expandafter\url\expandafter{\liveurl parabolic_crosssection.html}%
AP\endurl)
\endcaption
\endtexonly
\figrdef{fig:parabolic bowl cross-section}
\htmlfigure{Partial_Differentiation_parabolic_cross_section.html}
\htmlonly
\begincaption
$f(x,y)=x^2 + y^2$, cut by the plane $x+y=1$
\endcaption
\endhtmlonly
\endfigure

Imagine a particular point on a surface; what might we be able to say
about how steep it is? We can limit the question to make it more
familiar: how steep is the surface in a particular direction? What
does this even mean? Here's one way to think of it: Suppose we're
interested in the point $(a,b,c)$. Pick a straight line in the $x$-$y$
plane through the point $(a,b,0)$, then extend the line vertically
into a plane. Look at the intersection of the plane with the surface.
If we pay attention to just the plane, we see the chosen straight line
where the $x$-axis would normally be, and the intersection with the
surface shows up as a curve in the plane. Figure~\xrefn{fig:parabolic
  bowl cross-section} shows the parabolic surface from
figure~\xrefn{fig:parabolic bowl}, exposing its cross-section above
the line $x+y=1$.

In principle, this is a problem we know how to solve: find the slope
of a curve in a plane. Let's start by looking at some particularly
easy lines: those parallel to the $x$ or $y$ axis. Suppose we are
interested in the cross-section of $f(x,y)$ above the line $y=b$. If
we substitute $b$ for $y$ in $f(x,y)$, we get a function in one
variable, describing the height of the cross-section as a function of
$x$. Because $y=b$ is parallel to the $x$-axis, if we view it from a
vantage point on the negative $y$-axis, we will see what appears to be
simply an ordinary curve in the $x$-$z$ plane.

\figure
\texonly
\vbox{\beginpicture
\normalgraphs
\ninepoint
\setcoordinatesystem units <3truecm,3truecm>
\setplotarea x from 0 to 4, y from 0 to 1
\put {\hbox{\epsfxsize8cm\epsfbox{parabolic_crosssection2.eps}}} at 0 0
\put {\hbox{\epsfxsize5cm\epsfbox{parabolic_crosssection3.eps}}} at 2.5 0
\endpicture}
\begincaption
$f(x,y)=x^2 + y^2$, cut by the plane $y=2$
(\expandafter\url\expandafter{\liveurl parabolic_crosssection2.html}%
AP\endurl)
\endcaption
\endtexonly
\figrdef{fig:parabolic bowl cross-section two}
\htmlfigure{Partial_Differentiation_parabolic_cross_section_2.html}
\htmlonly
\begincaption
$f(x,y)=x^2 + y^2$, cut by the plane $y=2$
\endcaption
\endhtmlonly
\endfigure

Consider again the parabolic surface $f(x,y)=x^2+y^2$. The
cross-section above the line $y=2$ consists of all points
$(x,2,x^2+4)$. Looking at this cross-section from somewhere on
the negative $y$ axis, we see what appears to be just the curve
$f(x)=x^2+4$. At any point on the cross-section, $(a,2,a^2+4)$, the
steepness of the surface {\em in the direction of the line\/} $y=2$ is
  simply the slope of the curve $f(x)=x^2+4$, namely $2x$.
Figure~\xrefn{fig:parabolic bowl cross-section two} shows the same
parabolic surface as before, but now cut by the plane $y=2$. The left
graph shows the cut-off surface, the right shows just the
cross-section, looking up from the negative $y$-axis toward the
origin. 

If, say, we're interested in the point $(-1,2,5)$ on the surface, then
the slope in the direction of the line $y=2$ is $2x=2(-1)=-2$. This
means that starting at $(-1,2,5)$ and moving on the surface, above the
line $y=2$, in the direction of increasing $x$ values, the surface
goes down; of course moving in the opposite direction, toward
decreasing $x$ values, the surface will rise.

If we're interested in some other line $y=k$, there is really no
change in the computation. The equation of the cross-section above
$y=k$ is $x^2+k^2$ with derivative $2x$. We can save ourselves the
effort, small as it is, of substituting $k$ for $y$: all we are in
effect doing is temporarily assuming that $y$ is some constant. With
this assumption, the derivative ${d\over dx}(x^2+y^2)=2x$. To
emphasize that we are only temporarily assuming $y$ is constant, we
use a slightly different notation: ${\partial\over \partial
  x}(x^2+y^2)=2x$; the ``$\partial$'' reminds us that there are more
variables than $x$, but that only $x$ is being treated as a variable.
We read the equation as ``the partial derivative of $(x^2+y^2)$ with
respect to $x$ is $2x$.'' A convenient alternate notation for the
partial derivative of $f(x,y)$ with respect to $x$ is
is $f_x(x,y)$.

\example The partial derivative with respect to $x$ of 
$x^3+3xy$ is $3x^2+3y$. Note that the partial derivative includes the
variable $y$, unlike the example $x^2+y^2$. It is somewhat unusual for
the partial derivative to depend on a single variable; this example is
more typical.
\endexample

Of course, we can do the same sort of calculation for lines parallel
to the $y$-axis. We temporarily hold $x$ constant, which gives us the
equation of the cross-section above a line $x=k$. We can then compute
the derivative with respect to $y$; this will measure the steepness of
the curve in the $y$ direction.

\example The partial derivative with respect to $y$ of 
$f(x,y)=\sin(xy)+3xy$ is 
$$f_y(x,y)={\partial\over\partial y}\sin(xy)+3xy=\cos(xy){\partial\over\partial
  y}(xy)+ 3x=x\cos(xy)+3x.
$$
\vskip-10pt
\endexample

So far, using no new techniques, we have succeeded in measuring the
slope of a surface in two quite special directions. For functions of
one variable, the derivative is closely linked to the notion of
tangent line. For surfaces, the analogous idea is the tangent
plane---a plane that just touches a surface at a point, and has the
same ``steepness'' as the surface in all directions. Even though we
haven't yet figured out how to compute the slope in all directions, we
have enough information to find tangent planes. Suppose we want the
plane tangent to a surface at a particular point $(a,b,c)$. If we compute the
two partial derivatives of the function for that point, we get enough
information to determine two lines tangent to the surface, both 
through $(a,b,c)$ and both tangent to the surface in their respective
directions. These two lines determine a plane, that is, there is
exactly one plane containing the two lines: the tangent
plane. Figure~\xrefn{fig:sphere with tangent plane} 
shows (part of) two tangent lines at a point,
and the tangent plane containing them. 

\figure
\texonly
\vbox{\beginpicture
\normalgraphs
\ninepoint
\setcoordinatesystem units <3truecm,3truecm>
\setplotarea x from -1 to 3, y from 0 to 1
\put {\hbox{\epsfxsize4cm\epsfbox{tangent_vectors.eps}}} at 0 -0.15
\put {\hbox{\epsfxsize5.8cm\epsfbox{tangent_plane.eps}}} at 2.5 0
\endpicture}
\begincaption
Tangent vectors and tangent plane.
(\expandafter\url\expandafter{\liveurl tangent_plane.html}%
AP\endurl)
\endcaption
\endtexonly
\figrdef{fig:sphere with tangent plane}
\htmlfigure{Partial_Differentiation_sphere_with_tangent_plane.html}
\begincaption
Tangent vectors and tangent plane.
\endcaption
\endfigure


How can we discover an equation for this tangent plane? We know a
point on the plane, $(a,b,c)$; we need a vector normal to the
plane. If we can find two vectors, one parallel to each of the tangent
lines we know how to find, then the cross product of these vectors
will give the desired normal vector.

\figure
\texonly
\vbox{\beginpicture
\normalgraphs
\ninepoint
\setcoordinatesystem units <1.5truecm,1.5truecm>
\setplotarea x from 0 to 4.1, y from 0 to 3.1
\axis left ticks numbered from 0 to 3 by 1 /
\axis bottom ticks numbered from 0 to 4 by 1 /
\put {$z$} [b] <0pt,3pt> at 0 3.1
\put {$x$} [l] <3pt,0pt> at 4.1 0
\put {$f_x(2,b)$} [l] <3pt,0pt> at 3 1.35
\putrule from 2 1.17 to 3 1.17
\putrule from 3 1.17 to 3 1.5
\setlinear
%\plot 2 1.17 3 1.5 /
\arrow <5pt> [0.17, 0.5] from 2 1.17 to 3 1.5
\setquadratic
\plot 0.000 1.167 0.100 1.135 0.200 1.107 0.300 1.082 0.400 1.060 
0.500 1.042 0.600 1.027 0.700 1.015 0.800 1.007 0.900 1.002 
1.000 1.000 1.100 1.002 1.200 1.007 1.300 1.015 1.400 1.027 
1.500 1.042 1.600 1.060 1.700 1.082 1.800 1.107 1.900 1.135 
2.000 1.167 2.100 1.202 2.200 1.240 2.300 1.282 2.400 1.327 
2.500 1.375 2.600 1.427 2.700 1.482 2.800 1.540 2.900 1.602 
3.000 1.667 3.100 1.735 3.200 1.807 3.300 1.882 3.400 1.960 
3.500 2.042 3.600 2.127 3.700 2.215 3.800 2.307 3.900 2.402 
4.000 2.500 /
%0.000 1.333 0.100 1.270 0.200 1.213 0.300 1.163 0.400 1.120 
%0.500 1.083 0.600 1.053 0.700 1.030 0.800 1.013 0.900 1.003 
%1.000 1.000 1.100 1.003 1.200 1.013 1.300 1.030 1.400 1.053 
%1.500 1.083 1.600 1.120 1.700 1.163 1.800 1.213 1.900 1.270 
%2.000 1.333 2.100 1.403 2.200 1.480 2.300 1.563 2.400 1.653 
%2.500 1.750 2.600 1.853 2.700 1.963 2.800 2.080 2.900 2.203 
%3.000 2.333 3.100 2.470 3.200 2.613 3.300 2.763 3.400 2.920 
%3.500 3.083 3.600 3.253 3.700 3.430 3.800 3.613 3.900 3.803 
%4.000 4.000 / 
\endpicture}
\endtexonly
\figrdef{fig:tangent vector}
\htmlfigure{Partial_Differentiation_a_tangent_vector.html}
\begincaption
A tangent vector.
\endcaption
\endfigure

How can we find vectors parallel to the tangent lines? Consider first
the line tangent to the surface above the line $y=b$. A vector
$\langle u,v,w\rangle$ parallel to this tangent line must have $y$
component $v=0$, and we may as well take the $x$ component to be
$u=1$. The ratio of the $z$ component to the $x$ component is the
slope of the tangent line, precisely what we know how to compute. The
slope of the tangent line is $f_x(a,b)$, so
$$ f_x(a,b)={w\over u} ={w\over1} = w.$$
In other words, a vector parallel to this tangent line is
$\langle 1,0,f_x(a,b)\rangle$, as shown in figure~\xrefn{fig:tangent vector}. 
If we repeat the reasoning for the
tangent line above $x=a$, we get the vector $\langle
0,1,f_y(a,b)\rangle$.

Now to find the desired normal vector we compute the cross product,
$\langle 0,1,f_y\rangle\times\langle 1,0,f_x\rangle=
\langle f_x,f_y,-1\rangle$. From our earlier discussion of planes, we
can write down the equation we seek: $f_x(a,b)x+f_y(a,b)y-z=k$, and
$k$ as usual can be computed by substituting a known point:
$f_x(a,b)(a)+f_y(a,b)(b)-c=k$. There are various more-or-less nice
ways to write the result:
$$\displaylines{
f_x(a,b)x+f_y(a,b)y-z=f_x(a,b)a+f_y(a,b)b-c\cr
f_x(a,b)x+f_y(a,b)y-f_x(a,b)a-f_y(a,b)b+c=z\cr
f_x(a,b)(x-a)+f_y(a,b)(y-b)+c=z\cr
f_x(a,b)(x-a)+f_y(a,b)(y-b)+f(a,b)=z\cr
}$$

\example Find the plane tangent to $x^2+y^2+z^2=4$ at
$(1,1,\sqrt2)$. This point is on the upper hemisphere, so 
we use $\ds f(x,y)=\sqrt{4-x^2-y^2}$. Then 
$\ds f_x(x,y)=-x(4-x^2-y^2)^{-1/2}$ and $\ds
f_y(x,y)=-y(4-x^2-y^2)^{-1/2}$, so $f_x(1,1)=f_y(1,1)=-1/\sqrt2$
and the equation of the plane is 
$$z=-{1\over\sqrt2}(x-1)-{1\over\sqrt2}(y-1)+\sqrt2.$$
The hemisphere and this tangent plane are pictured in
figure~\xrefn{fig:sphere with tangent plane}.
\endexample

So it appears that to find a tangent plane, we need only find two
quite simple ordinary derivatives, namely $f_x$ and $f_y$. This is
true {\em if the tangent plane exists}. It is, unfortunately, not
always the
case that if $f_x$ and $f_y$ exist there is a tangent plane. 
Consider the function  $xy^2/(x^2+y^4)$ pictured in 
figure~\xrefn{fig:weird limit}. This function has value 0 when $x=0$
or $y=0$, and we can ``plug the hole'' by agreeing that
$f(0,0)=0$. Now it's clear that $f_x(0,0)=f_y(0,0)=0$, because in the
$x$ and $y$ directions the surface is simply a horizontal line. But
it's also clear from the picture that this surface does not have
anything that deserves to be called a ``tangent plane'' at the origin,
certainly not the $x$-$y$ plane containing these two tangent lines.

When does a surface have a tangent plane at a particular point? What
we really want from a tangent plane, as from a tangent line, is that
the plane be a ``good'' approximation of the surface near the
point. Here is how we can make this precise:

\defn Let $\Delta x=x-x_0$, $\Delta y=y-y_0$, and $\Delta z=z-z_0$
where $z_0=f(x_0,y_0)$. The
function $z=f(x,y)$ is differentiable\index{differentiable} at
$(x_0,y_0)$ if
$$\Delta z=f_x(x_0,y_0)\Delta x+f_y(x_0,y_0)\Delta y+\epsilon_1\Delta
x + \epsilon_2\Delta y,$$ and both $\epsilon_1$ and $\epsilon_2$
approach 0 as $(x,y)$ approaches $(x_0,y_0)$.
\enddef

This definition takes a bit of absorbing. Let's rewrite the central
equation a bit:
$$\eqalignno{
z&=f_x(x_0,y_0)(x-x_0)+f_y(x_0,y_0)(y-y_0)+f(x_0,y_0)+
\epsilon_1\Delta x + \epsilon_2\Delta y.&
\eqrdef{eq:f is differentiable}(\xrefn{eq:f is differentiable})\cr
}$$ 
The first three terms on
the right are the equation of the tangent plane, that is,
$$f_x(x_0,y_0)(x-x_0)+f_y(x_0,y_0)(y-y_0)+f(x_0,y_0)$$
 is the $z$-value
of the point on the plane above $(x,y)$. 
Equation~\xrefn{eq:f is differentiable} says that
the $z$-value of a point on the surface is equal to the $z$-value of a
point on the plane plus a ``little bit,'' namely $\epsilon_1\Delta x +
\epsilon_2\Delta y$. As $(x,y)$ approaches $(x_0,y_0)$, both $\Delta
x$ and $\Delta y$ approach 0, so this little bit $\epsilon_1\Delta x +
\epsilon_2\Delta y$ also approaches 0, and the $z$-values on the
surface and the plane get close to each other. But that by itself is
not very interesting: since the surface and the plane both contain the
point $(x_0,y_0,z_0)$, the $z$ values will approach $z_0$ and hence
get close to each other whether the tangent plane is ``tangent'' to
the surface or not. The extra condition in the definition says that as
$(x,y)$ approaches $(x_0,y_0)$, the $\epsilon$ values approach
0---this means that $\epsilon_1\Delta x + \epsilon_2\Delta y$
approaches 0 much, much faster, because $\epsilon_1\Delta x$ is much
smaller than either $\epsilon_1$ or $\Delta x$. It is this extra
condition that makes the plane a tangent plane.

We can see that the extra condition on $\epsilon_1$ and $\epsilon_2$
is just what is needed if we look at partial derivatives. Suppose we
temporarily fix $y=y_0$, so $\Delta y=0$. Then the equation from the
definition becomes
$$\Delta z=f_x(x_0,y_0)\Delta x+\epsilon_1\Delta x$$
or
$${\Delta z\over\Delta x}=f_x(x_0,y_0)+\epsilon_1.$$
Now taking the limit of the two sides as $\Delta x$ approaches 0, the
left side turns into the partial derivative of $z$ with respect to
$x$ at $(x_0,y_0)$, or in other words $f_x(x_0,y_0)$, and the right
side does the same, because as $(x,y)$ approaches $(x_0,y_0)$,
$\epsilon_1$ approaches 0. Essentially the same calculation works for 
$f_y$.

\exercises

\exercise Find $f_x$ and $f_y$ where $\ds f(x,y)=\cos(x^2y)+y^3$.
\answer $-2xy\sin(x^2y)$, $-x^2\sin(x^2y)+3y^2$
\endanswer
\endexercise

\exercise Find $f_x$ and $f_y$ where $\ds f(x,y)={xy\over x^2+y}$.
\answer $(y^2-x^2y)/(x^2+y)^2$, $x^3/(x^2+y)^2$
\endanswer
\endexercise

\exercise Find $f_x$ and $f_y$ where $\ds f(x,y)=e^{x^2+y^2}$.
\answer $2xe^{x^2+y^2}$, $2ye^{x^2+y^2}$
\endanswer
\endexercise

\exercise Find $f_x$ and $f_y$ where $\ds f(x,y)=xy\ln(xy)$.
\answer $y\ln(xy)+y$, $x\ln(xy)+x$
\endanswer
\endexercise

\exercise Find $f_x$ and $f_y$ where $\ds f(x,y)=\sqrt{1-x^2-y^2}$.
\answer $-x/\sqrt{1-x^2-y^2}$, $-y/\sqrt{1-x^2-y^2}$
\endanswer
\endexercise

\exercise Find $f_x$ and $f_y$ where $\ds f(x,y)=x\tan(y)$.
\answer $\tan y$, $x\sec^2 y$
\endanswer
\endexercise

\exercise Find $f_x$ and $f_y$ where $\ds f(x,y)={1\over xy}$.
\answer $-1/(x^2y)$, $-1/(xy^2)$
\endanswer
\endexercise

\exercise Find an equation for the plane tangent to 
$\ds 2x^2+3y^2-z^2=4$ at
$(1,1,-1)$. 
\answer $z=-2(x-1)-3(y-1)-1$
\endanswer
\endexercise

\exercise Find an equation for the plane tangent to 
$\ds f(x,y)=\sin(xy)$ at
$(\pi,1/2,1)$. 
\answer $z=1$
\endanswer
\endexercise

\exercise Find an equation for the plane tangent to 
$\ds f(x,y)=x^2+y^3$ at
$(3,1,10)$. 
\answer $z=6(x-3)+3(y-1)+10$
\endanswer
\endexercise

\exercise Find an equation for the plane tangent to 
$\ds f(x,y)=x\ln(xy)$ at
$(2,1/2,0)$. 
\exrdef{ex:ln tan plane}
\answer $z=(x-2)+4(y-1/2)$
\endanswer
\endexercise

\exercise Find an equation for the line normal to 
$\ds x^2+4y^2=2z$ at
$(2,1,4)$. 
\answer ${\bf r}(t)=\langle 2,1,4\rangle+t\langle 2,4,-1\rangle$
\endanswer
\endexercise

\exercise Explain in your own words why, when taking a partial derivative
  of a function of multiple variables, we can treat the variables not
  being differentiated as constants.
\endexercise

\exercise Consider a differentiable function, $f(x,y)$.  Give physical
  interpretations of the meanings of $f_x(a,b)$ and $f_y(a,b)$ as they
  relate to the graph of $f$.
\endexercise

\exercise In much the same way that we used the tangent line to
  approximate the value of a function from single variable calculus,
  we can use the tangent plane to approximate a function from
  multivariable calculus.  Consider the tangent plane found in
  Exercise~\xrefn{ex:ln tan plane}. Use this plane to approximate
  $f(1.98, 0.4)$.
\endexercise

\exercise Suppose that one of your colleagues has calculated the partial
  derivatives of a given function, and reported to you that
  $f_x(x,y)=2x+3y$ and that $f_y(x,y)=4x+6y$.  Do you believe them?
  Why or why not?  If not, what answer might you have accepted for
  $f_y$?
\endexercise

\exercise Suppose $f(t)$ and $g(t)$ are single variable differentiable
  functions.  Find $\partial z/\partial x$ and
  $\partial z/\partial y$ for each of the following two variable functions.

\beginlist
\item{a.} $z=f(x)g(y)$
\item{b.} $z=f(xy)$
\item{c.} $z=f(x/y)$
\endlist

\endexercise

\endexercises

