\section{Derivatives of the exponential and\titlebreak logarithmic functions}
{Derivatives of the exponential and logarithmic functions}{}
\nobreak
As with the sine, we don't know anything about derivatives that allows
us to compute the derivatives of the exponential and logarithmic
functions without going back to basics. Let's do a little work with
the definition again:
$$\eqalign{
{d\over dx}a^x&=\lim_{\Delta x\to 0} {a^{x+\Delta x}-a^x\over \Delta x}\cr
&=\lim_{\Delta x\to 0} {a^xa^{\Delta x}-a^x\over \Delta x}\cr
&=\lim_{\Delta x\to 0} a^x{a^{\Delta x}-1\over \Delta x}\cr
&=a^x\lim_{\Delta x\to 0} {a^{\Delta x}-1\over \Delta x}\cr
}$$
There are two interesting things to note here: As in the case of the
sine function we are left with a limit that involves $\Delta x$ but
not $x$, which means that whatever $\ds \lim_{\Delta x\to 0} (a^{\Delta
    x}-1)/\Delta x$ is, we know that it is a number, that is, a
constant. This means that $\ds a^x$ has a remarkable property: its
derivative is a constant times itself.

We earlier remarked that the hardest limit we would compute is
$\ds \lim_{x\to0}\sin x/x=1$; we now have a limit that is just a bit too
hard to include here. In fact the hard part is to see that
$\ds \lim_{\Delta x\to 0} (a^{\Delta
    x}-1)/\Delta x$ even exists---does this fraction really get closer
and closer to some fixed value? Yes it does, but we will not prove
this fact.

We can look at some examples. Consider $\ds (2^x-1)/x$ for some small
values of $x$: 1, $0.828427124$, $0.756828460$, $0.724061864$,
$0.70838051$, $0.70070877$ when $x$ is 1, $1/2$, $1/4$, $1/8$, $1/16$,
$1/32$, respectively. It looks like this is settling in around $0.7$,
which turns out to be true (but the limit is not exactly $0.7$).
Consider next $\ds (3^x-1)/x$: $2$,  $1.464101616$,
$1.264296052$, $1.177621520$, $1.13720773$, $1.11768854$, at the same
values of $x$. It turns out to be true that in the limit this
is about $1.1$. Two examples don't establish a pattern, but if you do
more examples you will find that the limit varies directly with the
value of $a$: bigger $a$, bigger limit; smaller $a$, smaller limit. As
we can already see, some of these limits will be less than 1 and some
larger than 1. Somewhere between $a=2$ and $a=3$ the limit will be
exactly 1; the value at which this happens is called $e$, so that
$$\lim_{\Delta x\to 0} {e^{\Delta x}-1\over \Delta x}=1.$$ As you
might guess from our two examples, $e$ is closer to 3 than to 2, and
in fact $e\approx 2.718$.

Now we see that the function $\ds e^x$ has a truly remarkable property:
$$\eqalign{
{d\over dx}e^x&=\lim_{\Delta x\to 0} {e^{x+\Delta x}-e^x\over \Delta x}\cr
&=\lim_{\Delta x\to 0} {e^xe^{\Delta x}-e^x\over \Delta x}\cr
&=\lim_{\Delta x\to 0} e^x{e^{\Delta x}-1\over \Delta x}\cr
&=e^x\lim_{\Delta x\to 0} {e^{\Delta x}-1\over \Delta x}\cr
&=e^x\cr
}$$
That is, $\ds e^x$ is its own derivative, or in other words the
slope of $\ds e^x$ is the same as its height, or the same as its second
coordinate: The function $\ds f(x)=e^x$ goes through the point $\ds (z,e^z)$
and has slope $\ds e^z$ there, no matter what $z$ is. It is sometimes
convenient to express the function $\ds e^x$ without an exponent, since
complicated exponents can be hard to read. In such cases we use
$\exp(x)$, e.g., $\ds \exp(1+x^2)$ instead of 
$\ds e^{1+x^2}$.\index{exp function}

What about the logarithm function? This too is hard, but as the
cosine function was easier to do once the sine was done, so the
logarithm is easier to do now that we know the derivative of the
exponential function. Let's start with $\ds \log_e x$, which as you
probably know is often abbreviated $\ln x$ and called the ``natural
logarithm'' function.

Consider the relationship between the two functions,
namely, that they are inverses, that one ``undoes'' the
other. Graphically this means that they have the same graph except
that one is ``flipped'' or ``reflected'' through the line $y=x$, as
shown in figure~\xrefn{fig:exponential and log functions}.

\figure
\texonly
\vbox{\beginpicture
\normalgraphs
\ninepoint
\setcoordinatesystem units <1truecm,1truecm> point at 0 0
\setplotarea x from -2 to 2, y from 0 to 4
\axis left shiftedto x=0 /
\axis bottom shiftedto y=0 /
\setquadratic
\plot  -2.000 0.250 -1.867 0.274 -1.733 0.301 -1.600 0.330 -1.467 0.362 
-1.333 0.397 -1.200 0.435 -1.067 0.477 -0.933 0.524 -0.800 0.574 
-0.667 0.630 -0.533 0.691 -0.400 0.758 -0.267 0.831 -0.133 0.912 
0.000 1.000 0.133 1.097 0.267 1.203 0.400 1.320 0.533 1.447 
0.667 1.587 0.800 1.741 0.933 1.910 1.067 2.095 1.200 2.297 
1.333 2.520 1.467 2.764 1.600 3.031 1.733 3.325 1.867 3.647 
2.000 4.000 /
\setcoordinatesystem units <1truecm,1truecm> point at -5 -2
\setplotarea x from 0 to 4, y from -2 to 2
\axis left shiftedto x=0 /
\axis bottom shiftedto y=0 /
\setquadratic
\plot 0.250 -2.000 0.375 -1.415 0.500 -1.000 0.625 -0.678 0.750 -0.415 
0.875 -0.193 1.000 0.000 1.125 0.170 1.250 0.322 1.375 0.459 
1.500 0.585 1.625 0.700 1.750 0.807 1.875 0.907 2.000 1.000 
2.125 1.087 2.250 1.170 2.375 1.248 2.500 1.322 2.625 1.392 
2.750 1.459 2.875 1.524 3.000 1.585 3.125 1.644 3.250 1.700 
3.375 1.755 3.500 1.807 3.625 1.858 3.750 1.907 3.875 1.954 
4.000 2.000 /
\endpicture}
\begincaption
The exponential and logarithm functions.
\endcaption
\endtexonly
\htmlfigure{Transcendental-ExpLogInverses.html}
\figrdef{fig:exponential and log functions}
\htmlonly
\begincaption
The exponential (green) and logarithmic (blue) functions.  As inverses
of each other, their graphs are reflections of each other across the
line $y=x$ (dashed).
\endcaption
\endhtmlonly
\endfigure

\noindent 
This means that the slopes of these two functions are closely related
as well: For example, the slope of $\ds e^x$ is $e$ at $x=1$; at the
corresponding point on the $\ln(x)$ curve, the slope must be $1/e$,
because the ``rise'' and the ``run'' have been interchanged. Since the
slope of $\ds e^x$ is $e$ at the point $(1,e)$, the slope of $\ln(x)$ is
$1/e$ at the point $(e,1)$.

\figure
\texonly
\vbox{\beginpicture
\normalgraphs
\ninepoint
\setcoordinatesystem units <1truecm,1truecm> point at 0 0
\setplotarea x from -2 to 2, y from 0 to 4
\axis left shiftedto x=0 /
\axis bottom shiftedto y=0 /
\setquadratic
\plot  -2.000 0.250 -1.867 0.274 -1.733 0.301 -1.600 0.330 -1.467 0.362 
-1.333 0.397 -1.200 0.435 -1.067 0.477 -0.933 0.524 -0.800 0.574 
-0.667 0.630 -0.533 0.691 -0.400 0.758 -0.267 0.831 -0.133 0.912 
0.000 1.000 0.133 1.097 0.267 1.203 0.400 1.320 0.533 1.447 
0.667 1.587 0.800 1.741 0.933 1.910 1.067 2.095 1.200 2.297 
1.333 2.520 1.467 2.764 1.600 3.031 1.733 3.325 1.867 3.647 
2.000 4.000 /
\setlinear
\plot -0.443 0 2 3.386 /
\setcoordinatesystem units <1truecm,1truecm> point at -5 -2
\setplotarea x from 0 to 4, y from -2 to 2
\axis left shiftedto x=0 /
\axis bottom shiftedto y=0 /
\setquadratic
\plot 0.250 -2.000 0.375 -1.415 0.500 -1.000 0.625 -0.678 0.750 -0.415 
0.875 -0.193 1.000 0.000 1.125 0.170 1.250 0.322 1.375 0.459 
1.500 0.585 1.625 0.700 1.750 0.807 1.875 0.907 2.000 1.000 
2.125 1.087 2.250 1.170 2.375 1.248 2.500 1.322 2.625 1.392 
2.750 1.459 2.875 1.524 3.000 1.585 3.125 1.644 3.250 1.700 
3.375 1.755 3.500 1.807 3.625 1.858 3.750 1.907 3.875 1.954 
4.000 2.000 /
\setlinear
\plot 0 -0.443 3.386 2 /
\endpicture}
\begincaption
Slope of the exponential and logarithm functions.
\endcaption
\endtexonly
\htmlfigure{Transcendental-ExpLogSlopes.html}
\figrdef{fig:slope of exponential and log functions}
\htmlonly
\begincaption
The exponential (green) and logarithmic (blue) functions.  The dashed
lines indicate the slope of the respective functions at the points
$(1,e)$ and $(e,1)$.  It is interesting to note that these lines
interesect at the origin.
\endcaption
\endhtmlonly
\endfigure

More generally, we know that the slope of $\ds e^x$ is $\ds e^z$ at
the point $\ds (z,e^z)$, so the slope of $\ln(x)$ is $\ds 1/e^z$ at
$\ds (e^z,z)$, as indicated in figure~\xrefn{fig:slope of exponential and
log functions}.  In other words, the slope of $\ln x$ is the
reciprocal of the first coordinate at any point; this means that the
slope of $\ln x$ at $(x,\ln x)$ is $1/x$. The upshot is:
$${d\over dx}\ln x = {1\over x}.$$
We have discussed this from the point of view of the graphs, which is
easy to understand but is not normally considered a rigorous
proof---it is too easy to be led astray by pictures that seem
reasonable but that miss some hard point. It is possible to do this
derivation without resorting to pictures, and indeed we will see an
alternate approach soon.

Note that $\ln x$ is defined only for $x>0$. It is sometimes useful to
consider the function $\ln |x|$, a function defined for
$x\not=0$. When $x<0$, $\ln |x|=\ln(-x)$ and 
$${d\over dx}\ln |x|={d\over dx}\ln (-x)={1\over -x}(-1)={1\over x}.$$
Thus whether $x$ is positive or negative, the derivative is the same.

What about the functions $\ds a^x$ and $\ds \log_a x$? We know that the
derivative of $\ds a^x$ is some constant times $\ds a^x$ itself, but what
constant? Remember that ``the logarithm is the exponent'' and you will
see that $\ds a=e^{\ln a}$. Then
$$a^x = (e^{\ln a})^x = e^{x\ln a},$$
and we can compute the derivative using the chain rule:
$${d\over dx} a^x = {d\over dx}(e^{\ln a})^x = {d\over dx}e^{x\ln a} = 
(\ln a)e^{x\ln a} =(\ln a)a^x.$$
The constant is simply $\ln a$. Likewise we can compute the derivative
of the logarithm function $\ds \log_a x$. Since
$$x=e^{\ln x}$$
we can take the logarithm base $a$ of both sides to get
$$
\log_a(x)=\log_a(e^{\ln x})=\ln x \log_a e.
$$
Then
$${d\over dx}\log_a x = {1\over x}\log_a e.$$
This is a perfectly good answer, but we can improve it slightly.
Since 
$$\eqalign{
a&=e^{\ln a}\cr
\log_a(a) &= \log_a(e^{\ln a}) = \ln a\log_a e\cr
1&=\ln a\log_a e\cr
{1\over \ln a}&=\log_a e,\cr
}$$
we can replace $\ds \log_a e$ to get
$${d\over dx}\log_a x = {1\over x\ln a}.$$

You may if you wish memorize the formulas
$${d\over dx}a^x = (\ln a)a^x \quad \hbox{and}\quad
{d\over dx}\log_a x = {1\over x\ln a}.$$
Because the ``trick'' $\ds a=e^{\ln a}$ is often useful, and sometimes
essential, it may be better to remember the trick, not the formula.

\example
Compute the derivative of $\ds f(x)=2^x$.
$$\eqalign{
{d\over dx}2^{x} &= {d\over dx}(e^{\ln 2})^x\cr
&= {d\over dx}e^{x\ln 2}\cr
&= \left({d\over dx} x\ln 2\right) e^{x\ln 2}\cr
&= (\ln 2)  e^{x\ln 2}=2^x\ln2\cr
}$$
\vskip-10pt
\endexample

\example
Compute the derivative of $\ds f(x)=2^{x^2}=2^{(x^2)}$.
$$\eqalign{
{d\over dx}2^{x^2} &= {d\over dx}e^{x^2\ln 2}\cr
&= \left({d\over dx} x^2\ln 2\right) e^{x^2\ln 2}\cr
&= (2\ln 2) x  e^{x^2\ln 2}\cr
&=  (2\ln 2) x 2^{x^2}\cr
}$$
\vskip-10pt
\endexample

\example
Compute the derivative of $\ds f(x)=x^x$. At first this appears to be a
new kind of function: it is not a constant power of $x$, and it does
not seem to be an exponential function, since the base is not
constant. But in fact it is no harder than the previous example.
$$\eqalign{
{d\over dx}x^x&={d\over dx}e^{x\ln x}\cr
&=\left({d\over dx}x\ln x\right)e^{x\ln x}\cr
&=(x{1\over x}+\ln x)x^x\cr
&=(1+\ln x)x^x\cr
}$$
\vskip-10pt
\endexample

\example Recall that we have not justified the power\index{power rule}
rule except when the exponent is a positive or negative integer. We
can use the exponential function to take care of other exponents.
$$\eqalign{
{d\over dx}x^r&={d\over dx}e^{r\ln x}\cr
&=\left({d\over dx}r\ln x\right)e^{r\ln x}\cr
&=(r{1\over x})x^r\cr
&=rx^{r-1}\cr
}$$
\vskip-10pt
\endexample

\exercises
In 1--19, find the derivatives of the functions.

\twocol

\exercise $\ds 3^{x^2}$
\answer $\ds 2\ln(3)x3^{x^2}$
\endanswer
\endexercise

\exercise $\ds {\sin x \over e^x}$
\answer $\ds {\cos x-\sin x \over e^x}$
\endanswer
\endexercise

\exercise $\ds (e^x)^2$
\answer $\ds 2e^{2x}$
\endanswer
\endexercise

\exercise $\ds \sin(e^x)$
\answer $\ds e^x\cos(e^x)$
\endanswer
\endexercise

\exercise $\ds e^{\sin x}$
\answer $\ds  \cos (x) e^{\sin x}$
\endanswer
\endexercise

\exercise $\ds x^{\sin x}$
\answer $\ds x^{\sin x}\left(\cos x\ln x+{\sin x\over x}\right)$
\endanswer
\endexercise

\exercise  $\ds x^3e^x$
\answer $\ds 3x^2e^x+x^3e^x$
\endanswer
\endexercise

\exercise  $\ds x+2^x$
\answer $\ds 1+2^x\ln(2)$
\endanswer
\endexercise

\exercise  $\ds (1/3)^{x^2}$
\answer $\ds -2x\ln(3)(1/3)^{x^2}$
\endanswer
\endexercise

\exercise  $\ds e^{4x}/x$
\answer $\ds e^{4x}(4x-1)/x^2$
\endanswer
\endexercise

\exercise  $\ds \ln(x^3+3x)$
\answer $\ds (3x^2+3)/(x^3+3x)$
\endanswer
\endexercise

\exercise  $\ds \ln(\cos(x))$
\answer $\ds -\tan(x)$
\endanswer
\endexercise

\exercise  $\ds\sqrt{\ln(x^2)}/x$
\answer $\ds (1-\ln(x^2))/(x^2\sqrt{\ln(x^2)})$
\endanswer
\endexercise

\exercise  $\ds \ln(\sec(x) + \tan(x))$
\answer $\ds \sec(x)$
\endanswer
\endexercise

\exercise  $\ds x^{\cos(x)}$
\answer $\ds x^{\cos(x)}(\cos(x)/x-\cos(x)\ln(x))$
\endanswer
\endexercise

\exercise $\ds x\ln x$
\endexercise

\exercise $\ln (\ln (3x) )$
\endexercise

\exercise $\ds {1+\ln (3x^2 )\over 1+ \ln(4x)}$
\endexercise

\exercise $\ds {x^8 (x-23)^{1/2}\over 27 x^6(4x-6)^8 }$

\endtwocol
\bsk
\endexercise

\exercise Find the value of $a$ so that the tangent line to
$y=\ln(x)$ at $x=a$ is a line through the origin.  Sketch the
 resulting situation.
\answer $e$
\endanswer
\endexercise

\exercise If $\ds f(x) = \ln(x^3 + 2)$ compute $\ds f'(e^{1/3})$.
\endexercise

\exercise If $\ds y=\log_a x$ then $\ds a^y=x$. Use implicit
differentiation to find $\ds y'$.
\endexercise

\endexercises

