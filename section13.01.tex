\section{Space Curves}{}{}

We have already seen that a convenient way to describe a line in three
dimensions is to provide a vector that ``points to'' every point on
the line as a parameter $t$ varies, like
$$\langle 1,2,3\rangle+t\langle 1,-2,2\rangle
=\langle 1+t,2-2t,3+2t\rangle.$$
Except that this gives a particularly simple geometric object, there
is nothing special about the individual functions of $t$ that make up
the coordinates of this vector---any vector with a parameter, like
$\langle f(t),g(t),h(t)\rangle$, will describe some curve in three
dimensions as $t$ varies through all possible values. 

\example Describe the curves
$\langle \cos t,\sin t,0\rangle$,
$\langle \cos t,\sin t,t\rangle$, and
$\langle \cos t,\sin t,2t\rangle$.
\thmrdef{exam:helixes}

As $t$ varies, the first two coordinates in all three functions
trace out the points on the unit circle, starting
with $(1,0)$ when $t=0$ and proceeding counter-clockwise around the
circle as $t$ increases. In the first case, the $z$ coordinate is
always 0, so this describes precisely the unit circle in the $x$-$y$
plane. In the second case, the $x$ and $y$ coordinates still describe
a circle, but now 
the $z$ coordinate varies, so that the
height of the curve matches the value of $t$. When $t=\pi$, for
example, the resulting vector is $\langle -1,0,\pi\rangle$. A bit of
thought should convince you that the result is a helix. In the third
vector, the $z$ coordinate varies twice as fast as the parameter $t$,
so we get a stretched out helix. Both are shown in
figure~\xrefn{fig:helixes}. On the left is the first helix, shown for
$t$ between 0 and $4\pi$; on the right is the second helix, shown for
$t$ between 0 and $2\pi$. Both start and end at the same point, but
the first helix takes two full ``turns'' to get there, because its $z$
coordinate grows more slowly.
\endexample

\figure
\texonly
\vbox{\beginpicture
\normalgraphs
\ninepoint
\setcoordinatesystem units <3truecm,3truecm>
\setplotarea x from 0 to 1.1, y from 0 to 1.1
\put {\hbox{\epsfxsize5cm\epsfbox{helix1.eps}}} at 0 0
\put {\hbox{\epsfxsize5cm\epsfbox{helix2.eps}}} at 2 0
\endpicture}
\begincaption
Two helixes.
(\expandafter\url\expandafter{\liveurl helixes.html}%
AP\endurl)
\endcaption
\endtexonly
\figrdef{fig:helixes}
\htmlfigure{Vector_Functions_two_helixes.html}
\begincaption
Two helixes.
\endcaption
\endfigure

A vector expression of the form $\langle f(t),g(t),h(t)\rangle$ is called
a {\dfont vector function\index{vector!function}}; it is a function from
the real numbers $\R$ to the set of all three-dimensional vectors.
We can alternately think of it as three separate functions, 
$x=f(t)$, $y=g(t)$, and $z=h(t)$, that describe points in space. In
this case we usually refer to the set of equations as {\dfont
parametric equations\index{parametric equations}} for the curve, just
as for a line. While the parameter $t$ in a vector function might
represent any one of a number of physical quantities, or be simply a
``pure number'', it is often convenient and useful to think of $t$ as
representing time. The vector function then tells you where in space
a particular object is at any time.

Vector functions can be difficult to understand, that is, difficult to
picture. When available, computer software can be very helpful. When
working by hand, one useful approach is to consider the
``projections'' of the curve onto the three standard coordinate
planes. We have already done this in part: in
example~\xrefn{exam:helixes} we noted that all three curves project to
a circle in the $x$-$y$ plane, since $\langle \cos t,\sin t\rangle$ is
a two dimensional vector function for the unit circle.

\example Graph the projections of $\langle \cos t,\sin
t,2t\rangle$ onto the $x$-$z$ plane and the $y$-$z$ plane.
The two dimensional vector function for the
projection onto the $x$-$z$ plane is $\langle \cos t, 2t\rangle$, or in
parametric form, $x=\cos t$, $z=2t$. By eliminating $t$ we get the
equation $x=\cos(z/2)$, the familiar curve shown on the left in
figure~\xrefn{fig:helix projections}. For the projection onto the $y$-$z$ plane, we start
with the vector function $\langle \sin t, 2t\rangle$, which is the
same as $y=\sin t$, $z=2t$. Eliminating $t$ gives $y=\sin(z/2)$, as
shown on the right in figure~\xrefn{fig:helix projections}.
\endexample


\figure
\texonly
\vbox{\beginpicture
\normalgraphs
\ninepoint
\setcoordinatesystem units <15truemm,4truemm>
\setplotarea x from -1.25 to 1.25, y from 0 to 14
\axis left shiftedto x=0 ticks length <2pt> withvalues {$2\pi$} {$4\pi$} /
  at 6.28 12.57 / /
\axis bottom ticks length <2pt> numbered from -1 to 1 by 1 /
\arrow <4pt> [0.35, 1] from 0 14 to 0 14.05
\arrow <4pt> [0.35, 1] from 1.25 0 to 1.3 0
\put {$x$} [tl] <3pt,0pt> at 1.3 0 
\put {$z$} [br] <0pt,3pt> at 0 14.05 
\plot 
1.000 0.000 0.995 0.209 0.978 0.419 0.951 0.628 0.914 0.838 0.866
1.047 0.809 1.257 0.743 1.466 0.669 1.676 0.588 1.885 0.500 2.094
0.407 2.304 0.309 2.513 0.208 2.723 0.105 2.932 0.000 3.142 -0.105
3.351 -0.208 3.560 -0.309 3.770 -0.407 3.979 -0.500 4.189 -0.588 4.398
-0.669 4.608 -0.743 4.817 -0.809 5.027 -0.866 5.236 -0.914 5.445
-0.951 5.655 -0.978 5.864 -0.995 6.074 -1.000 6.283 -0.995 6.493
-0.978 6.702 -0.951 6.912 -0.914 7.121 -0.866 7.330 -0.809 7.540
-0.743 7.749 -0.669 7.959 -0.588 8.168 -0.500 8.378 -0.407 8.587
-0.309 8.796 -0.208 9.006 -0.105 9.215 0.000 9.425 0.105 9.634 0.208
9.844 0.309 10.053 0.407 10.263 0.500 10.472 0.588 10.681 0.669 10.891
0.743 11.100 0.809 11.310 0.866 11.519 0.914 11.729 0.951 11.938 0.978
12.147 0.995 12.357 1.000 12.566 /
\setcoordinatesystem units <15truemm,4truemm> point at -4 0
\setplotarea x from -1.25 to 1.25, y from 0 to 14
\axis left shiftedto x=0 ticks length <2pt>  withvalues {$2\pi$} {$4\pi$} /
  at 6.28 12.57 / /
\axis bottom ticks length <2pt> numbered from -1 to 1 by 1 /
\arrow <4pt> [0.35, 1] from 0 14 to 0 14.05
\arrow <4pt> [0.35, 1] from 1.25 0 to 1.3 0
\put {$y$} [tl] <3pt,0pt> at 1.3 0 
\put {$z$} [br] <0pt,3pt> at 0 14.05 
\plot 0.000 0.000 0.105 0.209 0.208 0.419 0.309 0.628 0.407 0.838 0.500
 1.047 0.588 1.257 0.669 1.466 0.743 1.676 0.809 1.885 0.866 2.094
 0.914 2.304 0.951 2.513 0.978 2.723 0.995 2.932 1.000 3.142 0.995
 3.351 0.978 3.560 0.951 3.770 0.914 3.979 0.866 4.189 0.809 4.398
 0.743 4.608 0.669 4.817 0.588 5.027 0.500 5.236 0.407 5.445 0.309
 5.655 0.208 5.864 0.105 6.074 0.000 6.283 -0.105 6.493 -0.208 6.702
 -0.309 6.912 -0.407 7.121 -0.500 7.330 -0.588 7.540 -0.669 7.749
 -0.743 7.959 -0.809 8.168 -0.866 8.378 -0.914 8.587 -0.951 8.796
 -0.978 9.006 -0.995 9.215 -1.000 9.425 -0.995 9.634 -0.978 9.844
 -0.951 10.053 -0.914 10.263 -0.866 10.472 -0.809 10.681 -0.743 10.891
 -0.669 11.100 -0.588 11.310 -0.500 11.519 -0.407 11.729 -0.309 11.938
 -0.208 12.147 -0.105 12.357 0.000 12.566 /
\endpicture}
\endtexonly
\figrdef{fig:helix projections}
\htmlfigure{Vector_Functions_helix_projections.html}
\begincaption
The projections of $\langle \cos t,\sin
t,2t\rangle$ onto the $x$-$z$ and $y$-$z$ planes.
\endcaption
\endfigure

\exercises

\exercise Investigate the curve ${\bf r}=\langle \sin t,\cos t,\cos
8t\rangle$.
\endexercise

\exercise Investigate the curve 
${\bf r}=\langle t\cos t,t\sin t,t\rangle$.
\endexercise

\exercise Investigate the curve 
${\bf r}=\langle t,t^2,\cos t\rangle$.
\endexercise

\exercise Investigate the curve 
${\bf r}=\langle \cos(20t)\sqrt{1-t^2},\sin(20t)\sqrt{1-t^2},t\rangle$
\endexercise

\exercise Find a vector function for the curve of intersection of
$x^2+y^2=9$ and $y+z=2$.
\answer $\langle 3\cos t, 3\sin t, 2-3\sin t\rangle$
\endanswer
\endexercise

\exercise A bug is crawling outward along the spoke of a wheel that lies along
a radius of the wheel. The bug is crawling at 1 unit per second and
the wheel is rotating at 1 radian per second. Suppose the wheel lies
in the $y$-$z$ plane with center at the origin, and at time $t=0$ the
spoke lies along the positive $y$ axis and the bug is at the origin.
Find a vector function ${\bf r}(t)$
for the position of the bug at time $t$.
\answer $\langle 0,t\cos t,t\sin t\rangle$
\endanswer
\endexercise

\exercise What is the difference between the parametric curves
$f(t)=\langle t, t, t^2 \rangle$, $g(t)=\langle t^2, t^2, t^4
\rangle$, and $h(t)=\langle \sin(t), \sin(t), \sin^2(t) \rangle$as $t$
runs over all real numbers?
\endexercise

\exercise Plot each of the curves below in 2 dimensions, projected
onto each of the three standard planes (the $x$-$y$, $x$-$z$, and
$y$-$z$ planes).

\beginlist
\item{a.} $f(t)=\langle t, t^3, t^2 \rangle$, $t$ ranges over all real numbers
\item{b.} $f(t)=\langle t^2, t-1, t^2+5 \rangle$  for $0\leq t \leq 3$
\endlist
\endexercise

\exercise Given points $A=(a_1, a_2, a_3)$ and $B=(b_1, b_2, b_3)$, give
parametric equations for the line {\em segment} connecting $A$ and
$B$. Be sure to give appropriate $t$ values.
\endexercise

\exercise With a parametric plot and a set of $t$ values, we can associate
a `direction'.  For example, the curve $\langle \cos t, \sin t
\rangle$ is the unit circle traced counterclockwise.  How can we amend
a set of given parametric equations and $t$ values to get the same
curve, only traced backwards?
\endexercise

\endexercises

