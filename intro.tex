\chapter{Introduction}{}

The emphasis in this course is on problems---doing calculations and story
problems.  To master problem solving one needs a tremendous amount of
practice doing problems. The more problems you do the better you will
be at doing them, as patterns will start to emerge in both the
problems and in successful approaches to them. You will learn fastest
and best if you devote some time to doing problems every day.

Typically the most difficult problems are story problems, since they
require some effort before you can begin calculating.
Here are some pointers for doing story problems:

\beginlist

\item{\question}
Carefully read each problem twice before writing anything.

\item {\question}
Assign letters to quantities that are described only in words;
draw a diagram if appropriate.

\item{\question}
Decide which letters are constants and which are variables.  A letter
stands for a constant if its value remains the same throughout the problem.

\item{\question}
Using mathematical notation, write down what you know and then write down
what you want to find.

\item{\question}
Decide what category of problem it is (this might be obvious if the
problem comes at the end of a particular chapter, but will not necessarily
be so obvious if it comes on an exam covering several chapters).

\item{\question}
Double check each step as you go along; don't wait until the end to
check your work.

\item{\question}
Use common sense; if an answer is out of the range of practical
possibilities, then check your work to see where you went wrong.

\endlist

\bsk
\leftline{\bf Suggestions for Using This Text}

\beginlist
\item{\question}
Read the example problems carefully, filling in any steps that are
left out (ask someone for help if you can't follow the solution to a worked
example).

\item {\question}
Later use the worked examples to study by covering the solutions,
and seeing if you can solve the problems on your own.

\item{\question} Most exercises have answers in
  Appendix~\xrefn{chap:answers}; the availability of an answer is
  marked by ``$\Rightarrow$'' at the end of the exercise. In the pdf
  version of the full text, clicking on the arrow will take you to the
  answer.  The answers should be used only as a final check on your
  work, not as a crutch.  Keep in mind that sometimes an answer could
  be expressed in various ways that are algebraically equivalent, so
  don't assume that your answer is wrong just because it doesn't have
  exactly the same form as the answer in the back.

\item{\question} A few figures in the pdf and print versions of the
  book are marked with ``(AP)'' at the end of the caption. Clicking on
  this should open a related interactive applet or Sage worksheet in
  your web browser. Occasionally another link will do the same thing,
  like \expandafter\url\expandafter{\sageurl animated_cycloid}% 
  this example.\endurl\ (Note to users of a printed text: the words
  ``this example'' in the pdf file are blue, and are a link to a Sage
  worksheet.)  \endlist
