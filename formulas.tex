\htmlonly
\chapter{Useful formulas}
\chaprdef{chap:formulas}
\endhtmlonly
\msk\lefthead{\txtbold Algebra}

\item{} Remember that the common algebraic operations have {\dfont
  precedences\index{algebraic precedence}\index{precedence!of
  algebraic operations}\/} relative to each other: for example,
  multiplication and division take precedence over addition and
  subtraction, but are ``tied'' with each other. In the case of ties,
  work left to right.  This means, for example, that $1/2x$ means
  $(1/2)x$: do the division, then the multiplication in left to right
  order. It sometimes is a good idea to use more parentheses than
  strictly necessary, for clarity, but it is also a bad idea to use
  too many parentheses.

\msk Completing the square\index{completing the square}:
    $x^2+bx+c=(x+{b\over 2})^2-{b^2\over 4}+c$.

\msk Quadratic formula\index{quadratic formula}: the roots of $ax^2+bx+c$ are
$\ds {-b\pm\sqrt{b^2-4ac}\over 2a}$.

\msk Exponent rules:
$$\eqalign{a^b\cdot a^c&=a^{b+c}\cr
{a^b\over a^c}&=a^{b-c}\cr
(a^b)^c&=a^{bc}\cr
a^{1/b}&={\root b \of a}\cr}$$

\msk\lefthead{\txtbold Geometry}

\msk Circle:\index{circle!circumference}\index{circle!area}
 $\hbox{circumference}=2\pi r$, $\hbox{area}=\pi r^2$.

\msk Sphere:\index{sphere!surface area}\index{sphere!volume}
$\hbox{vol}=4\pi r^3/3$, $\hbox{surface area}=4\pi r^2$.

\msk Cylinder:\index{cylinder!lateral area}\index{cylinder!volume}
\index{cylinder!surface area}
 $\hbox{vol} =\pi r^2h$, $\hbox{lateral area}= 2\pi rh$,
$\hbox{total surface area} =2\pi rh+2\pi r^2$. 

\msk\item{} Cone:\index{cone!lateral area}\index{cone!volume}
\index{cone!surface area}
  $\hbox{vol}=\pi r^2h/3$, 
$\hbox{lateral area}=\pi r\sqrt{r^2+h^2}$,
$\hbox{total surface area}= \pi r\sqrt{r^2+h^2}+\pi r^2$.

\msk\lefthead{\txtbold Analytic geometry}
\nobreak
\msk\item{} Point-slope formula\index{point-slope formula} 
for straight line through the point
$(x_0,y_0)$ with slope $m$: $y=y_0+m(x-x_0)$.

\msk Circle with radius\index{circle!equation of}
$r$ centered at $(h,k)$:
$(x-h)^2+(y-k)^2=r^2$.

\msk\item{} Ellipse\index{ellipse!equation of}
 with axes on the $x$-axis and
$y$-axis: $\ds {x^2\over a^2}+{y^2\over b^2}=1$.

\msk\lefthead{\txtbold Trigonometry}

\msk$\sin(\theta) = \hbox{opposite}/\hbox{hypotenuse}$

\msk $\cos(\theta)= \hbox{adjacent}/\hbox{hypotenuse}$

\msk $\tan(\theta)= \hbox{opposite}/\hbox{adjacent}$

\msk $\sec(\theta)= 1/\cos(\theta)$
\msk $\csc(\theta)= 1/\sin(\theta)$
\msk $\cot(\theta)= 1/\tan(\theta)$
\msk $\tan(\theta)= \sin(\theta)/\cos(\theta)$
\msk $\cot(\theta)= \cos(\theta)/\sin(\theta)$

\msk \index{trigonometric identities}
$\sin (\theta)=\cos\left({\pi\over 2}-\theta\right)$

\msk $\cos (\theta)=\sin\left({\pi\over 2}-\theta\right)$

\msk $\sin(\theta+\pi)=-\sin (\theta)$
\msk $\cos(\theta+\pi)=-\cos (\theta)$

\msk Law of cosines\index{cosines!law of}\index{law of cosines}: $a^2=b^2+c^2-2bc\cos A$

\msk Law of sines\index{sines!law of}\index{law of sines}: $\ds{a\over\sin A}={b\over \sin B}={c\over \sin C}$

\msk Sine of sum of angles: $\sin(x+y)=\sin x\cos y+\cos x\sin y$

\msk Sine of double angle: $\sin(2x)=2\sin x\cos x$

\msk Sine of difference of angles: $\sin(x-y)=\sin x\cos y-\cos x\sin y$

\msk Cosine of sum of angles: $\cos (x+y)=\cos x\cos y-\sin x\sin y$

\msk Cosine of double angle: $\cos (2x)=\cos^2 x-\sin^2 x = 
2\cos^2 x-1 = 1-2\sin^2x$

\msk Cosine of difference of angles: $\cos (x-y)=\cos x\cos y+\sin x\sin y$

\msk Tangent of sum of angles: $\ds \tan (x+y)={\tan x+\tan y\over
1-\tan x\tan y}$

\msk $\sin^2(\theta)$ and $\cos^2(\theta)$ formulas: 
$$\eqalign{\sin^2(\theta)+ \cos^2(\theta)&=1\cr
\tan^2(\theta)+ 1&=\sec^2(\theta)\cr
\sin^2(\theta)&={1-\cos(2\theta)\over 2}\cr
\cos^2(\theta)&={1+\cos(2\theta)\over 2}\cr}$$

