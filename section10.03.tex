\section{Areas in polar coordinates}{}{}
\nobreak
We can use the equation of a curve in polar coordinates to compute
some areas bounded by such curves.
The basic approach is the same as with any application of integration:
find an approximation that approaches the true value. For areas in
rectangular coordinates, we approximated the region using rectangles;
in polar coordinates, we use sectors of circles, as depicted in
figure~\xrefn{fig:approximating area with sectors}. Recall that the
area of a sector of a circle is $\ds \alpha r^2/2$, where $\alpha$ is the
angle subtended by the sector. If the curve is given by $r=f(\theta)$,
and the angle subtended by a small sector is $\Delta\theta$, 
the area is $\ds (\Delta\theta)(f(\theta))^2/2$.
Thus we approximate the total area as
$$\sum_{i=0}^{n-1} {1\over 2} f(\theta_i)^2\;\Delta\theta.$$
In the limit this becomes
$$\int_a^b {1\over 2} f(\theta)^2\;d\theta.$$

\example
We find the area inside the cardioid $r=1+\cos\theta$.
$$\int_0^{2\pi}{1\over 2} (1+\cos\theta)^2\;d\theta=
{1\over 2}\int_0^{2\pi} 1+2\cos\theta+\cos^2\theta\;d\theta=
{1\over 2}\left.(\theta +2\sin\theta+
{\theta\over2}+{\sin2\theta\over4})\right|_0^{2\pi}={3\pi\over2}.$$
\endexample

\figure
\texonly
\vbox{\beginpicture
\normalgraphs
\ninepoint
\setcoordinatesystem units <20truemm,20truemm>
\setplotarea x from 0 to 2.5, y from 0  to 1.5
\axis left shiftedto x=0 /
\axis bottom shiftedto y=0 /
\setquadratic
\plot 1.941 0.393 1.900 0.506 1.850 0.613 1.791 0.715 1.723 0.811
1.648 0.900 1.565 0.981 1.476 1.054 1.382 1.118 1.284 1.172
1.182 1.217 1.078 1.252 0.973 1.278 0.867 1.293 0.762 1.299
0.659 1.295 0.559 1.283 0.462 1.262 0.369 1.233 0.281 1.196
0.199 1.153 /
\setlinear
\plot 0 0 0.75 1.299 /
\circulararc -15 degrees from 0.75 1.299 center at 0 0 
\plot 0 0 1.207 1.207 /
\circulararc -15 degrees from 1.207 1.207 center at 0 0
\plot 0 0 1.616 0.933 / 
\circulararc -15 degrees from 1.616 0.933 center at 0 0
\plot 0 0 1.9 0.51 /
\endpicture}
\endtexonly
\figrdef{fig:approximating area with sectors}
\htmlfigure{Polar-Parametric-area_in_polar.html}
\begincaption
Approximating area by sectors of circles.
\endcaption
\endfigure

\example We find the area between the circles $r=2$ and
$r=4\sin\theta$, as shown in figure~\xrefn{fig:polar area between curves}.
The two curves intersect where $2=4\sin\theta$, or $\sin\theta=1/2$,
so $\theta=\pi/6$ or $5\pi/6$. The area we want is then
$$
  {1\over2}\int_{\pi/6}^{5\pi/6}
  16\sin^2\theta-4\;d\theta={4\over3}\pi + 2\sqrt{3}.
$$
\endexample

\figure
\texonly
\hbox{\hfill\tikzpicture[domain=-2:2,x=6mm,y=6mm]
\draw (-2.1,0) -- (2.1,0) ;
\draw (0,-2.1) -- (0,4.1) ;
\gpad
\draw[color=black] plot[parametric,id=\the\gpnum,domain=0:2*3.1416,samples=50] 
function{2*cos(t),2*sin(t)};
\gpad
\draw[color=black] plot[parametric,id=\the\gpnum,domain=0:3.1416,samples=50] 
function{4*sin(t)*cos(t),4*sin(t)*sin(t)};
\gpad
\fill[opacity=0.5,fill=red!20] 
plot[parametric,id=\the\gpnum,domain=0.5236:2.618]
function{4*sin(t)*cos(t),4*sin(t)*sin(t)} node {\gpad}
plot[parametric,id=\the\gpnum,domain=0.5236:2.618]
function{2*cos(3.1416-t),2*sin(3.1416-t)} -- cycle;
\endtikzpicture\hfill}
\endtexonly
\figrdef{fig:polar area between curves}
\htmlfigure{Polar-Parametric-area_between_polar_curves.html}
\begincaption
An area between curves.
\endcaption
\endfigure

This example makes the process appear more straightforward than it
is. Because points have many different representations in polar
coordinates, it is not always so easy to identify points of
intersection. 

\example We find the shaded area in the first graph of
figure~\xrefn{fig:harder area between polar curves} as the difference
of the other two shaded areas. The cardioid is $r=1+\sin\theta$ and
the circle is $r=3\sin\theta$. We attempt to find points of intersection:
$$\eqalign{
  1+\sin\theta&=3\sin\theta\cr
  1&=2\sin\theta\cr
  1/2&=\sin\theta.\cr}
$$
This has solutions $\theta=\pi/6$ and $5\pi/6$; $\pi/6$ corresponds to
the intersection in the first quadrant that we need.  Note that no
solution of this equation corresponds to the intersection point at the
origin, but fortunately that one is obvious. The cardioid goes through
the origin when $\theta=-\pi/2$; the circle goes through the origin at
multiples of $\pi$, starting with $0$.

Now the larger region has area
$$
  {1\over2}\int_{-\pi/2}^{\pi/6} (1+\sin\theta)^2\;d\theta=
  {\pi\over2}-{9\over16}\sqrt{3}
$$
and the smaller has area
$$
  {1\over2}\int_{0}^{\pi/6} (3\sin\theta)^2\;d\theta=
  {3\pi\over8} - {9\over16}\sqrt{3}
$$
so the area we seek is $\pi/8$.
\endexample


\figure
\texonly
\hbox to \hsize{\hfill\tikzpicture[domain=-2:2,x=9mm,y=9mm]
\draw (-1.6,0) -- (1.6,0) ;
\draw (0,-1) -- (0,3.1) ;
\gpad
\draw[color=black] plot[parametric,id=\the\gpnum,domain=0:2*pi,samples=50] 
function{(1+sin(t))*cos(t),(1+sin(t))*sin(t)};
\gpad
\draw[color=black] plot[parametric,id=\the\gpnum,domain=0:pi,samples=50] 
function{3*sin(t)*cos(t),3*sin(t)*sin(t)};
\gpad
\fill[opacity=0.5,fill=red!20] 
plot[parametric,id=\the\gpnum,domain=-pi/2:pi/6]
function{(1+sin(t))*cos(t),(1+sin(t))*sin(t)} node {\gpad}
plot[parametric,id=\the\gpnum,domain=0:pi/6] 
function{3*sin(pi/6-t)*cos(pi/6-t),3*sin(pi/6-t)*sin(pi/6-t)}; 
\endtikzpicture
\quad
\tikzpicture[domain=-2:2,x=9mm,y=9mm]
\draw (-1.6,0) -- (1.6,0) ;
\draw (0,-1) -- (0,3.1) ;
\gpad
\draw[color=black] plot[parametric,id=\the\gpnum,domain=0:2*pi,samples=50] 
function{(1+sin(t))*cos(t),(1+sin(t))*sin(t)};
\gpad
\draw[color=black] plot[parametric,id=\the\gpnum,domain=0:pi,samples=50] 
function{3*sin(t)*cos(t),3*sin(t)*sin(t)};
\gpad
\fill[opacity=0.5,fill=red!20] 
plot[parametric,id=\the\gpnum,domain=-pi/2:pi/6]
function{(1+sin(t))*cos(t),(1+sin(t))*sin(t)}; 
\endtikzpicture
\quad
\tikzpicture[domain=-2:2,x=9mm,y=9mm]
\draw (-1.6,0) -- (1.6,0) ;
\draw (0,-1) -- (0,3.1) ;
\gpad
\draw[color=black] plot[parametric,id=\the\gpnum,domain=0:2*pi,samples=50] 
function{(1+sin(t))*cos(t),(1+sin(t))*sin(t)};
\gpad
\draw[color=black] plot[parametric,id=\the\gpnum,domain=0:pi,samples=50] 
function{3*sin(t)*cos(t),3*sin(t)*sin(t)};
\gpad
\fill[opacity=0.5,fill=red!20] 
plot[parametric,id=\the\gpnum,domain=0:pi/6] 
function{3*sin(t)*cos(t),3*sin(t)*sin(t)}; 
\endtikzpicture
\hfill}
\endtexonly
\figrdef{fig:harder area between polar curves}
\htmlfigure{Polar-Parametric-area_between_circle_cardioid.html}
\begincaption
An area between curves.
\endcaption
\endfigure

\exercises
% Mostly Keisler

\noindent Find the area enclosed by the curve.

\twocol

\exercise $\ds r=\sqrt{\sin\theta}$
\answer $1$
\endanswer
\endexercise

\exercise $\ds r=2+\cos\theta$
\answer $9\pi/2$
\endanswer
\endexercise

\exercise $\ds r=\sec\theta, \pi/6\le\theta\le\pi/3$
\answer $\ds \sqrt3/3$
\endanswer
\endexercise

\exercise $\ds r=\cos\theta, 0\le\theta\le\pi/3$
\answer $\ds \pi/12+\sqrt3/16$
\endanswer
\endexercise

\exercise $\ds r=2a\cos\theta, a>0$
\answer $\ds \pi a^2/4$
\endanswer
\endexercise

\exercise $\ds r=4+3\sin\theta$
\answer $41\pi/2$
\endanswer

\endtwocol
\endexercise

\exercise Find the area inside the loop formed by
$\ds r=\tan(\theta/2)$.
\answer $2-\pi/2$
\endanswer
\endexercise

\exercise Find the area inside one loop of $\ds r=\cos(3\theta)$.
\answer $\pi/12$
\endanswer
\endexercise

\exercise Find the area inside one loop of $\ds r=\sin^2\theta$.
\answer $3\pi/16$
\endanswer
\endexercise

\exercise Find the area inside the small loop of $\ds r=(1/2)+\cos\theta$.
\answer $\ds \pi/4-3\sqrt3/8$
\endanswer
\endexercise

\exercise Find the area inside $\ds r=(1/2)+\cos\theta$, including the
area inside the small loop.
\answer $\ds \pi/2+3\sqrt3/8$
\endanswer
\endexercise

\exercise Find the area inside one loop of $\ds r^2=\cos(2\theta)$.
\answer $1$
\endanswer
\endexercise

\exercise Find the area enclosed by $r=\tan\theta$ and 
$\ds r={\csc\theta\over\sqrt2}$.
\answer $3/2-\pi/4$
\endanswer
\endexercise

\exercise Find the area inside $r=2\cos\theta$ and outside
$r=1$.
\answer $\ds \pi/3+\sqrt3/2$
\endanswer
\endexercise

\exercise Find the area inside $r=2\sin\theta$ and above
the line $r=(3/2)\csc\theta$.
\answer $\ds \pi/3-\sqrt3/4$
\endanswer
\endexercise

\exercise Find the area inside $r=\theta$, $0\le\theta\le2\pi$.
\answer $\ds 4\pi^3/3$
\endanswer
\endexercise

\exercise Find the area inside $\ds r=\sqrt{\theta}$, $0\le\theta\le2\pi$.
\answer $\ds \pi^2$
\endanswer
\endexercise

\exercise Find the area inside both $\ds r=\sqrt3\cos\theta$ and
$r=\sin\theta$.
\answer $\ds 5\pi/24-\sqrt3/4$
\endanswer
\endexercise

\exercise Find the area inside both $r=1-\cos\theta$
and $r=\cos\theta$.
\answer $\ds 7\pi/12-\sqrt3$
\endanswer
\endexercise

\exercise The center of a circle of radius 1 is on the 
circumference of a circle of radius 2. Find the area 
of the region inside both circles.
\answer $\ds 4\pi-\sqrt{15}/2-7\arccos(1/4)$
\endanswer
\endexercise

\exercise Find the shaded area in figure~\xrefn{fig:area inside spiral}. 
The curve is $r=\theta$, $0\le\theta\le3\pi$.
\answer $\ds 3\pi^3$
\endanswer

\figure
\texonly
\hbox to \hsize{\hfill
\tikzpicture[domain=-2:2,x=4mm,y=4mm]
\draw (-10,0) -- (7,0) ;
\draw (0,-5.2) -- (0,8.5) ;
\gpad
\draw[color=black] plot[parametric,id=\the\gpnum,domain=0:3*pi,samples=100] 
function{(t)*cos(t),(t)*sin(t)};
\gpad
\fill[opacity=0.5,fill=red!20] (0,0) -- (2*pi,0)
plot[parametric,id=\the\gpnum,domain=2*pi:3*pi]
function{(t)*cos(t),(t)*sin(t)} node {\gpad} -- (-pi,0)
plot[parametric,id=\the\gpnum,domain=0:pi] 
function{(pi-t)*cos(pi-t),(pi-t)*sin(pi-t)}; 
\endtikzpicture
\hfill}
\endtexonly
\figrdef{fig:area inside spiral}
\htmlfigure{Polar-Parametric-spiral_area.html}
\begincaption
An area bounded by the spiral of Archimedes.
\endcaption
\endfigure

\endexercise

\endexercises
