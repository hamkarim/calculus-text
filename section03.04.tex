\section{The Quotient Rule}{}{}
\nobreak
\index{quotient rule}
What is the derivative of $\ds (x^2+1)/(x^3-3x)$? More generally, we'd
like to have a formula to compute the derivative of $f(x)/g(x)$ if we
already know $f'(x)$ and $g'(x)$. Instead of attacking this problem
head-on, let's notice that we've already done part of the problem:
$f(x)/g(x)= f(x)\cdot(1/g(x))$, that is, this is ``really'' a product,
and we can compute the derivative if we know $f'(x)$ and
$(1/g(x))'$. So really the only new bit of information we need is
$(1/g(x))'$ in terms of $g'(x)$. As with the product rule, let's set
this up and see how far we can get:
$$
\eqalign{
{d\over dx}{1\over g(x)}&=\lim_{\Delta x\to0} 
{{1\over g(x+\Delta x)}-{1\over g(x)}\over\Delta x}\cr
&=\lim_{\Delta x\to0} {{g(x)-g(x+\Delta x)\over g(x+\Delta x)g(x)}\over\Delta x}\cr
&=\lim_{\Delta x\to0} {g(x)-g(x+\Delta x)\over g(x+\Delta x)g(x)\Delta x}\cr
&=\lim_{\Delta x\to0} -{g(x+\Delta x)-g(x)\over \Delta x}
 {1\over g(x+\Delta x)g(x)}\cr
&=-{g'(x)\over g(x)^2}\cr
}$$
Now we can put this together with the product rule:
$${d\over dx}{f(x)\over g(x)}=f(x){-g'(x)\over g(x)^2}+f'(x){1\over
  g(x)}={-f(x)g'(x)+f'(x)g(x)\over g(x)^2}=
  {f'(x)g(x)-f(x)g'(x)\over g(x)^2}.
$$

\example
Compute the derivative of $\ds (x^2+1)/(x^3-3x)$.
$${d\over dx}{x^2+1\over
  x^3-3x}={2x(x^3-3x)-(x^2+1)(3x^2-3)\over(x^3-3x)^2}=
  {-x^4-6x^2+3\over (x^3-3x)^2}.
$$
\vskip-10pt
\endexample

It is often possible to calculate derivatives in more than one way, as
we have already seen. Since every quotient can be written as a
product, it is always possible to use the product rule to compute the
derivative, though it is not always simpler.

\example
Find the derivative of $\ds \sqrt{625-x^2}/\sqrt{x}$ in two ways: using the
quotient rule, and using the product rule.

Quotient rule:
$${d\over dx}{\sqrt{625-x^2}\over\sqrt{x}} = 
{\sqrt{x}(-x/\sqrt{625-x^2})-\sqrt{625-x^2}\cdot 1/(2\sqrt{x})\over
x}.$$
Note that we have used $\ds \sqrt{x}=x^{1/2}$ to compute the derivative of
$\ds \sqrt{x}$ by the power rule.

Product rule:
$${d\over dx}\sqrt{625-x^2} x^{-1/2} = 
\sqrt{625-x^2} {-1\over 2}x^{-3/2}+{-x\over \sqrt{625-x^2}}x^{-1/2}.
$$

With a bit of algebra, both of these simplify to
$$-{x^2+625\over 2\sqrt{625-x^2}x^{3/2}}.$$

\endexample

Occasionally you will need to compute the derivative of a quotient
with a constant numerator, like $\ds 10/x^2$. Of course you can use the
quotient rule, but it is usually not the easiest method. If we do use
it here, we get 
$${d\over dx}{10\over x^2}={x^2\cdot 0-10\cdot 2x\over x^4}=
{-20\over x^3},$$
since the derivative of 10 is 0. But it is simpler to do this:
$${d\over dx}{10\over x^2}={d\over dx}10x^{-2}=-20x^{-3}.$$
Admittedly, $\ds x^2$ is a particularly simple denominator, but we will
see that a similar calculation is usually possible. Another approach
is to remember that
$${d\over dx}{1\over g(x)}={-g'(x)\over g(x)^2},$$
but this requires extra memorization. Using this formula,
$${d\over dx}{10\over x^2}=10{-2x\over x^4}.$$
Note that we first use linearity of the derivative to pull the 10 out
in front.

% Hack
% \vfill\eject

\exercises

Find the derivatives of the functions in 1--4
using the quotient rule.

\twocol

\exercise $\ds {x^3\over x^3-5x+10}$
\answer $\ds {3x^2\over x^3-5x+10}-{x^3(3x^2-5)\over (x^3-5x+10)^2}$
\endanswer
\endexercise

\exercise $\ds {x^2+5x-3\over x^5-6x^3+3x^2-7x+1}$
\answer $\ds {2x+5\over x^5-6x^3+3x^2-7x+1}-
{(x^2+5x-3)(5x^4-18x^2+6x-7)\over(x^5-6x^3+3x^2-7x+1)^2}$
\endanswer

\ssk
\endexercise

\exercise $\ds {\sqrt{x}\over\sqrt{625-x^2}}$
\answer $\ds {1\over 2\sqrt{x}\sqrt{625-x^2}}+{x^{3/2}\over(625-x^2)^{3/2}}$
\endanswer
\endexercise

\exercise $\ds {\sqrt{625-x^2}\over x^{20}}$
\answer $\ds {-1\over x^{19}\sqrt{625-x^2}}-{20\sqrt{625-x^2}\over x^{21}}$
\endanswer

\endtwocol
\bsk
\endexercise

\exercise Find an equation for the tangent line to $\ds f(x) = (x^2 -
4)/(5-x)$ at $x= 3$.  
\answer $\ds y=17x/4-41/4$ 
\endanswer
\endexercise

\exercise  Find an equation for the tangent line to 
$\ds f(x) = (x-2)/(x^3 + 4x - 1)$ at $x=1$.
\answer $y=11x/16-15/16$
\endanswer
\endexercise

\exercise Let $P$ be a polynomial of degree $n$ and let $Q$ be a
polynomial of degree $m$ (with $Q$ not the zero polynomial). 
Using sigma notation we can write
$$P=\sum _{k=0 } ^n a_k x^k,\qquad
Q=\sum_{k=0}^m b_k x^k.
$$
Use sigma notation to write the derivative of the 
{\dfont rational function\index{rational function}\/}
$P/Q$.
%% \answer $\left(\sum_{k=0}^m b_k x^k\sum _{k=1}^n ka_k x^{k-1}-
%% \sum _{k=0 } ^n a_k x^k\sum_{k=1}^m kb_k x^k\right)/
%% \left(\sum_{k=0}^m b_k x^k\right)^2$
%% \endanswer
\endexercise

\exercise The curve $\ds y=1/(1+x^2)$ is an example of a class of
curves each of which is called a {\dfont witch of
Agnesi\index{witch of Agnesi}}. 
Sketch the curve and find the tangent line to the curve at
$x= 5$. (The word {\em witch\/} here is a mistranslation of the
original Italian, as described at
\texonly
$$\hbox{\url{http://mathworld.wolfram.com/WitchofAgnesi.html} 
\vb|http://mathworld.wolfram.com/WitchofAgnesi.html|\endurl}$$
and 
$$\eqalign{
\hbox{\url{http://instructional1.calstatela.edu/sgray/Agnesi/WitchHistory/Historynamewitch.html} 
\vb|http://|\endurl}%
&\!\!\hbox{\url{http://instructional1.calstatela.edu/sgray/Agnesi/WitchHistory/Historynamewitch.html} 
\vb|instructional1.calstatela.edu/sgray/Agnesi/|\endurl}\cr
&\hbox{\url{http://instructional1.calstatela.edu/sgray/Agnesi/WitchHistory/Historynamewitch.html} 
\vb|WitchHistory/Historynamewitch.html|\endurl.)}\cr
}$$\endtexonly
\htmlonly
<a href="http://mathworld.wolfram.com/WitchofAgnesi.html">http://mathworld.wolfram.com/WitchofAgnesi.html</a> and
<a href="http://instructional1.calstatela.edu/sgray/Agnesi/WitchHistory/Historynamewitch.html">http://instructional1.calstatela.edu/sgray/Agnesi/WitchHistory/Historynamewitch.html</a>.
\endhtmlonly
\answer $y=19/169-5x/338$
\endanswer
%% \footnote{Due to a mistranslation of the Italian word
%%   \emph{versiera} which actually refers to a rope that turns the
%%   sail.}.  
\endexercise

\exercise If $f'(4) = 5$, $g'(4) = 12$, $(fg)(4)= f(4)g(4)=2$, and $g(4) = 6$,
compute $f(4)$ and $\ds{d\over dx}{f\over g}$ at 4.
\answer $13/18$
\endanswer
\endexercise

\endexercises

