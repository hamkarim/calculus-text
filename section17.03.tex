\section{First Order Linear Equations}{}{}
%\secrdef{sec:first order homogeneous linear}
\nobreak
As you might guess, a first order linear differential equation has the form 
$\ds \dot y + p(t)y = f(t)$. Not only is this closely related in form
to the first order homogeneous linear equation, we can use what we
know about solving homogeneous equations to solve the general linear
equation. 

Suppose that $y_1(t)$ and $y_2(t)$ are solutions to 
$\ds \dot y + p(t)y = f(t)$. Let $\ds g(t)=y_1-y_2$. Then
$$\eqalign{
 g'(t)+p(t)g(t)&=y_1'-y_2'+p(t)(y_1-y_2)\cr
&=(y_1'+p(t)y_1)-(y_2'+p(t)y_2)\cr
&=f(t)-f(t)=0.\cr}
$$
In other words, $\ds g(t)=y_1-y_2$ is a solution to the homogeneous
equation $\ds \dot y + p(t)y = 0$. Turning this around, any solution
to the linear equation $\ds \dot y + p(t)y = f(t)$, call it $y_1$, can
be written as $y_2+g(t)$, for some particular $y_2$ and some solution
$g(t)$ of the homogeneous equation $\ds \dot y + p(t)y = 0$. Since we
already know how to find all solutions of the homogeneous equation,
finding just one solution to the equation $\ds \dot y + p(t)y = f(t)$
will give us all of them.

How might we find that one particular solution to $\ds \dot y + p(t)y
= f(t)$? Again, it turns out that what we already know helps. We know
that the general solution to the homogeneous equation
$\ds \dot y + p(t)y = 0$ looks like $\ds Ae^{P(t)}$. We now make an
inspired guess: consider the function $\ds v(t)e^{P(t)}$, in which we
have replaced the constant parameter $A$ with the function
$v(t)$. This technique is called 
{\dfont variation of parameters\index{variation of parameters}}.
For
convenience write this as $s(t)=v(t)h(t)$ where $\ds h(t)=e^{P(t)}$ 
is a solution to the
homogeneous equation. Now let's compute a bit with $s(t)$:
$$\eqalign{
s'(t)+p(t)s(t)&=v(t)h'(t)+v'(t)h(t)+p(t)v(t)h(t)\cr
&=v(t)(h'(t)+p(t)h(t)) + v'(t)h(t)\cr
&=v'(t)h(t).\cr}
$$
The last equality is true because $\ds h'(t)+p(t)h(t)=0$, since $h(t)$
is a solution to the homogeneous equation. We are hoping to find a
function $s(t)$ so that $\ds s'(t)+p(t)s(t)=f(t)$; we will have such a
function if we can arrange to have $\ds v'(t)h(t)=f(t)$, that is,
$\ds v'(t)=f(t)/h(t)$. But this is as easy (or hard) as finding an
anti-derivative of $\ds f(t)/h(t)$. Putting this all together, the
general solution to $\ds \dot y + p(t)y = f(t)$ is
$$v(t)h(t)+Ae^{P(t)} = v(t)e^{P(t)}+Ae^{P(t)}.$$

\example Find the solution of the initial value problem
$\ds \dot y+3y/t=t^2$, $y(1)=1/2$. First we find the general solution;
since we are interested in a solution with a given condition at $t=1$,
we may assume $t>0$.
We start by solving the homogeneous equation as usual; call the
solution $g$:
$$g=Ae^{-\int (3/t)\,dt}=Ae^{-3\ln t}=At^{-3}.$$
Then as in the discussion, $\ds h(t)=t^{-3}$ and
$\ds v'(t)=t^2/t^{-3}=t^5$, so $\ds v(t)=t^6/6$. We know that
every solution to the equation looks like
$$v(t)t^{-3}+At^{-3}={t^6\over6}t^{-3}+At^{-3}={t^3\over6}+At^{-3}.$$
Finally we substitute to find $A$:
$$\eqalign{
{1\over 2}&={(1)^3\over6}+A(1)^{-3}={1\over6}+A\cr
A&={1\over 2}-{1\over6}={1\over3}.\cr}
$$
The solution is then
$$y={t^3\over6}+{1\over3}t^{-3}.$$
\endexample

Here is an alternate method for finding a particular solution to the
differential equation, using an
{\dfont integrating factor\/}. In the differential equation $\ds \dot
y+p(t)y=f(t)$, we note that if we multiply through by a function
$I(t)$ to get $\ds I(t)\dot y+I(t)p(t)y=I(t)f(t)$, the left hand side
looks like it could be a derivative computed by the product rule:
$${d\over dt}(I(t)y)=I(t)\dot y+I'(t)y.$$
Now if we could choose $I(t)$ so that $I'(t)=I(t)p(t)$, this would be
exactly the left hand side of the differential equation. But this is
just a first order homogeneous linear equation, and we know a solution
is $\ds I(t)=e^{Q(t)}$, where $\ds Q(t)=\int p\,dt$; note that 
$Q(t)=-P(t)$, where $P(t)$ appears in the variation of parameters
method and $P'(t)=-p$. Now the modified differential equation is 
$$
\eqalign{
e^{-P(t)}\dot y+e^{-P(t)}p(t)y&=e^{-P(t)}f(t)\cr
{d\over dt}(e^{-P(t)}y)&=e^{-P(t)}f(t).\cr
}$$
Integrating both sides gives
$$
\eqalign{
e^{-P(t)}y&=\int e^{-P(t)}f(t)\,dt\cr
y&=e^{P(t)}\int e^{-P(t)}f(t)\,dt.\cr
}$$
If you look carefully, you will see that this is exactly the same
solution we found by variation of parameters, because
$\ds e^{-P(t)}f(t)=f(t)/h(t)$.

Some people find it easier to remember how to use the integrating
factor method than variation of parameters. Since ultimately they
require the same calculation, you should use whichever of the two you
find easier to recall. Using this method, the solution of the previous
example would look just a bit different: Starting with
$\ds \dot y+3y/t=t^2$, we recall that the integrating factor is
$\ds e^{\int 3/t}=e^{3\ln t}=t^3$. Then we multiply through by the
integrating factor and solve:
$$\eqalign{
t^3\dot y+t^3 3y/t&=t^3t^2\cr
t^3\dot y+t^2 3y&=t^5\cr
{d\over dt}(t^3 y)&=t^5\cr
t^3 y&=t^6/6\cr
y&=t^3/6.\cr}
$$
This is the same answer, of course, and the problem is then finished
just as before.

\exercises

In problems 1--10, find the general solution of the equation.

\exercise $\ds\dot y +4y=8$
\answer $\ds y=Ae^{-4t}+2$
\endanswer
\endexercise

\exercise $\ds\dot y-2y=6$
\answer $\ds y=Ae^{2t}-3$
\endanswer
\endexercise

\exercise $\ds\dot y +ty=5t$
\answer $\ds y=Ae^{-(1/2)t^2}+5$
\endanswer
\endexercise

\exercise $\ds\dot y+e^ty=-2e^t$
\answer $\ds y=Ae^{-e^t}-2$
\endanswer
\endexercise

\exercise $\ds\dot y-y=t^2$
\answer $\ds y=Ae^{t}-t^2-2t-2$
\endanswer
\endexercise

\exercise $\ds 2\dot y +y=t$
\answer $\ds y=Ae^{-t/2}+t-2$
\endanswer
\endexercise

\exercise $\ds t\dot y -2y=1/t$, $t>0$
\answer $\ds y=At^2-{1\over3t}$
\endanswer
\endexercise

\exercise $\ds t\dot y+y=\sqrt{t}$, $t>0$
\answer $\ds y={c\over t}+{2\over3}\sqrt t$
\endanswer
\endexercise

\exercise $\ds\dot y\cos t+y\sin t=1$, $-\pi/2<t<\pi/2$
\answer $\ds y= A\cos t+\sin t$
\endanswer
\endexercise

\exercise $\ds\dot y + y\sec t=\tan t$, $-\pi/2<t<\pi/2$
\answer $\ds y= {A\over\sec t+\tan t}+1-{t\over\sec t+\tan t}$
\endanswer
\endexercise

\endexercises
