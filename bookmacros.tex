%%%%%%%%%%%%%%%%%%%%%%%%%%%%%%%%%%%%%%%%%%%%%%%%%%%%%%%%%%%%%%%%%%%%
%
% Plain TeX Macros for the Addison-Wesley Upper Level Math Series
%
% David Guichard, June 1996
%
%%%%%%%%%%%%%%%%%%%%%%%%%%%%%%%%%%%%%%%%%%%%%%%%%%%%%%%%%%%%%%%%%%%%%
%
% You can use your text editor to search for these sections; for
% example, search for ``% Index'' to find the indexing macros.
%
% Sections:
%
% Choices
% Other macro packages
% Fonts
% Important dimensions
% Footnotes
% Footline and headline
% Table of contents
% Index
% Cross references
% Title page
% Chapters, sections, subsections
% Exercises 
% Suggested references 
% Extracts
% Lists 
% Output routine
% Theorem-like objects
% Figures
% Miscellaneous stuff
%
%%%%%%%%%%%%%%%%%%%%%%%%%%%%%%%%%%%%%%%%%%%%%%%%%%%%%%%%%%%%%%%%%%%%%
%
% Choices, choices. Note that the cameraready, times and
% printblankpages flags are set by making appropriate definitions in
% the main document, before this file is loaded. See the user's manual
% for details.
%
\def\FALSE{false}
\def\TRUE{true}
%
\tracingstats1
%
\newif\ifamsfonts
\amsfontstrue            % true if the AMS extra symbol fonts are available
%
% Should the table of contents include subsections?
%
\newif\ifsubsectionsintoc
\ifx\subsectionsintoc\FALSE\subsectionsintocfalse\else
\ifx\subsectionsintoc\TRUE\subsectionsintoctrue\else
\subsectionsintocfalse\fi\fi
%
\newif\ifprintblankpages
\ifx\printblankpages\FALSE\printblankpagesfalse\else
\ifx\printblankpages\TRUE\printblankpagestrue\else\printblankpagestrue\fi\fi
%
% \ifcamerareadytrue makes the pages the correct book size and puts
% registration marks at the corners. \ifcamerareadyfalse makes the
% pages full size (8.5 by 11), magnifies the document, and increases the
% baselineskip for easier reading and correcting. True is the default;
% it may be changed by including the command \def\cameraready{false}
% in the main document *before* the ``\input bookmacros'' line.
%
\newif\ifkindle
\ifx\kindle\FALSE\kindlefalse\else
\ifx\kindle\TRUE\kindletrue\else\kindlefalse\fi\fi
%
\newif\ifcameraready
\ifx\cameraready\FALSE\camerareadyfalse\else
\ifx\cameraready\TRUE\camerareadytrue\else\camerareadyfalse\fi\fi
%
\newif\iftimes
\ifx\usetimesfonts\FALSE\timesfalse\else
\ifx\usetimesfonts\TRUE\timestrue\else\timesfalse\fi\fi
%
\newif\ifthesis
\ifx\thesis\FALSE\thesisfalse\else
\ifx\thesis\TRUE\thesistrue\else\thesisfalse\fi\fi
%
\ifcameraready\else
\ifkindle
\magnification\magstep2\else
\magnification\magstep1
\fi\fi
%
% The next if controls whether an index has large capital letters to
% mark each section of the index. To use, do something like this:
%
%   \beginletteredindex
%   \input indexfile
%   \endindex
%
% You need to set \letteredindexfalse if there is an index after this
% one and you don't want it lettered.
%
\newif\ifletteredindex\letteredindexfalse
%
% Whether to number prefatory material with roman numerals.
\newif\ifromanintro
\ifx\romanintro\FALSE\romanintrofalse\else
\ifx\romanintro\TRUE\romanintrotrue\else\romanintrotrue\fi\fi
%
%\overfullrule0pt
%
%%%%%%%%%%%%%%%%%%%%%%%%%%%%%%%%%%%%%%%%%%%%%%%%%%%%%%%%%%%%%%%%%%%%%
%
% Other macro packages
%
\input idxmacdrg     % indexing macros--needs my modified version
\input tindex        % for a second index
\input eplaindrg     % This works with eplain 1.9 as modified by me.
%
%%%%%%%%%%%%%%%%%%%%%%%%%%%%%%%%%%%%%%%%%%%%%%%%%%%%%%%%%%%%%%%%%%%%%
\makeatletter
\def\\{\let\stoken= } \\
\long\def\unexpandedwrite#1#2{\def\finwrite{\immediate\write#1}%
  {\aftergroup\finwrite\aftergroup{\s@nitize#2\endsanity}}}
\def\s@nitize{\futurelet\next\sanswitch}
\def\sanswitch{\let\n@xt\endsanity \ifx\next\endsanity
  \else\ifcat\noexpand\next\stoken\aftergroup\space\let\n@xt=\eat
   \else\ifcat\noexpand\next\bgroup\aftergroup{\let\n@xt=\eat
    \else\ifcat\noexpand\next\egroup\aftergroup}\let\n@xt=\eat
     \else\let\n@xt=\copytok\fi\fi\fi\fi \n@xt}
\def\eat{\afterassignment\s@nitize \let\next= }
\long\def\copytok#1{\ifcat\noexpand#1\relax\aftergroup\noexpand\fi
  \ifcat\noexpand#1\noexpand~\aftergroup\noexpand\fi
  \aftergroup#1\s@nitize}
\def\endsanity\endsanity{}
\makeatother
%
% Fonts
%
\ifamsfonts
\input amssym.def    % If you have these files and the corresponding 
\input amssym        % font files for msam10 and msbm10, make \amsfontstrue
\fi                  % above.
%
\iftimes
\input mtplain
\input calligraphic
\input eightpointps
\input ninepointps
\input ninehalfpointps
\input tenpointps
\input tenhalfpointps
\input elevenpointps
\input twelvepointps
%
\else  % cm fonts
%
\input sevenpoint
\input eightpoint
\input ninepoint
\input tenpoint
\input elevenpoint
\input twelvepoint
\fi
%
\iftimes
\font\booktitlefont ptmb8r at 29truept
\let\booksubtitlefont\elevenit
\font\chaptitlefont ptmb8r at 24truept
\font\chapnumberfont ptmb8r at 42truept
%\font\thmfont ptmb8r at 10pt
%\font\secheaderfont ptmb8r at 8pt
\font\pagenofont ptmb8r at 11pt
\font\prooffont ptmbi8r at 10.5pt
\font\remarksfont ptmri8r at 10.5pt
\font\ctscnfont ptmb8r at 18pt
\let\authorfont\ctscnfont
%\font\scfont=pcrr8r
\font\scfont=ptmrsc at 12truept
\font\bscfont=ptmbsc at 14.4truept
\font\secfontmathrm=ptmb8r at 14.4truept
\font\secfontmathit=ptmbi8r at 14.4truept
\font\idxfont ptmbi8r at 12truept
\font\awfont phvb8r at 12truept
\font\exerciseheadfont ptmbi8r at 11pt
\let\institutionfont\exerciseheadfont
\font\referenceheadfont ptmbi8r at 10pt
\doublehyphendemerits=1000000
\else
\font\booktitlefont cmbx10 at 29.86truept
\font\booksubtitlefont cmti10
\font\chaptitlefont cmbx10 at 24.88truept
\font\chapnumberfont cmbx10 at 44.58truept
\font\pagenofont cmbx10
\font\prooffont cmbxti10
\font\remarksfont cmti10
\font\ctscnfont cmbx10 at 17.28pt
\let\authorfont\ctscnfont
\font\scfont=cmcsc10
\font\bscfont=gmbcsc10 scaled \magstep2
%\font\bscfont=ptmbsc at 14.4truept
\font\secfontmathrm=cmbx10 scaled \magstep2
\font\secfontmathit=cmbxti10 scaled \magstep2
\font\idxfont cmbxti10 at 12truept
\font\awfont cmssbx10
\font\exerciseheadfont cmbxti10 scaled \magstephalf
\let\institutionfont\exerciseheadfont
\font\referenceheadfont cmbxti10 
\fi
%
\newfam\smcap \def\sc{\fam\smcap\scfont} \textfont\smcap=\scfont
%
% secfont needs fleshing out
\long\def\setupamsfonts{\csname newif\endcsname\ifamsfonts\amsfontstrue}
\ifx\ifamsfonts\undefined\let\next\setupamsfonts
\else\let\next\relax\fi\next
\long\def\setupbbbold{\csname newfam\endcsname\bbbfam}
\ifamsfonts\ifx\bbbfam\undefined\let\next\setupbbbold\else
\let\next\relax\fi\else\let\next\relax\fi\next
%% \ifamsfonts
%% \font\tenbbbold=msbm10
%% \font\tensubbbbold=msbm7
%% \fi
%% \def\secfont{\def\rm{\fam0\bscfont}%
%%   \textfont0=\secfontmathrm \scriptfont0=\sevenrm \scriptscriptfont0=\fiverm
%%   \textfont1=\secfontmathit \scriptfont1=\seveni \scriptscriptfont1=\fivei
%%   \textfont2=\tensy \scriptfont2=\sevensy \scriptscriptfont2=\fivesy
%%   \textfont3=\tenex \scriptfont3=\sevenex \scriptscriptfont3=\sevenex
%%   \textfont\itfam=\tenit  \def\it{\fam\itfam\tenit}%
%%   \textfont\slfam=\tensl  \def\sl{\fam\slfam\tensl}%
%%   \textfont\ttfam=\tentt  \def\tt{\fam\ttfam\tentt}%
%%   \textfont\bffam=\tenbf  \scriptfont\bffam=\sevenbf
%%   \scriptscriptfont\bffam=\fivebf  \def\bf{\fam\bffam\tenbf}%
%%   \ifamsfonts
%%   \textfont\bbbfam\tenbbbold \scriptfont\bbbfam\tensubbbbold
%%   \def\bbbold{\fam\bbbfam\tenbbbold}\fi%
%%   \normalbaselineskip=12pt%
%%   \setbox\strutbox=\hbox{\vrule height8.5pt depth3.5pt width0pt}%
%%   \normalbaselines\rm}

\ifamsfonts
\font\fourteenbbbold=msbm10 scaled \magstep2
\font\fourteensubbbbold=msbm10
\fi
\def\secfont{\def\rm{\fam0\bscfont}%
  \textfont0=\secfontmathrm \scriptfont0=\tenrm \scriptscriptfont0=\sevenrm
  \textfont1=\secfontmathit \scriptfont1=\tenit \scriptscriptfont1=\sevenit
  \textfont2=\tensy \scriptfont2=\tensy \scriptscriptfont2=\sevensy
  \textfont3=\tenex \scriptfont3=\tenex \scriptscriptfont3=\sevenex
%%   \textfont\itfam=\tenit  \def\it{\fam\itfam\tenit}%
%%   \textfont\slfam=\tensl  \def\sl{\fam\slfam\tensl}%
%%   \textfont\ttfam=\tentt  \def\tt{\fam\ttfam\tentt}%
%%   \textfont\bffam=\fourteenbf  \scriptfont\bffam=\tenbf
%%   \scriptscriptfont\bffam=\sevenbf  \def\bf{\fam\bffam\fourteenbf}%
  \ifamsfonts
  \textfont\bbbfam\fourteenbbbold \scriptfont\bbbfam\fourteensubbbbold
  \def\bbbold{\fam\bbbfam\fourteenbbbold}\fi%
  \normalbaselineskip=18pt%
  \setbox\strutbox=\hbox{\vrule height12.5pt depth5.5pt width0pt}%
  \normalbaselines\rm}

%\def\secfont{\bscfont}
\def\subsecfont{\elevenpoint\bf}
%
\iftimes
\def\secheaderfont{\eightpoint\bf}
\def\thmfont{\tenpoint\bf}             % For the words "Theorem", "Lemma", etc
\else
\def\secheaderfont{\eightpoint\bf}
\def\thmfont{\tenpoint\bf}             % For the words "Theorem", "Lemma", etc
\fi
%
% blackboard bold
%
%\newfam\bbbfam
\ifamsfonts
\font\tenbbbold=msbm10\font\tensubbbbold=msbm7
\textfont\bbbfam\tenbbbold \scriptfont\bbbfam\tensubbbbold
\def\bbbold{\fam\bbbfam\tenbbbold}
\def\bbb{\bbbold}
\else
\def\bbb{\bf}
\fi
% bold math
\newtoks\hexf
\def\sethex#1{\ifcase#1 \hexf={0} \or \hexf={1} \or \hexf={2} \or \hexf={3}
     \or \hexf={4} \or \tsth={5} \or \hexf={6} \or \hexf={7} \or \hexf={8} 
     \or \hexf={9} \or \hexf={A} \or \hexf={B} \or \hexf={C} \or \hexf={D} 
     \or \hexf={E} \or \hexf={F}\fi}
\def\domathdef#1{\def\dodef##1{#1}\expandafter\dodef\the\hexf}
\newfam\boldmath
% \def\bmit{\fam\boldmath}
% \textfont\boldmath\bmath
% \scriptfont\boldmath\bmathsub
\font\bgreek cmmib10
\font\bgreeksub cmmib10 at 7pt
\newfam\boldgreek
\sethex\boldgreek
\domathdef{\mathchardef\bdelta="0#10E}
\domathdef{\mathchardef\bsigma="0#11B}
\textfont\boldgreek\bgreek
\scriptfont\boldgreek\bgreeksub
% 
\def\dfont{\bf}
\def\em{\it}           % for emphasis
\def\thmstmtfont{}                    % For the statement of a theorem object
%
%%%%%%%%%%%%%%%%%%%%%%%%%%%%%%%%%%%%%%%%%%%%%%%%%%%%%%%%%%%%%%%%%%%%%
%
% Important dimensions
%
% used for cameraready output
%
\newdimen\lm
\newdimen\oddlm\oddlm=0.8125truein
\newdimen\evenlm\evenlm=0.75truein
\newdimen\pageheight\pageheight9.25truein
\newdimen\pagewidth\pagewidth6.375truein
% 
% used for non-cameraready output
%
\newdimen\pageshift
\pageshift=0.2truein
%
\newskip\abovesectionskip\abovesectionskip 18pt plus6pt
\newskip\abovesubsectionskip\abovesubsectionskip 13pt plus3pt
%
\newskip\beforethmskip\beforethmskip 12pt plus1pt
\newskip\afterproofskip\afterproofskip=6pt
%
\newskip\beforeexerciseskip\beforeexerciseskip 12pt plus6pt
\newskip\afterexerciseskip\afterexerciseskip 12pt
%
\newskip\beforereferenceskip\beforereferenceskip 12pt
%
\newskip\extractindentamount\extractindentamount=\parindent
\newskip\aboveextractskip\aboveextractskip=6pt
\newskip\belowextractskip\belowextractskip=6pt
%
\newskip\beforeclaimlistskip\beforeclaimlistskip=6pt
\newskip\afterclaimlistskip\afterclaimlistskip=6pt
\newskip\claimlistseparationskip\claimlistseparationskip=3pt
%
\newskip\exerciselistindentamount\exerciselistindentamount=36pt
\newskip\exerciseseparationskip\exerciseseparationskip=3pt
\newskip\beforeexerciselistskip\beforeexerciselistskip=3pt
\newskip\afterexerciselistskip\afterexerciselistskip=3pt
\newskip\exerciselistseparationskip\exerciselistseparationskip=3pt
\newskip\afterexerciseremarkskip\afterexerciseremarkskip=6pt
\newskip\beforeexerciseremarkskip\beforeexerciseremarkskip=6pt
%
\newskip\listindentamount\listindentamount=38pt
\newskip\beforelistskip\beforelistskip=3pt
\newskip\afterlistskip\afterlistskip=3pt
\newskip\listseparationskip\listseparationskip=3pt
%
\newskip\basaltextbaselineskip
\newdimen\basalparindent
%
% bottomskip is the amount of space left at the bottom of a page.
% This should be 0pt plus the allowable stretch.
%
\newskip\bottomskip
%
\abovecolumnskip = 0pt
\belowcolumnskip = 0pt
\topskip10truept
%
\ifcameraready
\basaltextbaselineskip12pt plus 0.2pt
\basalparindent18truept
\bottomskip 0pt plus 6pt
\hsize29truepc
\vsize578truept
\abovedisplayskip6pt plus 1pt
\abovedisplayshortskip0pt plus 1pt
\belowdisplayskip6pt plus 1pt
\belowdisplayshortskip6pt plus 1pt
\parskip0pt
\makeatletter
\def\raggedbottom{\topskip 10\p@ plus0\p@ \r@ggedbottomtrue}
\def\normalbottom{\topskip 10\p@ \r@ggedbottomfalse} % undoes \raggedbottom
\makeatother
%
\else\ifthesis                  % thesis
\hsize=6truein
\advance\vsize by -0.35truein
\advance\hoffset by 0.5truein
\advance\voffset by 0.15truein
\basaltextbaselineskip12pt
\basalparindent\parindent
\bottomskip=0pt plus 1fil
\makeatletter
\def\raggedbottom{\topskip 30\p@ plus0\p@ \r@ggedbottomtrue}
\def\normalbottom{\topskip 30\p@ \r@ggedbottomfalse} % undoes \raggedbottom
\makeatother
\else
%\hsize5.1truein
%\vsize8.76truein
\basaltextbaselineskip14pt
\basalparindent\parindent
\bottomskip=0pt plus 1fil
\makeatletter
\def\raggedbottom{\topskip 10\p@ plus0\p@ \r@ggedbottomtrue}
\def\normalbottom{\topskip 10\p@ \r@ggedbottomfalse} % undoes \raggedbottom
\makeatother
\fi\fi
%
\baselineskip\basaltextbaselineskip
\parindent\basalparindent
%
% high penalties for widows and orphans
%
\postdisplaypenalty=1000
\widowpenalty=10000 \clubpenalty=10000
%
% Just in case these aren't set properly...the voice of experience.
%
\lefthyphenmin=2
\righthyphenmin=3
%
\newif\ifchapstart
\newif\ifchapopener
%
% The next four flags are set at the ends of appropriate items when
% extra space is left. If such items are immediately followed by items
% that leave extra space, we can use the flags to avoid leaving extra
% space twice. We should really have more than four of these, but
% since many items have the same before & after skip, we make do with
% these for simplicity, at the expense of some confusion.
%
\newif\ifproofflag
\newif\iflistflag
\newif\ifexerciseflag
\newif\ifexerciseremarkflag
%
\everypar{\global\proofflagfalse
          \global\listflagfalse
          \global\chapstartfalse
          \global\exerciseflagfalse
          \global\exerciseremarkflagfalse
          \ifdim\parindent=0pt\parindent18pt\fi
}
%
%%%%%%%%%%%%%%%%%%%%%%%%%%%%%%%%%%%%%%%%%%%%%%%%%%%%%%%%%%%%%%%%%%%%%
%
% Footnotes. Non-standard macros from eplain.
%
\footnoterulewidth5pc
%
\def\footnote#1{\numberedfootnote
{\eightpoint 
\abovedisplayskip2pt plus 0pt
\abovedisplayshortskip0pt plus 0pt
\belowdisplayskip2pt plus 0pt
\belowdisplayshortskip0pt plus 0pt
#1}}
%
\interfootnoteskip1pt
%
%%%%%%%%%%%%%%%%%%%%%%%%%%%%%%%%%%%%%%%%%%%%%%%%%%%%%%%%%%%%%%%%%%%%%
%
% Footline and headline
%
\newif\ifblankpage\blankpagefalse
%
% The footline contains the page number on the first page of a 
% chapter; otherwise it is blank.
%
\footline{\ifblankpage\global\blankpagefalse\else
   \ifchapopener \hfil\pagenofont\folio \else \hfil\fi\fi}
%
% Borrowed and modified from pictex.
%
\catcode`!=11
%\def\!empty{}
\def\!Nil{\!nil}
\def\!noop#1\!!#2#3{}
\def\dashnumber#1--{--#1}
\def\dopageref#1--#2/{ \xrefn{indexpage:#1}%
  \def\dashcheck{#2}\ifx\dashcheck\empty\else\dashnumber#2\fi}
\def\!myfor#1:=#2\do#3{%
  \edef\!fortemp{#2}%
  \ifx\!fortemp\empty
  \else
    \!myforloop#2, \!nil, \!nil\!!#1{#3}%
  \fi}
\def\!myforloop#1, #2\!!#3#4{%
  \def#3{#1}%
  \ifx #3\!Nil
    \let\!nextwhile=\!noop 
  \else
    #4\relax
    \let\!nextwhile=\!myforloop
  \fi
  \!nextwhile#2\!!#3{#4}}
\def\pagenowrapper#1{\def\commanext{}%
    \!myfor\pageref:=#1\do{\commanext\expandafter\dopageref\pageref--/\def\commanext{,}}}
\catcode`!=12

%
% The headline contains the page number on the outer corner of the
% page. Usually, it contains also the chapter title on even pages
% and the section title on odd pages. There are exceptions, and 
% things are complicated by the fact that the word ``chapter'' is not
% always an appropriate way to describe the current chapter-level 
% object. Also, we have to use TeX ``marks'' to get the section name
% right.
%
\def\makepagerdef#1.{\pagerdef{indexpage:#1}}
\headline{%\vbox to 20pt{\hbox to \hsize{%
   \edef\setpagerdef{\folio}%
   \expandafter\makepagerdef\setpagerdef.%
   \ifchapopener
     \hfil
   \else\ifblankpage
     \global\blankpagefalse
   \else\ifnum\chapnum<1
      \ifthesis \hfil{\eightpoint\bf \chaptertitle}\kern1truepc
         \pagenofont\folio
      \else\ifodd\pageno \hfil{\eightpoint\bf \chaptertitle}\kern1truepc
         \pagenofont\folio
      \else
         \pagenofont\folio\kern1truepc{\eightpoint\bf \chaptertitle}\hfil
      \fi\fi
   \else\ifx\chaptertitle\idxtitle
      \ifthesis 
        \hfil{\eightpoint\bf \chaptertitle}\kern1truepc\pagenofont\folio
      \else\ifodd\pageno \hfil{\eightpoint\bf \chaptertitle}\kern1truepc
        \pagenofont\folio
      \else
        \pagenofont\folio\kern1truepc{\eightpoint\bf \chaptertitle}\hfil
      \fi\fi
   \else\ifx\chapname\unletteredname
      \ifthesis 
        \hfil{\eightpoint\bf \chaptertitle}\kern1truepc\pagenofont\folio
      \else\ifodd\pageno \hfil{\eightpoint\bf \chaptertitle}\kern1truepc
        \pagenofont\folio
      \else
        \pagenofont\folio\kern1truepc{\eightpoint\bf \chaptertitle}\hfil
      \fi\fi
   \else
      \ifthesis 
        \hfil
        \ifnum\secnum=0
           {\eightpoint\bf \chapname \the\chapsymbol\enspace\chaptertitle}
        \else
           {\expandafter\secheader\botmark}
        \fi
        \kern1truepc\pagenofont\folio
      \else\ifodd\pageno 
        \hfil
        \ifnum\secnum=0
           {\eightpoint\bf \chapname \the\chapsymbol\enspace\chaptertitle}
        \else
           {\expandafter\secheader\botmark}
        \fi
        \kern1truepc\pagenofont\folio
      \else \pagenofont\folio\kern1truepc
            {\eightpoint\bf \chapname \the\chapsymbol\enspace
             \chaptertitle}\hfil
      \fi\fi
   \fi\fi\fi\fi\fi}%\vfill}}
%
\def\secheader#1#2{%
\ifnum#1=0
\eightpoint\bf \chapname \the\chapsymbol\enspace\chaptertitle
\else
\secheaderfont\the\chapsymbol.#1 \enspace#2
\fi}
%
%%%%%%%%%%%%%%%%%%%%%%%%%%%%%%%%%%%%%%%%%%%%%%%%%%%%%%%%%%%%%%%%%%%%%
%
% Table of contents. The heavy lifting is done
% by eplain.
%
\def\begincts{\vskip-26pt\bgroup\baselineskip14pt\parindent18pt}
\newbox\chapnumbox
\setbox\chapnumbox\hbox{\twelvepoint\bf 00.\enspace}
\def\endcts{\egroup}
%
\def\ctsdotfill{{\tenpoint\rm\leaders\hbox{\kern3pt.\kern3pt}\hfill}}
%
% How to set a chapter entry in the toc. The first one is different
% because we don't want any stretch above it, but the others are
% allowed some stretch.
%
\def\tocctsentry#1#2#3{\relax }
%
\def\parmrelax{\relax }
\newcount\tocchapnum\tocchapnum=0
\def\tocchapterentry#1#2#3{\advance\tocchapnum by 1\goodbreak\vskip22pt
\hbox to 44pt{\hfill\ctscnfont #2\hfill}\par\nointerlineskip\vskip6pt
\hbox{\vrule height0.5pt width 44pt}\nobreak
\nointerlineskip\vskip2pt%
\leftline{\vbox{\hsize30pc\def\tocbreak{\hfill\break}%
\twelvepoint\baselineskip14pt\bf\noindent\vrule height14pt width0pt
#1\endgraf }%
\hfill\elevenpoint\bf \xrefn{chappage:#1}}\nobreak
\vskip4pt\global\let\tocchapterentry\regulartocchapterentry}
%
\def\regulartocchapterentry#1#2#3{%
\def\parmtwo{#2}%
\advance\tocchapnum by 1%
\filbreak\vskip22pt plus6pt
%\ifx#2\relax\nobreak\nointerlineskip\nobreak
\ifx\parmtwo\parmrelax\nobreak\nointerlineskip\nobreak
%\else{\ctscnfont #2}\par\nobreak\nointerlineskip\nobreak
\else\hbox to 44pt{\hfill\ctscnfont #2\hfill}\par\nobreak\nointerlineskip\nobreak
\vskip6pt\fi
\hbox{\vrule height0.5pt width 44pt}\nobreak
\nointerlineskip\vskip2pt%
%\leftline{\vbox{\hsize21pc\def\tocbreak{\hfill\break}%
\leftline{\vbox{\hsize30pc\def\tocbreak{\hfill\break}%
\twelvepoint\baselineskip14pt\bf\noindent\vrule height14pt width0pt
#1\endgraf }%
%\hfill\elevenpoint\bf #3}\nobreak
\hfill\elevenpoint\bf \xrefn{chappage:#1}}\nobreak
\vskip4pt}
%
% How to set a section entry in the toc
%
\newbox\secnumbox
\def\maketocsectionpageref#1/#2{\xrefn{secpage:#1-#2}}
\setbox\secnumbox\hbox{{\bf 00.00}\kern1em}
\def\tocsectionentry#1#2#3{%
{\def\tocbreak{\hfil\egroup\hfil
\egroup\baselineskip11pt\line\bgroup\indent\hbox to \wd\secnumbox
{\hfil}\hbox to 0.8\hsize\bgroup}
\line{\indent\hbox to \wd\secnumbox{\bf #2\hfil}%
\hbox to 0.8\hsize{#1\ctsdotfill%\xrefn{secpage:\the\tocchapnum-#1}}\hfill}}}
\expandafter\maketocsectionpageref\the\tocchapnum/{#1}}\hfill}}}
%
% How to set a subsection entry in the toc.
%
\newbox\subsecnumbox
\setbox\subsecnumbox\hbox{{\bf 0.0.0}\kern1em}
\newskip\subsecwidth
\ifsubsectionsintoc
  \def\tocsubsectionentry#1#2#3{%
  {\subsecwidth=0.8\hsize
%  \advance\subsecwidth by -\parindent
  \advance\subsecwidth by -\wd\subsecnumbox
%  \advance\subsecwidth by \wd\secnumbox
  \def\tocbreak{\hfil\egroup\hfil
  \egroup\baselineskip11pt\line\bgroup\indent\indent\hbox to \wd\subsecnumbox
  {\hfil}\hbox to \subsecwidth\bgroup}
  \line{\indent\indent\hbox to \wd\subsecnumbox{\bf
  #2\hfil}\hbox to \subsecwidth{#1\ctsdotfill \xrefn{subsecpage:#1}}\hfill}}}
\else
  \def\tocsubsectionentry#1#2#3{\relax}
\fi
%
\newdimen\briefwidth
\briefwidth\hsize\advance\briefwidth by -\wd\chapnumbox
%
% Brief Contents
%
\def\beginbriefcts{\vskip-26pt\bgroup\twelvepoint\baselineskip18pt%
\parindent0pt
\def\tocchapterentry##1##2##3{%
{%
 \def\tocbreak{\hfil\egroup\hfil
 \egroup\baselineskip14pt%
 \line\bgroup\hbox to \wd\chapnumbox{\hfil}%
 \hbox to \briefwidth\bgroup\bf}
 %
 \line{\hbox to \wd\chapnumbox{\hfil\bf ##2\ifx##2\relax\relax
  \else.\fi\enspace}%
% \hbox to \briefwidth{\bf ##1\hfill\elevenpoint\bf ##3}\hfill}%
 \hbox to \briefwidth{\bf ##1\hfill\elevenpoint\bf \xrefn{chappage:##1}}\hfill}%
}}
%
\let\tocctsentry\tocchapterentry
%
\def\tocsectionentry##1##2##3{\relax}%
\def\tocsubsectionentry##1##2##3{\relax}%
}
%
%%%%%%%%%%%%%%%%%%%%%%%%%%%%%%%%%%%%%%%%%%%%%%%%%%%%%%%%%%%%%%%%%%%%%
%
% Index
%
% We generate two index files: a normal index and a ``theorem'' index.
% The macros for theorem-like objects and figures write an entry for
% each object in the tindex. This can be processed by makeindex and
% used to create a separate index of all such items (as an appendix, say)
% or it may be ignored.
%
%\makeatletter
%
%\@djustvsize={\number@fcolumns\baselineskip}
%\l@stcolumnfill 0pt plus1fill
%
\makeatletter
\@idxitemindent21pt
\subitemindent\@idxitemindent \divide\subitemindent by 3
\subsubitemindent\subitemindent \multiply\subsubitemindent by 2
\makeatother
%
\makeindex
\maketindex
%
\newbox\letterbox\newdimen\tempdepth
%
\def\firstitem#1{\setbox\letterbox\hbox{\idxfont\uppercase{#1}}
\tempdepth\dp\letterbox
\advance\tempdepth by \ht\letterbox
\advance\tempdepth by -10pt
\dp\letterbox=\tempdepth\ht\letterbox=10pt
\leftline{\box\letterbox}\nobreak%
\vskip 8pt plus 2pt minus2pt\nobreak\let\item\regularitem\item #1}
%
\def\beginletteredindex{\letteredindextrue\beginindex}
%
\makeatletter
\def\beginindex{\vskip-24pt\begingroup\let\item\@idxitem\topskip10pt
%
%
\@djustvsize={\number@fcolumns\baselineskip}
\l@stcolumnfill 0pt plus1fill
\normalbottom
\def\pagecontents{\ifvoid\topins\else\unvbox\topins\fi
  \dimen@=\dp\@cclv \unvbox\@cclv % open up \box255
  \ifvoid\footins\else % footnote info is present
    \vskip\skip\footins
    \vfill
    \footnoterule
    \unvbox\footins\fi
    \kern-\dimen@\vss}
\ifletteredindex
\def\indexspace{\par \vskip 10pt plus 2pt minus2pt\let\item\firstitem}
\let\item\firstitem\let\regularitem\@idxitem
\else
\def\indexspace{\par\vskip 10pt plus10pt minus2pt}
\fi
%
% we need to ignore the begin/end that makeindex puts into 
% the index files.
%
\def\begin##1{}\def\end##1{}
%
% A high penalty for hyphens in the index
%
\hyphenpenalty1000\normalbottom
\ninepoint\baselineskip10pt plus0.2pt minus0.1pt
%
\doublecolumns
%
% we set up a more extreme version of raggedright for the index
%
\rightskip0pt plus4em \spaceskip=.3333em\xspaceskip=.5em
%
\parskip0pt\parindent0pt\everypar{}
}
\makeatother
%
\def\endindex{\flushall\singlecolumn\endgroup\letteredindexfalse}
%
% To get a three column index, use \beginindex\tricol
%
\def\tricol{%
\endcolumns%\vskip-24pt
\triplecolumns\rightskip0pt plus4em \spaceskip=.3333em\xspaceskip=.5em
\parskip0pt\parindent0pt\everypar{}}
%
%%%%%%%%%%%%%%%%%%%%%%%%%%%%%%%%%%%%%%%%%%%%%%%%%%%%%%%%%%%%%%%%%%%%%
%
% Cross references
%
% Eplain provides only limited cross-referencing, while we want to 
% cross-reference theorems, examples, exercises, lemmas...so we have 
% to roll our own. The code is almost identical to eplain's page and 
% equation cross reference macros.
%
% Eplain provides a single macro to cite any reference. Use
%      \xrefn{thm:name}  or \xrefn{sec:name} or whatever to get the number
% of the object.
%
%%%%%%%%%%%%%%%%%%%%
%
% xrefs for theorem-like objects; suggested usages:
%       \thmrdef{thm:some good name for theorem}
%       \thmrdef{lem:some good name for lemma}
%
% Since we number lemmas, theorems, etc, in one sequence, this will 
% do for all.
%
\makeatletter
\readauxfile
\def\thmrdef#1{%
  \@readauxfile
  \begingroup
    \xrlabel{#1}%
    \edef\@wr{\@writethmrdef{\the\xrlabeltoks}}%
    \@wr
  \endgroup
  \special{html:<a name="\the\xrlabeltoks">}\special{html:</a>}%
  \ignorespaces
}%
\def\@writethmrdef#1{%
  \@writeaux{%
    \string\gdef\expandafter\string\csname#1\endcsname {\the\chapsymbol.\the\secnum.\the\thmnumber}%
  }%
}%
%
% xrefs for exercises; suggested usage:
%          \exrdef{ex:name for exercise}
%
\def\exrdef#1{%
  \@readauxfile
  \begingroup
    \xrlabel{#1}%
    \edef\@wr{\@writeexrdef{\the\xrlabeltoks}}%
    \@wr
  \endgroup
  \special{html:<a name="\the\xrlabeltoks">}\special{html:</a>}%
  \ignorespaces
}%
\def\@writeexrdef#1{%
  \@writeaux{%
    \string\gdef\expandafter\string\csname#1\endcsname {\the\questnum}%
  }%
}%
%
% xrefs for sections; usage:
%          \secrdef{sec:name of section}
%
\def\secrdef#1{%
  \@readauxfile
  \begingroup
    \xrlabel{#1}%
    \edef\@wr{\@writesecrdef{\the\xrlabeltoks}}%
    \@wr
  \endgroup
  \special{html:<a name="\the\xrlabeltoks">}\special{html:</a>}%
  \ignorespaces
}%
\def\@writesecrdef#1{%
  \@writeaux{%
    \string\gdef\expandafter\string\csname#1\endcsname
     {\the\chapsymbol.\the\secnum}%
  }%
}%
%
% xrefs for subsections; usage:
%          \subsecrdef{subsec:name of subsection}
%
\def\subsecrdef#1{%
  \@readauxfile
  \begingroup
    \xrlabel{#1}%
    \edef\@wr{\@writesubsecrdef{\the\xrlabeltoks}}%
    \@wr
  \endgroup
  \special{html:<a name="\the\xrlabeltoks">}\special{html:</a>}%
  \ignorespaces
}%
\def\@writesubsecrdef#1{%
  \@writeaux{%
    \string\gdef\expandafter\string\csname#1\endcsname
     {\the\chapsymbol.\the\secnum.\the\subsecnum}%
  }%
}%
%
% xrefs for chapters; usage:
%            \chaprdef{chap:chapter name}
%
\def\chaprdef#1{%
  \@readauxfile
  \begingroup
    \xrlabel{#1}%
    \edef\@wr{\@writechaprdef{\the\xrlabeltoks}}%
    \@wr
  \endgroup
  \special{html:<a name="\the\xrlabeltoks">}\special{html:</a>}%
  \ignorespaces
}%
\def\@writechaprdef#1{%
  \@writeaux{%
    \string\gdef\expandafter\string\csname#1\endcsname {\the\chapsymbol}%
  }%
}%
%
% xrefs for figures; usage:
%            \figrdef{fig:figure name}
%
\def\figrdef#1{%
  \@readauxfile
  \begingroup
    \xrlabel{#1}%
    \edef\@wr{\@writefigrdef{\the\xrlabeltoks}}%
    \@wr
  \endgroup
  \special{html:<a name="\the\xrlabeltoks">}\special{html:</a>}%
  \ignorespaces
}%
\def\@writefigrdef#1{%
  \@writeaux{%
    \string\gdef\expandafter\string\csname#1\endcsname
    {\the\chapsymbol.\the\secnum.\the\fignum}%
  }%
}%
%
% xrefs for equations; usage:
%            \eqrdef{eq:equation name}
%
\def\eqrdef#1{%
  \global\advance\eqnumber by 1%
  \@readauxfile
  \begingroup
    \xrlabel{#1}%
    \edef\@wr{\@writeeqrdef{\the\xrlabeltoks}}%
    \@wr
  \endgroup
  \special{html:<a name="\the\xrlabeltoks">}\special{html:</a>}%
  \ignorespaces
}%
\def\@writeeqrdef#1{%
  \@writeaux{%
    \string\gdef\expandafter\string\csname#1\endcsname
    {\the\chapsymbol.\the\secnum.\the\eqnumber}%
  }%
}%
%
% xrefs for pagenumbers---the same as xrdef from eplain
%
\let\pagerdef\xrdef
%
\makeatother
%
%%%%%%%%%%%%%%%%%%%%%%%%%%%%%%%%%%%%%%%%%%%%%%%%%%%%%%%%%%%%%%%%%%%%%%%%%%%
%
% Title page
%
\newtoks\booktitle
\newtoks\booksubtitle
\newtoks\author
\newtoks\institution
\newtoks\department
%
\def\titlepage{
\ifromanintro\pageno-1\fi
\null\vskip30pt
\hrule
\vskip24pt
\centerline{\vbox{%
 \def\titlebreak{\break}\vfil\rightskip0pt plus1fill
 \parskip0pt\parindent0pt\parfillskip0pt\baselineskip30truept
 \leftskip0pt plus1fill{\booktitlefont\the\booktitle}\break%
 {\booksubtitlefont\the\booksubtitle}\vfil}}
\vskip24pt\hrule
\vskip1truein
\centerline{\authorfont\the\author}
\vskip3truept
\ifthesis
\bsk
\centerline{A thesis submitted in partial fulfillment of requirements}
\centerline{for graduation with Honors in \the\department}
\vfill
\centerline{\institutionfont\the\institution}
\centerline{\institutionfont\the\year}
\else
\centerline{\institutionfont\the\institution}
\vfill
\fi
\blankpagetrue
%
% Put in the AW information only on camera-ready title pages
%
\eject
}
%
\def\pubdata{%
\ifcameraready
  \input pubdata
  \global\blankpagetrue
\else
  \input copyright
  \global\blankpagetrue
\fi
\eject}
%
%%%%%%%%%%%%%%%%%%%%%%%%%%%%%%%%%%%%%%%%%%%%%%%%%%%%%%%%%%%%%%%%%%%%%%%%%%%
%
% Chapters, sections, subsections
%
\newcount\chapnum \chapnum=0
\newcount\appendixnum\appendixnum0
\newcount\secnum
\newcount\subsecnum
\newtoks\chapsymbol                 % The number for a chapter, letter
                                    % for an appendix
\def\chapname{Chapter }
\def\idxtitle{Index}
\def\unletteredname{unlettered}
\def\nothing{\relax}
%\let\testnothing=\nothing\show\testnothing
%
% Every chapter should start with \chapter{Title}
%
\newdimen\chaprulewidth\newdimen\chapruleindent
\newdimen\chapruledepth
\ifcameraready
\chapruleindent18pt
\else
\chapruleindent72pt
\fi
%
\chaprulewidth\hsize\advance\chaprulewidth by -2\chapruleindent
\chapruledepth0pt
%
\def\bookmarkc#1{\special{pdf:out 1 << /Title (#1) /Dest [ @thispage /FitH @ypos ] >>}}
\def\bookmarks#1{\special{pdf:out 2 << /Title (#1) /Dest [ @thispage /FitH @ypos ] >>}}
%
\def\chapter#1{
          \vfill\supereject              % new chapter starts on a new page
          \ifodd\pageno\else
            \ifprintblankpages\ \global\blankpagetrue
            \else\global\blankpagefalse\global\advancepageno
            \fi
            \vfill\eject\fi % start on an odd page
          \ifnum\chapnum > -1 
                \advance\chapnum by 1 \secnum=0 \fignum=1 \eqnumber=0
                \bookmarkc{\the\chapnum. #1}\chapsymbol={\the\chapnum}
                \ifromanintro\ifnum\chapnum=1\pageno=1\fi\fi % stop roman numeral page numbers
                \def\sectiontitle{}\def\chaptertitle{#1}
                \null\ifx\chaptertitle\idxtitle\vskip75truept
                \else\vskip33truept
                \centerline{\chapnumberfont\the\chapsymbol}
%                \immediate\write\solns{\string\chapter{#1}}
                \unexpandedwrite\solns{\chapter{#1}}
                \vskip1truepc\fi
                \nointerlineskip\line{\kern\chapruleindent\vrule height0truept
                  depth\chapruledepth width\chaprulewidth\hfill}\nointerlineskip
                \centerline{\vbox to 5.5truepc{%
                  \def\titlebreak{\break}\vfil\rightskip0pt plus1fil
                  \parskip0pt\parindent0pt\parfillskip0pt\baselineskip26truept
                  \leftskip0pt plus1fil{\chaptitlefont #1}\vfil}}
                \nointerlineskip\line{\kern\chapruleindent\vrule height0truept
                  depth\chapruledepth width\chaprulewidth\hfill}\nointerlineskip
                \vskip112truept
          \else % chapter number < 0
                \advance\chapnum by 1 \secnum=0
                \bookmarkc{#1}\def\chaptertitle{#1}
                \null\vskip75truept
                \nointerlineskip\line{\kern\chapruleindent\vrule height0truept
                  depth\chapruledepth width\chaprulewidth\hfill}\nointerlineskip
                \centerline{\vbox to 5.5truepc{%
                  \def\titlebreak{\break}\vfil
                  \centerline{\chaptitlefont #1}\vfil}}
                \nointerlineskip\line{\kern\chapruleindent\vrule height0truept
                  depth\chapruledepth width\chaprulewidth\hfill}\nointerlineskip
                \vskip112truept
          \fi
          \footnotenumber=0\thmnumber=0
          \global\chapopenertrue\global\chapstarttrue
%
% We don't put prefatory chapters into the toc, but it would be easy
% to do so here.
%
          \ifnum\chapnum > 0
                  \ifx\chaptertitle\idxtitle
                    % \nothing is a necessary kludge
                    \writenumberedtocentry{chapter}{#1}{\nothing}
                  \else
                    \writenumberedtocentry{chapter}{#1}{\the\chapsymbol}
                  \fi
          \fi
          \parindent0pt
          \mark{{\the\secnum}{\null}}
          \pagerdef{chappage:#1}
          }
%
\def\appendix#1{
          \vfill\supereject             % new chapter starts on a new page
          \global\def\chapname{Appendix }
          \ifodd\pageno\else
            \ifprintblankpages\ \global\blankpagetrue
            \else\global\blankpagefalse\global\advance\pageno by 1
            \fi
            \vfill\eject\fi % start on an odd page
          \advance\chapnum by 1%
          \advance\appendixnum by 1 \secnum=0 \fignum=1
          \def\sectiontitle{}\def\chaptertitle{#1}
          \makeappendixletter
          \edef\realsymbol{\the\chapsymbol}
          \bookmarkc{\realsymbol. #1}
          \null\vskip32truept
          \centerline{\chapnumberfont\the\chapsymbol}
          \vskip1truepc
          \nointerlineskip\line{\kern\chapruleindent\vrule height0truept
            depth\chapruledepth width\chaprulewidth\hfill}\nointerlineskip
          \centerline{\vbox to 5.5truepc{%
                  \def\titlebreak{\break}\vfil\rightskip0pt plus1fil
            \parskip0pt\parindent0pt\parfillskip0pt\baselineskip26truept
            \leftskip0pt plus1fil{\chaptitlefont #1}\vfil}}
          \nointerlineskip\line{\kern\chapruleindent\vrule height0truept
            depth\chapruledepth width\chaprulewidth\hfill}\nointerlineskip
          \vskip112truept
          \thmnumber=0\footnotenumber=0
          \global\chapopenertrue\global\chapstarttrue
          \writenumberedtocentry{chapter}{#1}{\the\chapsymbol}
          \parindent0pt
          \pagerdef{chappage:#1}
          }
%
\def\unletteredappendix#1{
          \vfill\supereject              % new chapter starts on a new page
          \global\def\chapname{unlettered}
          \ifodd\pageno\else
              \ifprintblankpages\ \global\blankpagetrue
              \else\global\blankpagefalse\global\advance\pageno by 1
              \fi
              \vfill\eject
          \fi % start on an odd page  
          \bookmarkc{#1}
          \advance\appendixnum by 1 \secnum=0 \fignum=1
          \def\sectiontitle{}\def\chaptertitle{#1}\chapsymbol{#1}
          \null\vskip75truept
          \nointerlineskip\line{\kern\chapruleindent\vrule height0truept
            depth\chapruledepth width\chaprulewidth\hfill}\nointerlineskip
          \centerline{\vbox to 5.5truepc{%
                  \def\titlebreak{\break}\vfil\rightskip0pt plus1fil
            \parskip0pt\parindent0pt\parfillskip0pt\baselineskip26truept
            \leftskip0pt plus1fil{\chaptitlefont #1}\vfil}}
          \nointerlineskip\line{\kern\chapruleindent\vrule height0truept
            depth\chapruledepth width\chaprulewidth\hfill}\nointerlineskip
          \vskip112truept
          \thmnumber=0\footnotenumber=0
          \global\chapopenertrue\global\chapstarttrue
          \writenumberedtocentry{chapter}{#1}{\nothing}
          \parindent0pt
          \pagerdef{chappage:#1}
          }
%
\def\makeappendixletter{\chapsymbol=
  {\ifcase\appendixnum\or A\or B\or C\or D\or E\or F\or G\or H\or I\or J\or K
  \or L\or M\or N\or O\or P\or Q\or R\or S\or Y\or U\or V\or W\or X\or
  Y\or Z
\fi}}
%
\def\empty{}
%
\newbox\secnumberbox
\def\makesecpagedef#1/#2{\pagerdef{secpage:#1#2}}
\def\section#1#2#3{\advance\secnum by 1
               \unexpandedwrite\solns{\section{#1}}
%               \immediate\write\solns{\string\section{\tempsectionname}}
               \def\sectiontitle{#1}\subsecnum=0
               \ifchapstart
                 \vskip-2pt
               \else
               \ifexerciseflag
                 \ifdim\abovesectionskip>\afterexerciseskip
                   \vskip-\afterexerciseskip
                   \vskip\abovesectionskip
                 \fi
               \else\ifproofflag
                 \ifdim\abovesectionskip>\afterproofskip
                   \vskip-\afterproofskip
                   \vskip\abovesectionskip
                 \fi
               \else\iflistflag
                 \ifdim\abovesectionskip>\afterlistskip
                   \vskip-\afterlistskip
                   \vskip\abovesectionskip
                 \fi
               \else
                   \vskip\abovesectionskip
               \fi\fi\fi
               \fi
               \def\runningsectiontitle{#2}
               \ifx\runningsectiontitle\empty\def\runningsectiontitle{#1}\fi
               \def\bookmarkstitle{#3}
               \ifx\bookmarkstitle\empty\let\bookmarkstitle\runningsectiontitle\fi
               {%
                  \setbox\secnumberbox\hbox{\secfont
                    \the\chapsymbol.\the\secnum\kern1em}
                  \def\titlebreak{\hfil\egroup\nobreak\baselineskip12pt%
                    \line\bgroup\hbox to \wd\secnumberbox{\hfil}\secfont }
                  \goodbreak\bookmarks{\the\secnum. \bookmarkstitle}%
                    \line{\copy\secnumberbox\secfont #1\hfil%
                    \writenumberedtocentry{section}{#1}
                    {\the\chapsymbol.\the\secnum}}%
               }%
               \edef\setsecpagerdef{\the\chapnum-}
               \expandafter\makesecpagedef\setsecpagerdef/{#1}%
               \fignum=1\thmnumber=0\eqnumber=0%\pagerdef{secpage:#1}
               \mark{{\the\secnum}{\runningsectiontitle}}\nobreak\vskip6pt\nobreak\parindent0pt}
%
\newbox\subsecnumberbox
\def\subsection#1{\advance\subsecnum by 1
               \def\subsectiontitle{#1}
               \ifexerciseflag
                 \ifdim\abovesubsectionskip>\afterexerciseskip
                   \vskip-\afterexerciseskip
                   \vskip\abovesubsectionskip
                 \fi
               \else\ifproofflag
                 \ifdim\abovesubsectionskip>\afterproofskip
                   \vskip-\afterproofskip
                   \vskip\abovesubsectionskip
                 \fi
               \else\iflistflag
                 \ifdim\abovesubsectionskip>\afterlistskip
                   \vskip-\afterlistskip
                   \vskip\abovesubsectionskip
                 \fi
               \else
                   \vskip\abovesubsectionskip
               \fi\fi\fi
               {%
                  \setbox\subsecnumberbox\hbox{\subsecfont
                    \the\chapsymbol.\the\secnum.\the\subsecnum\kern1em}
                  \def\titlebreak{\hfil\egroup\nobreak\baselineskip12pt%
                    \line\bgroup\hbox to \wd\subsecnumberbox{\hfil}%
                    \subsecfont }
                  \line{\copy\subsecnumberbox\subsecfont #1\hfil%
                    \writenumberedtocentry{subsection}{#1}
                    {\the\chapsymbol.\the\secnum.\the\subsecnum}}%
               }%
               \pagerdef{subsecpage:#1}
               \nobreak\vskip6pt\nobreak\parindent0pt}
%
%
%%%%%%%%%%%%%%%%%%%%%%%%%%%%%%%%%%%%%%%%%%%%%%%%%%%%%%%%%%%%%%%%%%%%%%%%%%%
%
% Exercises 
%
\newcount\questnum
\def\question{\global\advance\questnum by 1 \bf\the\questnum .}
%
% \def\items#1#2{\par\noindent \hangindent48pt
% \hbox to \parindent{\hss{\bf #1}\enspace}\ignorespaces
% \hbox to 16pt{\hss{\bf #2}\enspace}\ignorespaces}
%
% \newdimen\secondindent
% \secondindent\exerciselistindentamount\advance\secondindent by -\parindent
% \def\items#1#2{\par\noindent \hangindent\exerciselistindentamount
% \hbox to \parindent{\hss{\bf #1}\enspace}\ignorespaces%
% \hbox to \secondindent{\hss{\bf #2}\enspace}\ignorespaces}
%
% \def\doubleitem#1#2#3#4{\par\noindent\indent\llap{{\bf #1}\enspace}\hbox to
% 2in{#2\hfill}\llap{{\bf #3}\enspace}#4}
\newdimen\halfhsize
\def\doubleitem#1#2#3#4{\par\halfhsize\hsize\divide\halfhsize by 2%
%\advance\halfhsize by -30pt%
\noindent\indent\llap{{\bf #1}\enspace}\hbox to
\halfhsize{#2\hfill}\llap{{\bf #3}\enspace}#4}
%
\def\tripleitem#1#2#3#4#5#6{\par\noindent\indent
\llap{{\bf #1}\enspace}\hbox to 1.5in{#2\hfill}%
\llap{{\bf #3}\enspace}\hbox to 1.5in{#4\hfill}%
\llap{{\bf #5}\enspace}#6}
%
\newbox\leftbox
\newbox\leftno
\newbox\rightno
\newbox\rightbox
\newif\ifleftcol

\def\twocol{\bgroup\global\leftcoltrue
\def\exercise{\par\ifleftcol
\global\setbox\leftno\hbox{\question}
\global\setbox\leftbox=\hbox\bgroup\def\par{\egroup
\global\leftcolfalse}
\else
\setbox\rightno\hbox{\question}
\setbox\rightbox=\hbox\bgroup\def\par{\egroup
\global\leftcoltrue
\doubleitem{\unhbox\leftno}{\unhbox\leftbox}{\unhbox\rightno}{\unhbox\rightbox}}
\fi}}
\def\endtwocol{\par\egroup
\ifleftcol\else
\doubleitem{\unhbox\leftno}{\unhbox\leftbox}{}{}
\fi}
%
\def\exercise{\par\parindent20pt
\ifexerciseremarkflag
  \ifdim\exerciseseparationskip>\afterexerciseremarkskip
    \vskip-\afterexerciseremarkskip
  \else
    \vskip-\exerciseseparationskip
  \fi
\else\iflistflag
    \vskip-\afterexerciselistskip
\fi\fi
\item{\question}}
%
\newif\ifexhead
\exheadtrue
\def\exercises{\futurelet\lookahead\xcheck}
\def\swallowstar#1{\xercises}
\def\xcheck{\if\lookahead*\global\exheadfalse\let\next\swallowstar\else%
        \global\exheadtrue\let\next\xercises\fi\next}
\def\xercises{\questnum0
      \ifexerciseflag
        \ifdim\beforeexerciseskip>\afterexerciseskip
          \vskip-\afterexerciseskip
          \vskip\beforeexerciseskip
        \fi
      \else\ifproofflag
        \ifdim\beforeexerciseskip>\afterproofskip
          \vskip-\afterproofskip
          \vskip\beforeexerciseskip
        \fi
      \else\iflistflag
        \ifdim\beforeexerciseskip>\afterlistskip
          \vskip-\afterlistskip
          \vskip\beforeexerciseskip
        \fi
      \else
          \vskip\beforeexerciseskip
      \fi\fi\fi
%      \vskip 0pt plus 50pt
%      \penalty-500
       \goodbreak
      \ifexhead\leftline{\exerciseheadfont Exercises \the\chapsymbol.\the\secnum.}%
      \nobreak\vskip4truept\fi\nobreak\bgroup
      \iftimes\ninehalfpoint\else\ninepoint\fi
      \let\beginlist\beginexerciselist
      \let\endlist\endexerciselist
      \let\remark\exerciseremark
      \let\endremark\endexerciseremark
      \parindent0pt
      \parskip\exerciseseparationskip
      \abovedisplayskip4pt\belowdisplayskip4pt}
%
\def\endexercises{%
\egroup\global\exerciseflagtrue\vskip\afterexerciseskip\bigbreak}
%
\def\exerciseremark#1{\par
\iflistflag\vskip-\afterexerciselistskip\fi
\vskip-\exerciseseparationskip
\vskip\beforeexerciseremarkskip
\noindent{\bf #1.}\kern1pc\bgroup\ignorespaces}
%
\def\endexerciseremark{\egroup\global\exerciseremarkflagtrue
\vskip\afterexerciseremarkskip}
%
\def\beginexerciselist{\par\nobreak\bgroup
\parindent\exerciselistindentamount
\parskip\exerciselistseparationskip
\vskip\beforeexerciselistskip
\vskip-\exerciselistseparationskip
%
\def\textindent##1{\indent\llap{{\bf ##1}\enspace}\ignorespaces}}
%
\def\endexerciselist{\par\egroup\global\listflagtrue
  \vskip\afterexerciselistskip}
\let\endexercise\relax
\def\ansno#1.#2.#3:{\smallbreak\noindent\hang
  \hbox to\parindent{\hss
  \special{html:<a name="#1.#2.#3">}\special{html:</a>}%
  \bf#1.#2.#3.\enspace}\ignorespaces}
%\def\ansno#1.#2.#3:{\smallbreak
%  \item{\bf#1.#2.#3.}\ignorespaces}
\newwrite\ans
\newwrite\solns
\newif \ifwritinganswers
\writinganswersfalse
\immediate\openout\ans=\jobname.answers % file for answers to exercises
\immediate\openout\solns=\jobname.solutions % file for answers to exercises
\immediate\write\ans{\string\relax}
\immediate\write\ans{\string\tolerance10000}
\immediate\write\ans{\string\doublecolumns}
\immediate\write\ans{\string\rightskip10pt\space plus\space 2em}
\font\msymbten miscsymbols10 at 10truept
%\outer\long\def\answer{%\par\medbreak
\long\def\answer{%\par\medbreak
  \pdflink{\the\chapnum.\the\secnum.\the\questnum}$\Rightarrow$\endpdflink
  \immediate\write\ans{}
  \immediate\write\ans{\string\ansno\the\chapnum.\the\secnum.\the\questnum:}
  \copytoendofanswer}
\long\def\solution{%
  \immediate\write\solns{}
  \immediate\write\solns{\string\ansno\the\chapnum.\the\secnum.\the\questnum:}
  \copytoendofsolution}
\outer\long\def\noanswer{%
  \immediate\write\ans{}
  \immediate\write\ans{\percent\string\ansno\the\chapnum.\the\secnum.\the\questnum:}
  \ignoretoendofanswer}
%%
\def\copytoendofanswer{\begingroup\setupcopy\copyans}
\def\copytoendofsolution{\begingroup\setupcopysolution\copysolution}
\def\ignoretoendofanswer{\begingroup\setupcopy\ignoreans}
\def\setupcopy{\def\do##1{\catcode`##1=\other}\dospecials\catcode`\^=12
  \catcode`\|=\other \obeylines}
{\catcode`\|=0\catcode`\\=\other
|gdef|closeanswer{\endanswer}}
{\obeylines \gdef\copyans#1
  {\def\next{#1}%
  \ifx\next\closeanswer\let\next=\endgroup %
  \else\immediate\write\ans{\next}\let\next=\copyans\fi\next}}
{\obeylines \gdef\ignoreans#1
  {\def\next{#1}%
  \ifx\next\closeanswer\let\next=\endgroup %
  \else\immediate\write\ans{\percent\next}\let\next=\ignoreans\fi\next}}
\def\setupcopysolution{\def\do##1{\catcode`##1=\other}\dospecials\catcode`\^=12
  \catcode`\|=\other \obeylines}
{\catcode`\|=0\catcode`\\=\other
|gdef|closesolution{\endsolution}}
{\obeylines \gdef\copysolution#1
  {\def\next{#1}%
  \ifx\next\closesolution\let\next=\endgroup %
  \else\immediate\write\solns{\next}\let\next=\copysolution\fi\next}}

%
%%%%%%%%%%%%%%%%%%%%%%%%%%%%%%%%%%%%%%%%%%%%%%%%%%%%%%%%%%%%%%%%%%%%%%%%%%%
%
% Suggested references 
%
%
\def\suggestedreferences{
               \ifexerciseflag
                 \ifdim\beforereferenceskip>\afterexerciseskip
                   \vskip-\afterexerciseskip
                   \vskip\beforereferenceskip
                 \fi
               \else\ifproofflag
                 \ifdim\beforereferenceskip>\afterproofskip
                   \vskip-\afterproofskip
                   \vskip\beforereferenceskip
                 \fi
               \else\iflistflag
                 \ifdim\beforereferenceskip>\afterlistskip
                   \vskip-\afterlistskip
                   \vskip\beforereferenceskip
                 \fi
               \else
                   \vskip\beforereferenceskip
               \fi\fi\fi
               \leftline{\referenceheadfont Suggested References}%
               \nobreak\vskip3truept\nobreak\bgroup
               \iftimes\ninehalfpoint\else\ninepoint\fi
               \parindent0pt
               \everypar{\hangafter1\hangindent\basalparindent}
               }
%
\def\endsuggestedreferences{%
\egroup\global\exerciseflagtrue\vskip\afterexerciseskip}
%
%%%%%%%%%%%%%%%%%%%%%%%%%%%%%%%%%%%%%%%%%%%%%%%%%%%%%%%%%%%%%%%%%%%%%%%%%%%
%
% Extracts 
%
\def\extract{\bgroup\advance\leftskip\extractindentamount
  \advance\rightskip\extractindentamount\parindent0pt
  \vskip\aboveextractskip\it }
\def\endextract{\egroup\global\proofflagtrue\vskip\belowextractskip}
%
%%%%%%%%%%%%%%%%%%%%%%%%%%%%%%%%%%%%%%%%%%%%%%%%%%%%%%%%%%%%%%%%%%%%%%%%%%%
%
% Lists 
%
\def\beginlist{\par\nobreak\bgroup\questnum0\parindent\listindentamount
\parskip\listseparationskip
\ifexerciseflag
  \ifdim\beforelistskip>\afterexerciseskip
    \vskip-\afterexerciseskip
    \vskip\beforelistskip
    \vskip-\listseparationskip
  \fi
\else\ifproofflag
  \ifdim\beforelistskip>\afterproofskip
    \vskip-\afterproofskip
    \vskip\beforelistskip
    \vskip-\listseparationskip
  \fi
\else\iflistflag
  \ifdim\beforelistskip>\afterlistskip
    \vskip-\afterlistskip
    \vskip\beforelistskip
    \vskip-\listseparationskip
  \fi
\else
  \vskip\beforelistskip
  \vskip-\listseparationskip
\fi\fi\fi
%
\def\textindent##1{\indent\llap{{\bf ##1}\enspace}%
\ignorespaces\checklisttitle}}
%
\def\endlist{\par\egroup\global\listflagtrue\vskip\afterlistskip}
%
\def\checklisttitle#1{\if#1(\listtitle\else#1\fi}
\def\listtitle#1){{\it #1}\enspace}
%
\def\beginclaimlist{\par\nobreak\bgroup
\ifexerciseflag
  \ifdim\beforeclaimlistskip>\afterexerciseskip
    \vskip-\afterexerciseskip
    \vskip\beforeclaimlistskip
    \vskip-\claimlistseparationskip
  \fi
\else\ifproofflag
  \ifdim\beforeclaimlistskip>\afterproofskip
    \vskip-\afterproofskip
    \vskip\beforeclaimlistskip
    \vskip-\claimlistseparationskip
  \fi
\else\iflistflag
  \ifdim\beforeclaimlistskip>\afterlistskip
    \vskip-\afterlistskip
    \vskip\beforeclaimlistskip
    \vskip-\claimlistseparationskip
  \fi
\else
    \vskip\beforeclaimlistskip
    \vskip-\claimlistseparationskip
\fi\fi\fi
%
\def\item##1{\vskip\claimlistseparationskip
\noindent{\remarksfont ##1.}\kern1pc\ignorespaces}%
}
%
\def\endclaimlist{\par\egroup\global\proofflagtrue\vskip\afterclaimlistskip}
%
%%%%%%%%%%%%%%%%%%%%%%%%%%%%%%%%%%%%%%%%%%%%%%%%%%%%%%%%%%%%%%%%%%%%%%%%%%%
%
% Output routine; supplies registration marks
%
\newcount\bighand\newcount\littlehand
\bighand=\time\divide\bighand by 60
\littlehand=\bighand\multiply\littlehand by -60
\advance\littlehand by\time
\output{\normaloutput}
\makeatletter
% Kludge to fix problem with tikz:
\newbox\my@utputbox
%
\ifcameraready
\def\normaloutput{%
  \hoffset=0pt\voffset=-0.5truein
  \ifodd\pageno\lm=\oddlm\else\lm=\evenlm\fi%
  \setbox\my@utputbox\vbox{%
      \hbox to \pagewidth{\tenpoint%
        \vrule width0.4truept height0.25truein depth0.15truein
        \quad\jobname.tex run at \the\bighand:\ifnum\littlehand<10{0}\fi
        \the\littlehand\ on \the\month/\the\day/\the\year
        \hfill\tenrm\folio\quad\vrule width0.4truept}
      \nointerlineskip\kern0.1truein
      \hbox to \pagewidth{\llap{\vrule height0.4truept 
        width0.4truein \kern0.1truein}\hfill%
      \rlap{\kern0.1truein \vrule height0.4truept width0.4truein}}
      \nointerlineskip
      \hbox{\hskip\lm\vbox to \pageheight{%
      \makeheadline\pagebody\makefootline\vfill}}
      \nointerlineskip
      \hbox to \pagewidth{\llap{\vrule height0.4truept
        width0.4truein\kern0.1truein}\hfill%
      \rlap{\kern0.1truein \vrule height0.4truept width0.4truein}}
      \nointerlineskip\kern0.1truein
      \hbox to \pagewidth{\vrule width0.4truept height0pt 
        depth0.4truein\hfill\vrule width0.4truept height0pt depth0.4truein}
    }
  \shipout\box\my@utputbox%
  \advancepageno\global\chapopenerfalse
  \ifnum\outputpenalty>-\@MM \else\dosupereject\fi}
\def\pagecontents{\ifvoid\topins\else\unvbox\topins\fi
  \dimen@=\dp\@cclv \unvbox\@cclv % open up \box255
  \ifvoid\footins\else % footnote info is present
    \vskip\skip\footins
    \vfill
    \footnoterule
    \unvbox\footins\fi
  \ifr@ggedbottom \kern-\dimen@ \vskip\bottomskip \fi}
%
\def\makeheadline{\setbox\headbox\vbox to 0.625truein{\vfill
  \line{\the\headline}}\copy\headbox\vskip-\dp\headbox\nointerlineskip}
\def\makefootline{\baselineskip24\p@\line{\the\footline}}
%
\else\ifthesis            % thesis
%
\def\normaloutput{%
%%  \shipout\vbox{\makeheadline\pagebody\makefootline}%
  \setbox\my@utputbox\vbox{\makeheadline\pagebody\makefootline}%
  \shipout\box\my@utputbox%
  \advancepageno\global\chapopenerfalse
  \ifnum\outputpenalty>-\@MM \else\dosupereject\fi}
\def\pagecontents{\ifvoid\topins\else\unvbox\topins\fi
  \dimen@=\dp\@cclv \unvbox\@cclv % open up \box255
  \ifvoid\footins\else % footnote info is present
    \vskip\skip\footins
    \vfill
    \footnoterule
    \unvbox\footins\fi
  \ifr@ggedbottom \kern-\dimen@ \vfil \fi}
\def\makeheadline{\setbox\headbox\vbox to 0truept{\vss
  \line{\the\headline}}\copy\headbox\vskip-\dp\headbox\nointerlineskip}
\def\makefootline{\baselineskip24\p@\line{\the\footline}}
%
\else
%
\def\normaloutput{%
  \ifodd\pageno\hoffset=\pageshift\else\hoffset=-\pageshift\fi%
%%  \shipout\vbox{\makeheadline\pagebody\makefootline}%
  \setbox\my@utputbox\vbox{\makeheadline\pagebody\makefootline}%
  \shipout\box\my@utputbox%
  \advancepageno\global\chapopenerfalse
  \ifnum\outputpenalty>-\@MM \else\dosupereject\fi}
%% \def\pagecontents{\ifvoid\topins\else\unvbox\topins\fi
%%   \dimen@=\dp\@cclv \unvbox\@cclv % open up \box255
%%   \ifvoid\footins\else % footnote info is present
%%     \vskip\skip\footins
%%     \vfill
%%     \footnoterule
%%     \unvbox\footins\fi
%%   \ifr@ggedbottom \kern-\dimen@ \vfil \fi}
%% \def\makeheadline{\setbox\headbox\vbox to 0truept{\vss
%%   \line{\the\headline}}\copy\headbox\vskip-\dp\headbox\nointerlineskip}
%% \def\makefootline{\baselineskip24\p@\line{\the\footline}}
\fi\fi
\makeatother
\newbox\headbox
%
%%%%%%%%%%%%%%%%%%%%%%%%%%%%%%%%%%%%%%%%%%%%%%%%%%%%%%%%%%%%%%%%%%%%%%%%%%%
%
% Theorem-like objects: They are numbered automatically,
% and they generate an index entry. To get makeindex to sort them properly,
% we need to use this kludgy "thmcounter" to control the order they go in the
% index. The theorem objects are counted consectutively throughout the book.
% They will be "invisibly numbered" starting with 1000, and will be ok as 
% long as the counter doesn't go above 9999. 
%
% Figures are counted with the invisible counter as well.
%
\newcount\thmcounter \thmcounter=999
%
% Now the actual number of the theorem, lemma, etc, in the chapter. We number
% all such objects in a single numbering scheme (but not figures).
%
\newcount\thmnumber
\newbox\highlightbox
\def\thm{\thmobject{Theorem}}
\def\rule{\thmobject{Rule}}
\def\axiom{\thmobject{Axiom}}
\def\lem{\thmobject{Lemma}}
\def\prop{\thmobject{Proposition}}
\def\cor{\thmobject{Corollary}}
\def\defn{\thmobject{Definition}}
\def\example{\thmobject{Example}}
\long\def\thmobject#1#2{%
\global\advance\thmcounter by 1\global\advance\thmnumber by 1
\ifexerciseflag
  \ifdim\beforethmskip>\afterexerciseskip
    \vskip-\afterexerciseskip
    \vskip\beforethmskip
  \fi
\else\ifproofflag
  \ifdim\beforethmskip>\afterproofskip
    \vskip-\afterproofskip
    \vskip\beforethmskip
  \fi
\else\iflistflag
  \ifdim\beforethmskip>\afterlistskip
    \vskip-\afterlistskip
    \vskip\beforethmskip
  \fi
\else
    \vskip\beforethmskip
\fi\fi\fi
%\setbox\highlightbox\vbox\begingroup
\noindent{\thmfont \uppercase{#1}
\tindex{#1 \the\thmcounter@#1 %
%\the\chapsymbol.\the\secnum.\the\thmnumber}\the\chapsymbol.\the\secnum.\the\thmnumber}\kern1pc
\the\chapsymbol.\the\secnum.\the\thmnumber}\the\chapsymbol.\the\secnum.\the\thmnumber}\hskip 1pc minus 4pt
\if#2(\thmtitle\else\bgroup#2\thmstmtfont\fi}
\def\thmtitle#1){{\thmfont #1}\kern1pc\bgroup\thmstmtfont}
%
% End proofs and unproved theorems with a box at the right margin.
% Usage: ...end of proof.~\qed
%
\newskip\saveparfillskip
\def\qed{\saveparfillskip\parfillskip%
\parfillskip0pt\penalty10000\hfill\kern1pc\vrule height5pt width5pt
depth0pt\par%
\parfillskip\saveparfillskip\egroup\global\proofflagtrue\vskip\afterproofskip}
\let\endproof\qed
\let\endthmnoproof\qed
%
% \openendbox provides an open box for the end of a theorem statement, if
% the proof does not follow immediately, and for other thm-like objects.
%
\def\openendbox{\saveparfillskip\parfillskip%
\parfillskip0pt\penalty10000\hfill\kern1pc%
\vbox{\hsize5pt
\hrule\hbox to \hsize{\vrule height4.2pt\hfill\vrule}\hrule}
\par%
\parfillskip\saveparfillskip\egroup\global\proofflagtrue\vskip\afterproofskip}
%\def\endthm{\egroup\global\proofflagtrue
%  \iflistflag
%    \endgraf\vskip-\afterlistskip
%  \fi
%  \vskip\afterproofskip}
%
\let\endthm\relax
\let\endlem\relax
\let\endthmlaterproof\openendbox
\let\enddef\openendbox
\let\endexample\openendbox
\let\endaxiom\openendbox
\let\endrule\openendbox
%
\def\proof{\par\egroup\vskip6pt\noindent{\prooffont Proof.}\kern1pc\bgroup}
%\def\proof{\vskip6pt\noindent{\prooffont Proof.}\kern1pc\bgroup}
\def\detachedproof#1{\vskip6pt\noindent{\prooffont #1}\kern1pc\bgroup}
%
%\def\solution{\par\egroup\vskip6pt\noindent{\prooffont Solution.}%
%\kern1pc\bgroup}
%
\def\remark#1{\par
\ifexerciseflag
  \ifdim\beforeclaimlistskip>\afterexerciseskip
    \vskip-\afterexerciseskip
    \vskip\beforeclaimlistskip
  \fi
\else\ifproofflag
  \ifdim\beforeclaimlistskip>\afterproofskip
    \vskip-\afterproofskip
    \vskip\beforeclaimlistskip
  \fi
\else\iflistflag
  \ifdim\beforeclaimlistskip>\afterlistskip
    \vskip-\afterlistskip
    \vskip\beforeclaimlistskip
  \fi
\else
    \vskip\beforeclaimlistskip
\fi\fi\fi
\noindent{\remarksfont #1.}\kern1pc\bgroup\ignorespaces}
%
\def\endremark{\egroup\global\proofflagtrue\vskip\afterclaimlistskip}
%
%%%%%%%%%%%%%%%%%%%%%%%%%%%%%%%%%%%%%%%%%%%%%%%%%%%%%%%%%%%%%%%%%%%%%%%%%%%
%
% Figures
%
%\makeatletter
%\def\endinsert{\egroup % finish the \vbox
%  \if@mid \dimen@\ht\z@ \advance\dimen@\dp\z@
%    \advance\dimen@\pagetotal
%    \ifdim\dimen@>\pagegoal\@midfalse\p@gefalse\fi\fi
%  \if@mid \box\z@\vskip12pt%
%  \else\insert\topins{\penalty100 % floating insertion
%    \splittopskip\z@skip
%    \splitmaxdepth\maxdimen \floatingpenalty\z@
%    \ifp@ge \dimen@\dp\z@
%    \vbox to\vsize{\unvbox\z@\kern-\dimen@}% depth is zero
%    \else \box\z@\nobreak\fi}\fi\endgroup}
%\makeatother
%
%
%\def\figure{\topinsert\vskip\topskip\hbox to \hsize\bgroup\hss}
%\def\figure{\midinsert\vskip\topskip\hbox to \hsize\bgroup\hss}
\def\figure{\midinsert\hbox to \hsize\bgroup\hss
\caption{\relax}}
%\def\endfigure#1{\hss\egroup
\def\endfigure{\hss\egroup
\vskip10pt
%\caption{#1}%\vskip-\topskip\vskip2pc%\vskip-\dp\capbox
\copy\capbox
\global\advance\thmcounter by 1\tindex{Figure \the\thmcounter@Figure %
\the\chapsymbol.\the\secnum.\the\fignum}\endinsert\global\advance\fignum by 1}
\newcount\fignum \fignum1
\newbox\capbox
\newdimen\capwidth
\capwidth\hsize\advance\capwidth by -2cm
%\def\caption#1{\global\advance\fignum by 1\setbox\capbox%
%\hbox to \hsize{\hss\ninepoint\bf Figure \the\chapsymbol.\the\fignum\kern1pc\rm
%#1\hss}\copy\capbox}
% \def\caption#1{\global\advance\fignum by 1\setbox\capbox%
% \hbox to \hsize{\hfill
% \vtop{\hbox{\bf Figure \the\chapsymbol.\the\fignum\kern1pc}}%
% \vtop{\advance\hsize by -4cm\ninepoint
% \noindent\rm
% #1}\hfill}\copy\capbox}
%% \def\caption#1{\setbox\capbox%
%% \hbox{\ninepoint\bf Figure \the\chapsymbol.\the\secnum.\the\fignum\kern1pc%
%% \rm #1}
%% \ifdim\wd\capbox>\capwidth
%% \setbox\capbox\hbox to \hsize{\hfill\ninepoint
%% \vtop{\hbox{\bf Figure \the\chapsymbol.\the\fignum\kern1pc}}%
%% \vtop{\advance\hsize by -3cm\ninepoint
%% \noindent\rm
%% #1}\hfill}\else
%% \setbox\capbox\hbox to \hsize{\hfill\unhbox\capbox\hfill}
%% \fi\copy\capbox}
\makeatletter
\def\begincaption{\c@ption}
\def\c@ption#1\endcaption{\caption{#1}}
\makeatother
\def\caption#1{\global\setbox\capbox%
\hbox{\ninepoint\bf Figure \the\chapsymbol.\the\secnum.\the\fignum\kern1pc%
\rm #1}
\ifdim\wd\capbox>\capwidth
\global\setbox\capbox\hbox to \hsize{\hfill\ninepoint
\vtop{\hbox{\bf Figure \the\chapsymbol.\the\secnum.\the\fignum\kern1pc}}%
\vtop{\advance\hsize by -3cm\ninepoint
\noindent\rm
#1}\hfill}\else
\global\setbox\capbox\hbox to \hsize{\hfill\unhbox\capbox\hfill}
\fi}
%
%%%%%%%%%%%%%%%%%%%%%%%%%%%%%%%%%%%%%%%%%%%%%%%%%%%%%%%%%%%%%%%%%%%%%%%%%%%
%
% Miscellaneous stuff
%
\gdef\titlebreak{ }
\gdef\tocbreak{}
\gdef\pubdatabreak{}
%
% Some standard symbols, using blackboard bold if
% available---otherwise \bbb will give bold face.
%
\def\A{{\bbb A}}
\def\N{{\bbb N}}
\def\Z{{\bbb Z}}
\def\Q{{\bbb Q}}
\def\R{{\bbb R}}
\def\C{{\bbb C}}
\def\U{{\bbb U}}
%
\def\frac#1/#2{\leavevmode
   \kern.1em \raise .5ex \hbox{\the\scriptfont0 #1}%
   \kern-.1em /%$/$%
   \kern-.15em \lower .5ex \hbox{\the\scriptfont0 #2}%
}
% The AMS fonts have a nice ``does not divide'' symbol; without the AMS
% fonts, we put together a passable substitute.
%
\ifamsfonts
\def\notdiv{\mathop{\nmid}}
\else
\def\notdiv{\mathord{\not\>\mid\>}}
\fi
%
% convenient abbreviations
%
\let\ssk\smallskip
\let\msk\medskip
\let\bsk\bigskip
%
% get started right
% 
\iftimes\tenhalfpoint\baselineskip\basaltextbaselineskip\fi
%
\ifcameraready\raggedbottom\else
\raggedbottom
\fi
\let\texonly\iftrue
\let\endtexonly\fi
\let\htmlonly\iffalse
\let\endhtmlonly\fi
\def\htmlfigure#1{\relax}
\def\htmlknowl#1#2{}
\let\txtbold\bf
\def\lefthead#1{\leftline{#1}}
