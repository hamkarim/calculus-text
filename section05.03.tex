\section{The second derivative test} {}{}
\nobreak
The basis of the first derivative test is that if the derivative
changes from positive to negative at a point at which the derivative
is zero then there is a local maximum at the point, and similarly for
a local minimum. If $f'$ changes from positive to negative it is
decreasing; this means that the derivative of $f'$, $f''$, might be negative,
and if in fact $f''$ is negative then $f'$ is definitely
decreasing, so there is a local maximum at the point in question. Note
well that $f'$ might change from positive to negative while $f''$ is
zero, in which case $f''$ gives us no information about the critical
value. Similarly, if $f'$ changes from negative to positive there is a
local minimum at the point, and $f'$ is increasing. If $f''>0$ at the
point, this tells us that $f'$ is increasing, and so there is a local
minimum. 

\example
Consider again $f(x)=\sin x + \cos x$, with $f'(x)=\cos x-\sin x$ and
$ f''(x)=-\sin x -\cos x$. Since $\ds f''(\pi/4)=-\sqrt{2}/2-\sqrt2/2=-\sqrt2<0$,
we know there is a local maximum at $\pi/4$. Since
$\ds f''(5\pi/4)=--\sqrt{2}/2--\sqrt2/2=\sqrt2>0$, there is a local
minimum at $5\pi/4$.
\endexample

When it works, the second derivative test is often the easiest way to
identify local maximum and minimum points. Sometimes the test fails,
and sometimes the second derivative is quite difficult to evaluate; in
such cases we must fall back on one of the previous tests.

\example
Let $\ds f(x)=x^4$. The derivatives are $\ds f'(x)=4x^3$ and
$\ds f''(x)=12x^2$. Zero is the only critical value, but $f''(0)=0$, so
the second derivative test tells us nothing. However, $f(x)$ is
positive everywhere except at zero, so clearly $f(x)$ has a local
minimum at zero. On the other hand, $\ds f(x)=-x^4$ also has zero as its
only critical value, and the second derivative is again zero, but
$\ds -x^4$ has a local maximum at zero.
\endexample

\exercises
Find all local maximum and minimum points by the second derivative
test. 

\twocol
\exercise $\ds y=x^2-x$ 
\answer min at $x=1/2$
\endanswer
\endexercise

\exercise $\ds y=2+3x-x^3$ 
\answer min at $x=-1$, max at $x=1$
\endanswer
\endexercise

\exercise $\ds y=x^3-9x^2+24x$
\answer max at $x=2$, min at $x=4$
\endanswer
\endexercise

\exercise $\ds y=x^4-2x^2+3$ 
\answer min at $x=\pm 1$, max at $x=0$.
\endanswer
\endexercise

\exercise $\ds y=3x^4-4x^3$
\answer min at $x=1$
\endanswer
\endexercise

\exercise $\ds y=(x^2-1)/x$
\answer none
\endanswer
\endexercise

\exercise $\ds y=3x^2-(1/x^2)$ 
\answer none
\endanswer
\endexercise

\exercise $y=\cos(2x)-x$ 
\answer min at $x=7\pi/12+n\pi$, max at $x=-\pi/12+n\pi$, for integer $n$.
\endanswer
\endexercise

\exercise $\ds y = 4x+\sqrt{1-x}$
\answer max at $x=63/64$
\endanswer
\endexercise

\exercise $\ds y = (x+1)/\sqrt{5x^2 + 35}$
\answer max at $x=7$
\endanswer
\endexercise

\exercise $\ds y= x^5 - x$
\answer max at $\ds -5^{-1/4}$, min at $\ds 5^{-1/4}$
\endanswer
\endexercise

\exercise $\ds y = 6x +\sin 3x$
\answer none
\endanswer
\endexercise

\exercise $\ds y = x+ 1/x$
\answer max at $-1$, min at $1$
\endanswer
\endexercise

\exercise $\ds y = x^2+ 1/x$
\answer min at $\ds 2^{-1/3}$
\endanswer
\endexercise

\exercise $\ds y = (x+5)^{1/4}$
\answer none
\endanswer
\endexercise

\exercise $\ds y = \tan^2 x$
\answer min at $n\pi$
\endanswer
\endexercise

\exercise $\ds y =\cos^2 x - \sin^2 x$
\answer max at $n\pi$, min at $\pi/2+n\pi$
\endanswer
\endexercise

\exercise $\ds y = \sin^3 x$
\answer max at $\pi/2+2n\pi$, min at $3\pi/2+2n\pi$
\endanswer

\endtwocol
\endexercise

\endexercises

