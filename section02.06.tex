% From Mike Wills
\section{More About Limits and Continuity}{}{}

When we gave the formal definition of a limit (\xrefn{def:limit}) we
assumed that $a$ and $L$ were finite. However, there are occasions
when this point of view is too limiting. We state here some limit
definitions when either $a$ or $L$ (or both) is infinite.

\defn \relax (i) Let $I$ be an open interval containing $a$. Let $f$ be a
(real-valued) function defined on $I$ except possibly at $a$.

We say that $f$ approaches $\infty $ as $x$ approaches $a$ if given
$N>0 $ there exists $\delta >0 $ such that $f(x) >N$ whenever $0<
|x-a| < \delta $.  In this situation, we write $\displaystyle{\lim
_{x\rightarrow a} f(x) }=\infty $.

We say that $f$ approaches $-\infty $ as $x$ approaches $a$ if given
 $N>0 $ there exists $\delta >0 $ such that $f(x) <-N $ whenever
 $0<|x-a| < \delta $. In this situation, we write $\displaystyle{\lim
 _{x\rightarrow a} f(x) }=-\infty $.


With some minor modifications, this definition can be modified to
one-sided limits.

 (ii) Let $I$ be an interval of the form $(b,\infty ) $. Let $f$ be a
 function defined on $I$.

 We say that $f$ approaches $L$ (a finite number) as $x$ approaches
  $\infty $ if given $\epsilon >0 $ there exists $M>0 $ such that
  $|f(x) -L |< \epsilon $ whenever $x>M $.  In this situation, we
  write $\displaystyle{\lim _{x\rightarrow \infty } f(x) }=L $.

We say that $f$ approaches $\infty $ as $x$ approaches $\infty $ if
  given $N >0 $ there exists $M>0 $ such that $f(x) > N $ whenever
  $x>M $.  In this situation, we write $\displaystyle{\lim
  _{x\rightarrow \infty } f(x) }=L $.

 We say that $f$ approaches $-\infty $ as $x$ approaches $\infty $ if
  given $N >0 $ there exists $M>0 $ such that $ f(x) < -N$ whenever
  $x>M $.  In this situation, we write $\displaystyle{\lim
  _{x\rightarrow \infty } f(x) }=L $.

 Limits as $x$ approaches $-\infty $ are defined analogously.

\enddef
\thmrdef{def:extended limits}

 \example Let $\ds f(x) ={1\over x^2} $. Then $\displaystyle{\lim
 _{x\rightarrow 0} f(x) }= 0$.
 
 Let $N>0 $. (When computing limits rigorously this is an excellent
 first statement.)  We need to find $\delta >0 $ such that if $0< |x|
 <\delta $ then $f(x) >N$.

 Now if $|x| < \delta $ then ${1\over |x|} > {1\over \delta } $ and
hence ${1\over |x|^2 } > {1\over \delta ^2} $.  By choosing $\delta
>0$ so that
$$ {1\over \delta ^2 } = N  ,
\eqrdef{eq:example choose delta}
\eqno{(\xrefn{eq:example choose delta})}
$$
the desired condition on $\delta $ will be satisfied. Solving for $\delta $ in equation \xrefn{eq:example choose delta} we obtain
$$\delta ={1\over\sqrt{N}}.$$

Thus, if $$0<|x| <{1\over \sqrt{N}} $$ then $f(x) >N $ as required.
 \endexam

 The above proof contained a lot of repetition. As one acquires a
 facility with these kinds of proofs much of the repetition can be
 dispensed with. However, there is rarely any harm in overwriting a
 proof, and you are encouraged to write a lot when asked to provide a
 proof.

%  In any case, you will only occasionally be asked to give formal
%   proofs since the focus of the calculus course here at Weber State
%   is computational. There is a time and place for serious proof
%   writing and it is called analysis. (Math 3810, 4210, and 4220 at
%   Weber State.)

\example Let $\ds f(x) ={\sin x\over x} $. Then $\displaystyle{\lim
_{x\rightarrow \infty } f(x) } =0 $.

Let $\epsilon >0 $. We need to find $M>0 $ such that $|f(x)| <
\epsilon $ whenever $x>M.$

Now if $$x>M >0$$ then $${1\over M} > {1\over x} >0.$$ Since $|\sin
x | \leq 1 $ for any real number $x$, it follows that
$$
   |f(x) | =\Big{|} {\sin x\over x} \Big{|} \leq \Big{|} {1\over x}
  \Big{|} < {1\over M}.\eqrdef{eq:example choose M}
  \eqno{(\xrefn{eq:example choose M})}
$$ 
By choosing $M ={1\over \epsilon } $ the calculation in
\xrefn{eq:example choose M} yields the desired result.
\endexam


There are several important limit laws that can aid with
calculations. We have seen one (\xrefn{thm:properties of limits});
here are some more:


\thm Suppose that $a$ is a finite number, and $f$ is a function
defined on an interval containing $a$ except (possibly) at $a$. Then
$\displaystyle{\lim _{x\to a} } $ exists if and only if both
$\displaystyle{\lim _{x\to a^+ } } $ and $\displaystyle{\lim
_{x\to a^-} } $ exist {\em and\/} are equal.
 \thmrdef{thm:more properties of limits}
\endthmnoproof

This result is frequently used to show that a limit does not exist. For example, $
 \displaystyle{\lim _{x\to 1} {1\over x-1}  }$ does not exist because the two one sided limits are not equal.

 \thm In this result, $a$ may be finite or $\pm \infty $. Suppose that
  $\displaystyle{\lim _{x\to a} f(x)
 } =\infty $,  $\displaystyle{\lim _{x\to a} g(x)
 } =\infty $, and $ \displaystyle{\lim _{x\to a} h(x)
 } =c $, where $f$, $g$, and $h$ are functions defined near $a$ and $c$ is a positive number.
 Then the following hold:

\beginlist

\item{ (i)} $\displaystyle{\lim _{x\to a} (f(x) +g(x))
 } =\infty $

\item{ (ii)} $\displaystyle{\lim _{x\to a} f(x)g(x)
 } =\infty $

\item{ (iii)} $\displaystyle{\lim _{x\to a} f(x)h(x)
 } =\infty $

\item{(iv)} $\displaystyle{\lim _{x\to a} f(x)(-g(x))
 } =-\infty $


\item{ (v)} $\displaystyle{\lim _{x\to a} f(x)(-h(x))
 } =-\infty $

\endlist
 \thmrdef{thm:yet more properties of limits}
\endthmnoproof

Notice that there was no result for $\displaystyle{\lim _{x\to a}
 (f(x) -g(x)) } $. This is because that limit may not exist, and even
 when it does, the value of the limit could be anywhere in $[-\infty ,
 \infty ] $- that is, any real number or $\pm \infty $. The expression
 $$ \displaystyle{\lim _{x\to a} f(x) } -\displaystyle{\lim _{x\to a}
 g(x) } =\infty -\infty $$ is an example of an 
{\dfont indeterminate form\index{indeterminate form}}. 
Another indeterminate form is $0\cdot \infty $. We will discuss
 such forms later when we look at
 L'H\^opital's rule.

If we replace all the limits in the preceding theorem with one-sided
limits (all approaching from the same side of $a$) the results remain
valid.

\thm (Comparison theorem)
Suppose that $f(x) \leq g(x) $ and that $\displaystyle{\lim _{x\to a} }f(x)=L $ and
 $\displaystyle{\lim _{x\to a} }g(x) =M $. Then  $L \leq M $.
\thmrdef{thm: comparison theorem}
\endthmnoproof

\thm (The squeeze theorem) \index{squeeze theorem}
Suppose that $f(x) \leq h(x) \leq g(x) $ and that
$\displaystyle{\lim _{x\to a} }f(x) =\displaystyle{\lim _{x\to a} }g(x)=L $.

Then $\displaystyle{\lim _{x\to a} }h(x) =L $.
\thmrdef{thm:the squeeze theorem}
\endthmnoproof

We've implicitly used the squeeze theorem (which follows from the comparison theorem) in several examples, mostly
 involving $\sin $ and $\cos $.

Continuity is a fundamental concept in advanced mathematics. Most of
the functions that we discuss in calculus are continuous or at least
{\dfont piecewise continuous\index{piecewise continuous}} (as defined
later in Definition \xrefn{def:piecewise continuous}).
This is a bit misleading; continuity is a rather special
condition and in some sense the probability of picking a function at
random defined on all of $\R=(-\infty, \infty )$ that is
continuous anywhere at all is zero. There are two reasons that we
usually restrict our attention to functions that are continuous. First
(and this is not really that good of a reason), they are relatively
easy to handle; second, many physical phenomena are well-approximated
by continuous functions.

We recall and extend the definition of continuity from the text:

\defn Let $I=(c,d) $ be an open interval and let $a$ be in $I$. Let
 $f:I \to \R$ be a function.  Then $f$ is continuous
 at $a$ if $\displaystyle{\lim _{x\to a} f(x) }= f(a). $

 If given any $a$ in $I$, $f$ is continuous at $a$ then $f$ is
 continuous on $I$.

 If $c$ and $d$ are finite numbers, and $f$ is defined at $c$ and $d$
  then $f$ is continuous on $[c,d] $ if $f$ is continuous on $I$,
  $\displaystyle{\lim _{x\to c^+ } f(x) }= f(c)$ and
  $\displaystyle{\lim _{x\to d^- } f(x) }= f(d)$. If $f$ is not
  continuous at $a$ then $f$ is {\dfont
  discontinuous\index{discontinuous}} at $a$.
\enddef 

For technical reasons, it is often convenient for us to define continuity
on closed intervals rather than open.

\thm If $f$ and $g$ are continuous at $a$ and $c$ is a constant,
 then $f\pm g$, $fg$, and $cf $ are all continuous at $a$. If $g(a) \neq 0 $, then
  ${f\over g} $ is also continuous at $a$.
  \endthmnoproof

The proof follows easily from the corresponding results for limits.

  \thm If $g$ is continuous at $a$ and $f$ is continuous at $g(a) $
   then $f\circ g $ is continuous at $a$.
   \endthmnoproof

Thus, the composition of continuous functions is continuous.


\example Let $P$ and $Q$ be polynomials. Then $P$ is
    continuous on all of $\R$, and $P/Q$ is continuous away from the
    roots of $Q$.

    Trigonometric functions and root functions are also continuous on
    their domains.

    In previous courses, you may have seen exponential and logarithmic
    functions. These functions are also continuous.
\endexam

It is often worthwhile considering how a function can fail to be
continuous. We give three important examples of discontinuities.

\defn Suppose that $\displaystyle{\lim _{x\to a^+ } f(x) }  =L$ and
   $\displaystyle{\lim _{x\to a^-} f(x) =M} $ with $L$ and $ M$ both finite.


 If $L=M$ but $f(a) \neq L $ (possibly because $f(a) $ is undefined)
  then $f$ has a {\dfont removable discontinuity\index{removable
  discontinuity}} 
 at $x= a$.

  If $f$ is undefined at $x=a$ and $a$ is a removable discontinuity of $f$ at $x=a$,
    a {\dfont continuous extension\index{continuous extension}} of $f$ to $a$
  is given by
$$g(x) =\cases{
f(x) & $x\neq a$,\cr
\ds\lim _{x\to a} f(x) & $x=a$\cr}
$$
\enddef

\example Note that $f(x) = {x^2 \over x} $ is undefined at
   $x=0$. However, the discontinuity is removable, and $g(x)=x $ is a
   continuous extension of $f$ to $x=0 $.
\endexam

\defn Suppose that $\displaystyle{\lim _{x\to a^+ } f(x) }  =L$ and
   $\displaystyle{\lim _{x\to a^-} f(x) =M} $ with $L$ and $ M$ both finite.  If $L\neq M$,
then $f$ has a {\dfont jump discontinuity\index{jump discontinuity}}
   at $a$.
\enddef

\example The discontinuities of the greatest integer function are jump
discontinuities.
\endexam


\defn If $f$ is discontinuous at $a$ but the discontinuity is neither
a jump discontinuity nor a removable discontinuity then the
discontinuity is {\dfont essential\index{essential discontinuity}}.
If at least one of the one-sided limits is $\pm \infty$ then the
discontinuity is {\dfont infinite\index{infinite discontinuity}}.
\enddef

\example The function $\ds f(x) =\sin (1/x)$ has an
essential discontinuity at $x=0 $.  (Try to graph $f$ on a graphing
calculator or on a computer algebra system.)
\endexam

\example If $P$ and $Q$ are non-zero polynomials,
 and $P(a) \neq 0 =Q(a) $ then ${P/Q}$ has an infinite discontinuity at $x=a $.
 \endexam

\defn Suppose that $f$ is a function defined on a closed bounded
 interval $I$. If $f$ has finitely many discontinuities, none of which
 are essential, then $f$ is 
{\dfont piecewise continuous\index{piecewise continuous}} on $I$.
\enddef
\thmrdef{def:piecewise continuous}

 \example Any continuous function on $I$ is piecewise continuous. The
 greatest integer function is piecewise continuous on any closed
 bounded interval. The function $f(x)= x/( x^2 -4)$ fails to be
 piecewise continuous on any closed bounded interval containing $2$ or
 $-2$. \endexam

Piecewise continuity plays an important role when we think about integration.

We close with an important result relating differentiability and continuity.

\thm If $f$ is differentiable at $a$ then $f$ is continuous at
$a$.

\proof We need to show that $\displaystyle{\lim _{x\to a} f(x) } =f(a) $.

Since $f$ is differentiable at $a$ we know that:
$$\eqalign{
\lim_{x\to a} {f(x)-f(a) \over x-a}  =f'(a)
   \qquad &(\hbox{Definition of derivative.})\cr
\lim _{x\to a } {f(x)-f(a) \over x-a}  = {\lim _{x\to a }f'(a) }
   \qquad & \vtop{\hbox{(Limit of a constant is}\hbox{a constant.)}}\cr
\displaystyle{\lim _{x\to a } {f(x)-f(a) \over x-a} } -  \displaystyle{\lim _{x\to a }f'(a) }
  = 0 \qquad&\cr
\displaystyle{\lim _{x\to a }\Big{(} {f(x)-f(a) \over x-a}  -f'(a) \Big{)}}
  = 0
\qquad& \vtop{\hbox{(Difference of limits is}\hbox{limit of difference.)}} \cr
 \displaystyle{\lim _{x\to a} (x-a)}
 \Big{(} \displaystyle{\lim _{x\to a } {f(x)-f(a) \over x-a}  -f'(a) \Big{)}}
 = 0\qquad&\cr
\displaystyle{\lim _{x\to a}\Big{(} (x-a) {f(x)-f(a) \over x-a}  -(x-a) f'(a) \Big{)}}
  = 0\qquad&\vtop{\hbox{(Product of limits is}\hbox{limit of product.)}}\cr 
\displaystyle{\lim _{x\to a}  \Big{(} f(x)-f(a)     -(x-a) f'(a) \Big{)} }
  = 0 \qquad&\cr
 \displaystyle{\lim _{x\to a}   \Big{(} f(x)-f(a)     -(x-a) f'(a) \Big{)} }
 = - \displaystyle{\lim _{x\to a} (x-a)f'(a) }
\qquad&
\vtop{\hbox{(Right hand side is just a}\hbox{fancy way of writing zero.)}}
 \cr \displaystyle{\lim _{x\to a}  \Big{(} f(x)-f(a)     -(x-a) f'(a) \Big{)} }
 +\displaystyle{\lim _{x\to a} (x-a)f'(a)
  }
  =0 \qquad&\cr
  \displaystyle{\lim _{x\to a}   \Big{(}f(x)-f(a) \Big{)} }   =0
\qquad&
\vtop{\hbox{(Sum of limits is}\hbox{limit of sum.)}}\cr 
\displaystyle{\lim _{x\to a}  \Big{(} f(x)-f(a) \Big{)} } +f(a)
=f(a)\qquad&\cr 
\ds\lim _{x\to a}\Big(f(x)-f(a)\Big) +\lim _{x\to a} f(a)  = f(a)
  \qquad&\vtop{\hbox{(Limit of a constant is}\hbox{a constant.)}}
 \cr \displaystyle{\lim _{x\to a}   f(x) }   =f(a)
\qquad&\vtop{\hbox{(Sum of limits is}\hbox{limit of sum.)}}\cr}$$
The last line is exactly what we wished to show.\endproof


This proof looks more complicated than it really is. The main idea is
to manipulate the various limits involved using the limit laws and
being careful not to assume a limit exists before we have shown the
existence of the limit- in other words, it is easy to get caught up in
circular reasoning if one is not careful.

We have now shown that differentiability implies continuity. The
converse is false. As a simple example, consider $f(x)= |x| $ at
$x=0$. In fact, it is possible to construct functions that are
continuous everywhere and differentiable nowhere. We shall not do so
in this course.

\exercises

Compute the following limits (without a formal proof). If the limit is
undefined, explain why.

\exercise $\ds{\lim _{x\to 1 }  {1\over x-1} } $
\endexercise

\exercise $\ds{ \lim _{x\to \infty } {\cos x\over x } }$
\endexercise

\exercise $\ds{\lim _{x\to \infty } x^3 - x } $ (Hint: Factor $x$ first.)
\endexercise

\exercise $\ds{\lim _{x\to -\infty } {1\over x^4 } }$
\endexercise

\exercise $\ds{\lim _{x\to 0^+ } {1\over x} }$
\endexercise

\exercise $\ds{\lim _{t\to \infty } (\cos ^2 t + \sin ^2 t )} $
\endexercise

\exercise Prove that $\ds{\lim _{x\to 0} } {1\over x^4 } =\infty $
\endexercise

\exercise Classify the discontinuities of $\ds f(x)  = \cases{ 1/ x & $x\neq 0$,\cr
 3 & $x=0$.\cr}$
\endexercise

\exercise Classify the discontinuities of $\ds f(x)  = {(x+1)^3\over x^2 -1}$.
\endexercise

\exercise Classify the discontinuities of $\ds f(x) = x\sin (1/x)$, $x\neq 0$.
\endexercise

\exercise Classify the discontinuities of $\ds f(x) = 
\cases{ 4-x^2   & $x\geq$  0,\cr
x^3 & $x< 0$.\cr}$
\endexercise

\exercise Explain why $\ds f(x) =\cases{ x^2\cos (1/ x^4) & $x\neq 0$\cr
0 &  $x=0$\cr}$
is continuous on all of $\R$.
\endexercise

\endexercises
